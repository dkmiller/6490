% !TEX root = 6490.tex

\section{Lie algebras}





A \emph{Lie algebra} is a linear object whose representation theory is (in 
principle) manageable. To any linear algebraic group $G$ we will associate a 
Lie algebra $\fg=\lie(G)$, and study the representation theory of $G$ via that 
of $\fg$. 





\subsection{Definition and first properties}

Fix a field $k$. Eventually we'll avoid characteristic $2$, and some theorems 
will only be valid in characteristic zero. 

\begin{definition}
A \emph{Lie algebra} over $k$ is a $k$-vector space $\fg$ equipped with a map 
$[\cdot,\cdot]:\fg\times \fg\to \fg$ such that the following hold:
\begin{itemize}
  \item $[\cdot,\cdot]$ is bilinear (it factors through $\fg\otimes\fg$). 
  \item $[x,x]=0$ for all $x\in \fg$ ($[\cdot,\cdot]$ factors through 
    $\bigwedge^2 \fg$).
  \item The Jacobi identity $[x,[y,z]]+[y,[z,x]]+[z,[x,y]]=0$ holds for all 
    $x,y,z\in \fg$.  
\end{itemize}
\end{definition}

We could have defined a Lie algebra over $k$ as being a $k$-vector space 
$\fg$ together with $[\cdot,\cdot]:\bigwedge^2 \fg\to \fg$ satisfying the 
Jacobi identity. We call $[\cdot,\cdot]$ the \emph{Lie bracket}. 

The Lie bracket satisfies a number of basic properties. For example, 
$[x,y]=-[y,x]$ because 
\begin{align*}
  0 &= [x+y,x+y] && \text{(alternating)} \\
    &= [x,x]+[x,y]+[y,x]+[y,y] && \text{(bilinear)} \\
    &= [x,y] + [y,x] . && \text{(alternating)}
\end{align*}
Later on, we'll use an alternate form of the Jacobi identity: 
\begin{equation*}\label{eq:jac-ident}\tag{$*$}
  [x,[y,z]]-[y,[x,z]] = [[x,y],z] .
\end{equation*}





\subsection{The main examples}

\begin{example}
Let $V$ be any $k$-vector space. Then the zero map $\bigwedge^2 V\to V$ 
trivially satisfies the Jacobi identity. We call any Lie algebra whose 
bracket is identically zero \emph{commutative} (or \emph{abelian}). 
\end{example}

\begin{example}[General linear]
Again, let $V$ be a $k$-vector space. The Lie algebra $\gl(V)=\End_k(V)$ as a 
$k$-vector space, with bracket $[X,Y]=X\circ Y-Y\circ X$. It is a good exercise 
(which everyone should do at least once in their life) to check that this 
actually is Lie algebra. When $V=k^n$, we write $\gl_n(k)$ instead of 
$\gl(k^n)$. 
\end{example}

Often, we will just write $\gl_n$ instead of $\gl_n(k)$. We will also do this 
for the other named Lie algebras. 

\begin{example}[Special linear]
Put $\Sl_n(k)=\{X\in \gl_n(k):\trace X=0\}$. Since $\trace [X,Y]=0$, this is a 
Lie subalgebra of $\gl_n$. In fact, $[\gl_n,\gl_n]\subset \Sl_n$. 
\end{example}

\begin{example}[Special orthogonal]
Put $\so_n(k) = \{X\in \gl_n(k):\transpose X=-X\}$. This is a Lie subalgebra 
of $\gl_n(k)$ because if $X,Y\in \so_n(k)$, then 
\begin{align*}
  \transpose{[X,Y]} 
    &= \transpose Y \transpose X-\transpose X \transpose Y \\
    &= (-Y)(-X) - (-X)(-Y) \\
    &= -[X,Y] .
\end{align*}
\end{example}

\begin{example}[Symplectic]
Let $J_n=\begin{pmatrix} & -I_n \\ I_n \end{pmatrix}\in \gl_{2n}(k)$. Put 
\[
  \fsp_{2n}(k) = \{X\in \gl_{2n}(k):J X+\transpose X\cdot J=0\} .
\]
As an exercise, check that this is a Lie subalgebra of $\gl_{2n}$. 
\end{example}

As an exercise, show that $\dR^3$ with the bracket $[u,v]=u\times v$ (cross 
product) is a Lie algebra over $\dR$. 

\begin{theorem}[Ado]
Any finite dimensional Lie algebra over $k$ is isomorphic to a Lie subalgebra 
of some $\gl_n(k)$. 
\end{theorem}
\begin{proof}
In characteristic zero, this is in \cite[I \S 7.3]{bourbaki-lie-alg-1-3}. The 
positive-characteristic case is dealt with in \cite[VI \S3]{jacobson-1979}.
\end{proof}

Another good exercise is to realize $\dR^3$ with bracket $[u,v]=u\times v$ as 
a subalgebra of some $\gl_n(\dR)$. 





\subsection{Homomorphisms and the adjoint representation}

\begin{definition}
A \emph{homomorphism} of Lie algebras $\varphi:\fg\to \fh$ is a $k$-linear map 
such that $\varphi([x,y])=[\varphi x,\varphi y]$. 
\end{definition}

If $\varphi:\fg\to \fh$ is a homomorphism of Lie algebras, then 
$\fa=\ker(\varphi)$ is a Lie subalgebra of $\fg$. This is easy: 
\begin{align*}
  \varphi[x,y]
    &= [\varphi(x),\varphi(y)] \\
    &= [0,0] \\
    &= 0 .
\end{align*}

\begin{definition}
A Lie subalgebra $\fa$ of $\fg$ is an \emph{ideal} if 
$[\fa,\fg]\subset \fa$, i.e.~$[a,x]\in \fa$ for all $a\in \fa$, $x\in \fg$. 
\end{definition}

Equivalently, $\fa\subset \fg$ is an ideal if $[\fg,\fa]\subset \fa$. If 
$\fa\subset \fg$ is an ideal, we define the \emph{quotient algebra} $\fg/\fa$ 
to be $\fg/\fa$ as a vector space, with bracket induced by that of $\fg$. So 
\[
  [x+\fa,y+\fa] = [x,y]+\fa .
\]
Let's check that this makes sense. If $a_1,a_2\in \fa$, then 
\begin{align*}
  [x+a_1,x+a_2] 
    &= [x,y] + [x,a_2]+[a_1,y]+[a_1,a_2] \\
    &\equiv [x,y] \pmod \fa .
\end{align*}

If $\varphi:\fg\to \fh$ is a Lie homomorphism, then we get an induced 
isomorphism  $\fg/\ker(\varphi)\iso \image(\varphi)\subset \fh$. 

\begin{example}
Give $k$ the trivial Lie bracket. Then $\trace:\gl_n(k) \to k$ is a a 
homomorphism. Indeed, 
\[
  \trace [X,Y]=\trace(X Y)-\trace(Y X)= 0 .
\]
We could have defined $\Sl_n(k)=\ker(\trace)$; this is an ideal in 
$\gl_n$, and $\gl_n(k)/\Sl_n(k)\iso k$. 
\end{example}

\begin{definition}
Let $\fg$ be a Lie algebra over $k$. The \emph{adjoint representation} of $\fg$ 
is the map $\adjoint:\fg\to \gl(\fg)$ defined by $\adjoint(x)(y) = [x,y]$ for 
$x,y\in \fg$. 
\end{definition}

It is easy to check that $\adjoint:\fg\to \gl(\fg)$ is $k$-linear, that is 
$\adjoint(c x)=c\adjoint(x)$ and $\adjoint(x+y)=\adjoint(x)+\adjoint(y)$. 
It is bit less obvious how the adjoint action behaves with respect to 
the Lie bracket. We compute 
\begin{align*}
  \adjoint([x,y])(z) 
    &= [[x,y],z] \\
    &=_\eqref{eq:jac-ident} [x,[y,z]]-[y,[x,z]] \\
    &= (\adjoint(x) \adjoint(y))(z) - (\adjoint(y)\adjoint(x))(z) .
\end{align*}
Thus $\adjoint[x,y] = [\adjoint(x),\adjoint(y)]$ and we have shown that 
$\adjoint:\fg\to \gl(\fg)$ is a homomorphism of Lie algebras. If 
$n=\dim_k(\fg)<\infty$, then $\adjoint:\fg\to \gl(\fg)\simeq \gl_n(k)$ is 
an interesting finite-dimensional representation of $\fg$. If 
$\dim(\fg)>1$, it cannot be surjective, and there are easy ways for it to 
fail to be injective. 

\begin{definition}
Let $\fg$ be a Lie algebra over $k$. The \emph{center} of $\fg$ is 
\[
  \zentrum(\fg) = \{x\in \fg:[x,y]=0\text{ for all }y\in \fg\} .
\]
\end{definition}

Note that $\zentrum(\fg)=\ker(\adjoint)$. There is a natural injection 
$\fg/\zentrum(\fg)\monic \gl(\fg)$. 

\begin{definition}
A Lie algebra $\fg$ is \emph{simple} if it is non-commutative, and has no 
ideals except $0$ and $\fg$. 
\end{definition}

Just as with simple algebraic groups, there is a classification theorem for 
simple Lie algebras. Even better, since Lie algebras are more ``rigid'' than 
algebraic groups, we don't have to worry about isogeny. 

\begin{theorem}
Let $k$ be an algebraically closed field of characteristic zero. Up to 
isomorphism, every Lie algebra is a member of the following list: 
\begin{center}
\begin{tabular}{c|c}
$\typeA_n$ ($n\geqslant 1$) & $\Sl_{n+1}$ \\
$\typeB_n$ ($n\geqslant 2$) & $\so_{2n+1}$ \\
$\typeC_n$ ($n\geqslant 3$) & $\fsp_{2n}$ \\
$\typeD_n$ ($n\geqslant 4$)  & $\so_{2n}$ \\
exceptional & $\fe_6$, $\fe_7$, $\fe_8$, $\ff_4$, $\fg_2$
\end{tabular}
\end{center}
\end{theorem}

The classification theorem for algebraic groups is proved via the 
classification theorem for Lie algebras, which ends up being a matter of 
combinatorics. 





\subsection{Lie algebra of an algebraic group}

For the sake of clarity, we give a scheme-theoretic definition of algebraic 
groups:

\begin{definition}
Let $S$ be a scheme. An \emph{algebraic group} over $S$ is an affine group 
scheme of finite type over $S$. 
\end{definition}

For the moment, we will say that an algebraic group $G$ over $S$ is 
\emph{linear} if $G$ is a subgroup scheme of $\GL(\sL)$ for some locally 
free sheaf $\sL$ on $S$. 

Fix a field $k$, and let $G\subset \GL(n)_{/k}$ be a linear algebraic group cut 
out by polynomials $f_1,\dots,f_r$. For any $k$-algebra $R$, we define
\[
  G(R) = \{g\in \GL_n(R):f_1(g)=\cdots = f_r(g)=0\} .
\]
This is a functor $\algebras k\to \sets$. Since $G$ is a subgroup scheme of 
$\GL(n)_{/k}$, the set $G(R)$ inherits the group structure from $\GL_n(R)$. 
So we will think of $G$ as a functor $G:\algebras k \to \groups$. By the Yoneda 
Lemma, the variety $G$ is determined by its functor of points 
$G:\algebras k\to \sets$. 

We are especially interested in the $k$-algebra of \emph{dual numbers}, 
$k[\varepsilon]=k[x]/x^2$. Informally, $\varepsilon$ should be thought of as a 
``infinitesimal quantity'' in the style of Newton and Leibniz. The scheme 
$\spectrum(k[\varepsilon])$ should be thought of as a ``point together with 
a direction.'' 

\begin{hard}
For the moment, let $k$ be an arbitrary base ring, $X_{/k}$ a scheme. We define 
the \emph{$n$-th Jet space} of $X$ to be the scheme $\jet_n X$ whose functor of 
points is $(\jet_n X)(A) = X(A[t]/t^{n+1})$. By \cite[1.8]{vojta-2007}, this 
functor is representable, so $\jet_n X$ is actually a scheme. The first jet 
space $\jet_1 X$ is called the \emph{tangent space} of $X$, and denoted 
$\tangent X$. The maps $A[t]/t^{n+1}\epic A[t]/t^n$ induce projections 
$\jet_{n+1} X\to \jet_n X$. In particular, the tangent space of $X$ comes with 
a canonical projection $\pi:\tangent X\to X$. 
\end{hard}

Consider $G(k[\varepsilon])$. As a set, this consists of invertible $n\times n$ 
matrices with entries in $k[\varepsilon]$ on which the $f_i$ vanish. The map 
$\varepsilon\mapsto 0$ from $k[\varepsilon]\to k$ induces a group homomorphism 
$\pi:G(k[\varepsilon])\to G(k)$. We will consider this map as the ``tangent 
space'' of $G(k)$. For each $g\in G(k)$, the fiber 
$\pi^{-1}(g)$ is the ``tangent space of $G$ at $g$.'' 

Note that 
\[
  \pi^{-1}(g) = \{g+\varepsilon v:v\in \matrices_n(k):f_i(g+\varepsilon v)=0\text{ for all }i\} .
\]
For each $s$, the polynomial $f_s$ has a Taylor series expansion 
\[
  f_s(x_{i j}) = f_s(g) + \sum_{i,j} \frac{\partial f_s}{\partial x_{i,j}}(g) (x_{i,j}-g_{i,j}) .
\]
Since $f_s(g)=0$, we get 
$f_s(g+\varepsilon v) = \varepsilon \sum_{i,j} \frac{\partial f_s}{\partial x_{i,j}}(g) v_{i,j}$. 
It follows that 
\[
  \pi^{-1}(g) = \left\{g+\varepsilon v:v\in \matrices_n(k):\sum_{i,j} \frac{\partial f_s}{\partial x_{i j}}(g) v_{i j}=0\text{ for all }1\leqslant s\leqslant r\right\} .
\]
We will be especially interested in $\fg=\pi^{-1}(1)$. We will turn this into a 
Lie algebra using the group structure on $G$. 

\begin{hard}
From the scheme-theoretic perspective, the identity element $1\in G(k)$ comes 
from a section $e:\spectrum(k)\to G$ of the structure $G\to \spectrum(k)$. The 
\emph{scheme-theoretic Lie algebra} of $G$ is the fiber product 
\[
  \fg = e^\ast 1 = \tangent(G)\times_G \spectrum(k) .
\]
By definition, $\fg(A)=\ker\left(G(A[\varepsilon])\to G(A)\right)$ for any 
$k$-algebra $A$. Just as with algebraic groups, we will write $\fg_{/k}$ for 
$\fg$ thought of as a group scheme 
over $k$, and just $\fg$ for the $\fg(k)$. 
\end{hard}
