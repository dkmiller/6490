% !TEX root = 6490.tex

\section{Special topics}





\subsection[Borel-Weil theorem]{Borel-Weil theorem\footnote{Balazs Elek}}

Let $k$ be a field of characteristic zero, $G_{/k}$ an algebraic group. We 
write $\modules(G)$ for the category whose objects are (not necessarily 
finite-dimensional) vector spaces $V$ together with $\rho:G\to \GL(V)$. Here, 
$\GL(V)$ is the fppf sheaf (not representable unless $V$ is finite-dimensional) 
$S\mapsto \GL(V_{\sO(S)})$. The action action of $G$ on itself by 
multiplication induces an action of $G$ on the ($k$-vector spaces) $\sO(G)$, 
which is not finite-dimensional unless $G$ is finite. By 
\cite[I 3.9]{jantzen-2003}, the category $\modules(G)$ has enough injectives; 
this enables us to define derived functors in the usual way. 

Let $H\subset G$ be an algebraic subgroup. There is an obvious functor 
$\restrict_H^G:\modules(G)\to \modules(H)$ which sends $(V,\rho:G\to \GL(V))$ 
to $(V,\rho|_H)$. It has a right adjoint, the \emph{induction functor}, 
determined by 
\[
  \hom_G(V,\induce_H^G U) = \hom_H(\restrict_H^G V,U) .
\]
We are especially interested in this when $U$ is finite-dimensional, in which 
case we have 
\[
  \induce_H^G U = \{f:G\to \dV(U):f(g h) = h^{-1} f(g)\text{ for all }g\in G\} .
\]
Since the induction functor is a right adjoint, it is left-exact, so it 
makes sense to talk about its derived functors $\eR^\bullet\induce_H^G$. It 
turns out that these can be computed as the sheaf cohomology of certain 
locally free sheaves on the quotient $G/H$. Let $\pi:G\epic G/H$ be the 
quotient map; for a representation $V$ of $H$, define an $\sO_{G/H}$-module 
$\sL(V)$ by 
\[
  \sL(V)(U) = (V\otimes \sO(\pi^{-1} U))^G .
\]
By \cite[I 5.9]{jantzen-2003}, the functor $\sL(-)$ is exact, and sends 
finite-dimensional representations of $H$ to coherent $\sO_{G/H}$-modules. 
Moreover, by \cite[I 5.12]{jantzen-2003} there is a canonical isomorphism 
\begin{equation*}\tag{$\ast$}\label{eq:induce-cohomology}
  \eR^\bullet\induce_H^G V = \h^\bullet(G/H,\sL(V)) .
\]

In general, both the vector spaces in \eqref{eq:induce-cohomology} will not be 
finite-dimensional. However, if $G/H$ is proper, then finiteness theorems 
for proper pushforward \cite[3.2.1]{ega3-i} tell us that 
$\h^\bullet(G/H,\sF)$ is finite-dimensional whenever $\sF$ is coherent. So 
we can use $\eR^i \induce_H^G$ to produce finite-dimensional representations of 
$G$ from finite-dimensional representations of $H$. 

\emph{Note}: for the remainder of this section, ``representation'' means 
\emph{finite-dimensional} representation, while ``module'' means possibly 
infinite-dimensional representation. 

Let $G_{/k}$ be a split reductive group, $B\subset G$ a Borel subgroup and 
$T\subset B$ a maximal torus. Let $N=\urad B$; one has $B\simeq T\rtimes N$. 
In particular, we can extend $\chi\in \characters^\ast(T)$ to a one-dimensional 
representation of $B$ by putting $\chi(t n) = \chi(t)$ for $t\in T$, $n\in N$. 

\begin{theorem}
Every irreducible representation of $G$ is of the form $\induce_B^G\chi$ for 
some $\chi\in\characters^\ast(T)$. 
\end{theorem}
\begin{proof}
Let $V$ be an irreducible representation of $G$. The group $B$ acts on the 
projective variety $\dP(V)$; by \ref{thm:borel-fixed} there is a fixed point 
$v$, i.e.~$\restrict_B^G V$ contains a one-dimensional subrepresentation. 
This corresponds to a $B$-equivariant map $\chi\monic \restrict_B^G V$ for some 
$\chi\in \characters^\ast(T)$. By the definition of induction functors, we get 
a nonzero map $\induce_B^G \chi \to V$. Since $V$ is simple, we have 
$V\simeq \induce_B^G(\chi)$. 
\end{proof}

One calls $\chi$ the \emph{highest weight} of $V$. A natural question is: for 
which $\chi$ is $\induce_B^G\chi$ irreducible? 

Recall that our choice of Borel $B\subset T$ induces a base 
$S\subset \roots(G,T)\subset \characters^\ast(T)$. We put an ordering on 
$\characters^\ast(T)$ by saying that $\lambda \leqslant \mu$ if 
$\mu-\lambda\in \dN\cdot S$. It is known that there exists 
$w_0\in W=\weyl(G,T)$ such that $w_0(R^+)=R^-$. Finally, the set of 
\emph{dominant weights} is: 
\[
  \characters^\ast(T)_+ = \{\chi\in \characters^\ast(T):\langle \chi,\check\alpha\rangle \geqslant 0\text{ for all }\alpha\in R^+\} .
\]
We can now classify irreducible representations of $G$. 

\begin{theorem}
Let \ldots
\end{theorem}

%Let $G=\GL_n(\dC)$, with the standard Borel and torus $(B,T)$. Let $V$ be an 
%irreducible representation of $G$. So we have a homomorphism $G\to \GL(V)$, 
%and thus an action of $B$ on $\dP(V)$. Since $B$ is solvable and $\dP(V)$ is 
%projective (hence proper), there is a fixed point $[v]$. In other words, 
%$b\cdot[v]=[v]$, which just means $b\cdot v=\lambda(v) v$ for all $b\in B$. 
%In other words, there is a character $\lambda:B\to \Gm$ such that 
%$\lambda$ is a suprepresentation of $V$. Since 
%$\characters^\ast(B)=\characters^\ast(T)$, we will say that some 
%$\lambda\in \characters^\ast(T)$ appears in $V$. The $\lambda$ so appearing 
%is called the \emph{highest weight} of $V$. 
%
%Fact. Irreducible representations of $G$ are in bijection with the 
%dominant weights of $G$. 
%
%Note that irreducible representations of $T$ are all just characters. If 
%$\chi\in \characters^\ast(G)$, write $\dC_\chi$ for $\dC$ with $T$-action 
%by $\chi$. Suppose we started with a dominant weight $\lambda$. Define 
%\[
%  L_\lambda = (G\times \dC_\lambda)/((g,v)\sim (g b,b^{-1}\cdot v) .
%\]
%This is a bundle on the flag manifold $G/B$. The fibers are lines. So 
%$L_\lambda$ is a line bundle on the flag manifold. We extend the 
%representation $\lambda:T\to \Gm$ to $B$ via $\lambda(t b)=\lambda(t)$. 
%[Levi decomposition\ldots do for general parabolic?] 
%
%Theorem. (Borel-Weil). For $\lambda$ anti-dominant, 
%$\h^0(G/B,L_\lambda)$ is the irreducible $G$-representation with highest 
%weight $w_0(\lambda)$. Here $w_0$ is the order-reversing permutation. 
%
%Bruhat decomposition: 
%\[
%  \GL_n = \bigcup_{w\in S_n} B w B .
%\]
%[Generally, $G(k)=\bigcup_{w\in W} B\dot w B$.]
%
%You prove the Borel-Weil theorem using the Bruhat decomposition. Think about 
%$s\in \h^0(U,L_\lambda)$, where $U\subset G/B$ is open. One has 
%$s(g B/B)=(g B/B,\tilde s(g))$, where $\tilde s\in \sO(p^{-1}(U))$ is such 
%that $\tilde s(g\cdot b) = \lambda(b) \tilde s(g)$. There is a unique open 
%cell $B \dot w_0 B$. You can use $B \dot w_0 B\subset G/B$ as the open set 
%and compute\ldots arrive at $\tilde s$ is determined by its value at 
%$w_0$. 
%\[
%  t\cdot \tilde s(w_0)=\tilde s(t^{-1} w_0) = \tilde s(w_0 w_0^{-1} t w_0) = \lambda(w_0^{-1} t w_0) \tilde s(w_0) = (w_0\cdot \lambda)(t)\tilde s(w_0) .
%\]




\subsection{Tannakian categories}


\subsection[Automorphisms of semisimple Lie algebras]{Automorphisms of semisimple Lie algebras\footnote{Sasha Patotski}}

Let $k$ be a field of characteristic zero, $\fg$ a split semisimple Lie algebra 
over $k$. We want to describe the group $\automorphisms(\fg)$. From [theorem], 
this is the same as $\automorphisms(G^\mathrm{sc})$, where $G^\mathrm{sc}$ is 
the unique simply connected semisimple group with Lie algebra $\fg$. We can give 
$\automorphisms(\fg)$ the structure of a linear algebraic group by putting 
\[
  \automorphisms(\fg)(S) = \automorphisms_S(\fg_S) .
\]
Clearly $\automorphisms(\fg)\subset \GL(\fg)$. Let $\ft\subset \fg$ be a maximal 
abelian subalgebra. Let $\Delta\subset \ft^\vee$ be the set of roots, 
$\Pi\subset \Delta$ a base of $\Delta$. 

Fact. Suppose $\theta\in \automorphisms(\fg)$ fixes $\ft$, i.e.~$\theta(\ft)=\ft$. 
Then $\transpose\theta\in \automorphisms(\Delta)$. 

It is not necessarily the case that $\transpose \theta(\Pi)\subset \Pi$. Let 
\[
  \automorphisms(\fg,\ft,\Pi) = \{\theta\in \automorphisms(\fg):\theta\text{ fixes }\ft\text{ and }\theta(\Pi)=\Pi\} .
\]
There is an obvious map $\eta:\automorphisms(\fg,\ft,\Pi)\to \automorphisms(\Pi)$, 
where $\automorphisms(\Pi)$ is by definition the set of automorphisms of 
$\dynkin(\Pi)$. We have $\eta(\theta)=\transpose\theta|_\Pi$. 

Fact. $\eta$ is surjective. 

\begin{proof}
We'll use the existence theorem: there exists a well-defined splitting 
$\xi:\automorphisms(\Pi)\to \automorphisms(\fg,\ft,\Pi)$, 
$\tau\mapsto \tilde\tau$. So 
$\automorphisms(\fg,\ft,\Pi) = \ker(\eta)\rtimes \automorphisms(\Pi)$. 
\end{proof}

We want to describe $\ker(\eta)$. Since $\fg$ is semisimple, the adjoint 
mapping $\adjoint:\fg\to \derivations(\fg)=\lie(\automorphisms(\fg))$ is an 
embedding. So $\adjoint(\ft)\simeq \ft$. Let $H\subset \inner(\fg)$ be the 
subgroup with Lie algebra $\adjoint(\ft)$. The claim is that $H=\ker(\eta)$. 

Pf. Clearly $H\subset \ker(\eta)$. Let $\theta\in \ker(\eta)$; we want 
$\theta=\exp(\adjoint x)$ for some $x\in \ft$. We know that $\theta|_\ft=1$. 
For some $e_\alpha\in \fg_\alpha$, we have 
$\theta e_\alpha\in \fg_\beta$ for some $\beta$. For $t\in \ft$, we have 
$[t,\theta e_\alpha] = [\theta t,\theta e_\alpha] = \theta [t,e_\alpha] = \alpha(t) \theta e_{\alpha}$. 
Thus $\alpha=\beta$. 

Use Lefschetz principle to reduce to $k=\dC$, where $\exp(-)$ makes sense. 
So $\theta e_\alpha=c_\alpha e_\alpha$. Let $x\in \ft$ be such that 
$x(e_\alpha) = \log(c_\alpha)$\ldots done. 

We've proved that $\automorphisms(\fg,\ft,\Pi) = H\rtimes \automorphisms(\Pi)$. 

Fact. $\automorphisms(\fg)=\automorphisms(\fg,\fh,\Pi)\cdot \inner(\fg)$. 

We use the fact that all Cartans are conjugate, and that the Weyl group acts 
transitively on the simple roots. Combining what we already know, we get we 
obtain 

Theorem. $\automorphisms(\fg)=\inner(\fg)\rtimes \automorphisms(\Pi)$. In 
particular, $\mathrm{Out}(\fg) = \automorphisms(\Pi)$. 

Example ($\typeA_n$). This is just $\Sl_{n+1}$. The Dynkin diagram [draw] 
has a unique automorphism (flip across the middle). So 
$\mathrm{Out}(\Sl_{n+1}) = \dZ/2$, with generator $x\mapsto -\transpose x$. 

The Dynkin diagrams $\typeB_n$ and $\typeC_n$ have no automorphisms, so 
$\mathrm{Out}(\mathfrak{so}_{2n+1})=\mathrm{Out}(\mathfrak{sp}_{2n})=1$. 

For $\typeD_n$ ($n\ne 3$) we have $\mathrm{Out}(\mathfrak{so}_{2n})=\dZ/2$, 
generated by ``conjugate by a matrix inside $\mathrm{O}(2n)$ with 
determinant $-1$.'' 

Also $\automorphisms(D_4)=S_3$ and $\automorphisms(\typeE_6)=\dZ/2$. 





\subsection[Exceptional isomorphisms]{Exceptional isomorphisms\footnote{Theodore Hui}}

[Draw $\typeA_n,\dots,\typeD_n$.]

We know that the set of isomorphism classes of Dynkin diagrams is the same as the 
set of isogeny classes of semisimple algebraic groups. 

$\typeA_1\simeq \typeB_1\simeq \typeC_1\simeq \typeD_2\simeq \typeE_1$. 

$\typeB_2\simeq \typeC_2$. 

$\typeD_3$ and $\typeA_3$. 

$\typeD_2\simeq \typeA_1\times \typeA_1$. [DYnkin diagram is $\bullet \bullet$ -- 
this is disconnected.]

We'll realize these explicitly. 

$\typeA_1\simeq \typeC_1$: the groups are the same. 

$\typeA_1\simeq\typeB_1$. 

Lemma. Suppose $\langle \cdot,\cdot\rangle$ is a bilinear non-degenerate and 
symmetric pairing on $V$ with $\dim(V)=3$. Let 
\[
  \automorphisms(V,\langle\cdot,\cdot\rangle) = \{g\in \GL(V):\langle g x,g y\rangle = \langle x,y\rangle\text{ for all }x,y\in V\} 
\]
Then $\automorphisms(V,\langle\cdot,\cdot\rangle)\simeq O(3)$. Use the 
Grahm-Schmidt (needs $k=\bar k$). 

Define $\typeA_1=\SL(2)\to \typeB_2=\SO(3)$ as follows. Let 
$V=\Sl_2=\{x\in \gl_2:\trace x=0\}$. Let $\langle x,y\rangle = \trace(x y)$ be 
the Killing form. Then $\phi:\SL(2)\to \automorphisms(V)$ is just the adjoint 
map: $\phi(g)(x) = g x g^{-1}$. The image lies in 
$\automorphisms(V,\langle \cdot,\cdot\rangle)$. Indeed, 
\begin{align*}
  \langle gx,gy\rangle 
    &= \trace(\phi(g)x \phi(g)y)\\
    &= \trace(g x g^{-1} g y g^{-1}) \\
    &= \trace(g x y g^{-1}) \\
    &= \trace(x y) \\
    &= \langle x,y\rangle .
\end{align*}
By the lemma, $\automorphisms(V,\langle\cdot,\cdot\rangle)=O(3)$. The image 
of $\SL(2)$ in $O(3)$ is connected, so it must contain $SO(3)$. A dimension 
count yields an exact sequence 
\[
  1\to \dmu_2\to SL(2)\to SO(3)\to 1 .
\]
We call this the ``double cover of $\typeB_1$ by $\typeA_1$.'' [not really 
good]

$\typeD_2=SO(4)$, $\typeA_1=SL(2)$. Our vector space is 
$V=\dC^2\otimes \dC^2$. Define $\omega:\dC^2\times \dC^2\to \bigwedge^2\dC^2=\dC$ 
by $(x,y)\mapsto x\wedge y$. Let $\langle\cdot,\cdot\rangle=\omega\otimes\omega:V\times V\to \dC$ 
be given by $\langle x_1\otimes x_2,y_1\otimes y_2\rangle = \omega(x_1,y_1)\omega(x_2,y_2)$. 
This is a symmetric nondegenerate pairing. 

Define $\varphi:SL(2)\times SL(2)\to \GL(V)$ by 
$\varphi(g,h)(v\otimes w) = (g v)\otimes (g w)$. The image of $\varphi$ lives 
inside $\automorphisms(V,\langle\cdot,\cdot\rangle)\simeq O(4)$. By connectedness, 
the image of $\varphi$ lives inside $SO(4)$. The kernel is 
$\{\pm(1,1)\}=\Delta(\dmu_2)$. By dimension counts, the map is an isogeny. 

[See Terence Tao's blog.]


\subsection[Constructing the exceptional groups]{Constructing the exceptional groups\footnote{Gautam Gopal}}

Let $k=\bar k$ be a field of characteristic zero. ($\ne 2,3$). 

Let $V$ be a vector space. Let $B:V\times V\to k$ be a symmetric bilinear form. 
We get a quadratic form $Q:V\to k$ defined by $v\mapsto B(v,v)$. THis is a bijection: 
if we started with $Q$ we could define 
\[
  B(v,w) = \frac 1 2 (Q(v+w)-Q(v)-Q(w)) .
\]

Definition. $C$ is a \emph{composition algebra} if $C$ is a not-necessarily 
associative algebra with identity element $e$, together with a non-degenerate 
quadratic form $Q$ that is multiplicative ($Q(xy)=Q(x)Q(y)$). 

Any composition algebra comes with a canonical involution. This is a map 
$C\to C$ written $x\mapsto \bar x$, defined by 
$\bar x=\langle x,e\rangle-x$. 

Theorem. Let $C$ be a composition algebra. Then $\dim(C)\in \{1,2,4,8\}$. 

[This is true over any field with characteristic $\ne 2,3$.]

An $8$-dimensional composition algebra is called an \emph{octonian 
algebra}. 

Theorem. 
Let $C$ be an octonian algebra. Then $\automorphisms(C)$ is a connected 
algebraic group of type $\typeG_2$. [automorphisms here are 
norm-preserving $k$-algebra maps.]

For $\gamma_1,\gamma_2,\gamma_3\in k^\times$, let 
\[
  \Gamma = \begin{pmatrix} \gamma^1 \\ & \gamma_2 \\ & & \gamma_3 \end{pmatrix} .
\]
Let $H_3(C,\Gamma)$ be the set 
\[
  \left\{\begin{pmatrix} z_1 & c_3 & \gamma_1^{-1} \gamma_2 \bar c_2 \\ \gamma_2^{-1} \gamma_1 \bar c_2 & z_2 & c_1 \\ c_2 & \gamma_3^{-1} \gamma_2 \bar c_1 & z_3 \end{pmatrix}:z_i\in k\text{ and }c_i\in C\right\} .
\]
One can check that 
\[
  H_3(C,\Gamma) = \{X\in M_3(C):X=\gamma^{-1} \transpose{\bar X} \Gamma\} .
\]
Put $A=H_3(C,\Gamma)$; this is called a reduced Albert algebra. 
Put $Q(X)=\frac 1 2 \trace(X^2)$. 

Def. $A$ is an \emph{Albert algebra} if $A\otimes_k L\simeq H_3(C,\Gamma)$ for some 
extension $L/k$. 

There is a cubic form 
\[
  \det(x) = z_1 z_2 z_3-\gamma_3^{-1} \gamma_2 z_1 Q(c_1) - \gamma_2^{-1} \gamma_3 z_2 Q(c_2) - \gamma_2^{-1} \gamma_1 z_3 Q(c_3)+\langle c_1 c_2,\bar c_3\rangle .
\]
This can be used in defining $\typeE_6$. 

Def. $u\in A$ is called idempotent if $u^2=u$. 

Lemma. If $u$ is an idempotent, then $Q(u)=1/2$ or $Q(u)=1$ if $u\ne 0,e$. 

If $u$ is an idempotent such that $Q(u)=1/2$, we call $u$ a primitive 
idempotent. 

Thm. If $A$ is an Albert algebra, then $G=\automorphisms(A)$ is a connected 
simple algebraic group of type $\typeF_4$. 

Idea of proof. 
Let $V$ be the set of primitive idempotents. $V$ is a closed irreducible variety 
on which $G$ acts transitively. 

$\dim(V)=16$. Pick $u\in V$, then $G_u=\stabilizer_G(u)$ is a spin group of 
a nine-dimensional quadratic form. THus 
$\dim(G_u)=36$. Thus 
$\dim(G)=16+36=52$. So at least $G$ has the right dimension. Since 
$G_u$ and $V$ are irreducible, $G$ is connected. The only possible 
semisimple group with dimension $52$ is $\typeF_4$. 

Let $W=e^\bot$ inside our Albert algebra $A$. Then $G$ acts on $W$. 
From this action, we get that $G$ is semisimple algebraic. 

[It turns out that any group of type $\typeF_4$ is obtained in this way.] 
In other words, $\h^1(k,\typeF_4)$ is the set of isomorphism classes of Albert 
algebras over $k$?

Similarly, $\h^1(k,\typeG_2)$ is the set of isomorphism classes of 
octonian algebras. 

$\typeE_6$. Let $H$ be the algebraic group of linear transformations of $A$ that 
[$A$ an Albert algebra] leave the cubic form $\det$ fixed on $A$. Then 
$H$ is of type $\typeE_6$. But this is not necessarily the only way that 
$\typeE_6$ is obtained.

