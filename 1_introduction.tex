% !TEX root = 6490.tex

\section{Introduction}





The latest version of these notes can be found online at 
\url{http://www.math.cornell.edu/~dkmiller/bin/6490.pdf}. 
Comments and corrections are appreciated. 





\subsection{Disclaimer}

These notes originated in the course MATH 6490: Linear algebraic groups and 
their Lie algebras, taught by David Zywina at Cornell University. However, the 
notes have been substantially modified since then, and are not an exact 
reflection of the content and style of the original lectures. In particular, 
the notes often switch from a somewhat elementary approach to a more 
sheaf-theoretic approach. Any errors are solely the fault of the author. 





\subsection{Notational conventions}

We follow Bourbaki in writing $\dN,\dZ,\dQ$\ldots for the natural numbers, 
integers, rationals, \ldots. The natural numbers are $\dN=\{1,2,\ldots\}$. 

If $A$ is a commutative ring, $M$ is an $A$-module, and $a\in A$, 
we write $M/a$ for $M/(a M)$. In particular, $A/a$ is the quotient of $A$ by 
the ideal generated by $a$. All abelian groups will be treated as 
$\dZ$-modules. So $A$ is an abelian group and $n\in \dZ$, the quotient 
$A/n$ means $A/n\cdot A$ even if $A$ is written multiplicatively. 

The notation $X_{/S}$ will mean ``$X$ is a scheme over $S$.'' If 
$S=\spectrum(A)$, we will write $X_{/A}$ to mean that $X$ is a scheme over 
$\spectrum(A)$. 

We write $\transpose x$ for the transpose of a matrix $x$. 

Examples are closed with a triangle $\triangleright$. 

Content that can be skipped will be delimited by a star $\star$. Usually 
this material will be much more advanced. 





\subsection{Main references}

The standard texts are \cite{borel-1991,humphreys-1975,springer-2009}. In 
these, the requisite algebraic geometry is done from scratch in an archaic 
language. A good reference for modern (scheme-theoretic) algebraic geometry is 
\cite{hartshorne-1977}, and a (very abstract) modern reference for algebraic 
groups is the three volumes on group schemes \cite{sga3-i,sga3-ii,sga3-iii} 
from the \emph{S\'eminaire de G\'eom\'etrie Alg\'ebrique}. A source lying 
somewhat in the middle is Jantzen's book \cite{jantzen-2003}. 





\subsection{A bestiary of examples}

Let $k$ be an algebraically closed field whose characteristic is \emph{not} 
$2$. For example, $k$ could be $\dC$ or $\overline{\dF_p(t)}$. For now, we 
define a \emph{linear algebraic group} over $k$ to be a subgroup 
$G\subset \GL_n(k)$ defined by polynomial equations. 

\begin{example}[General linear]
The archetypal example of an algebraic group is 
$\GL_n(k)=\{g\in \matrices_n(k):g\text{ is invertible}\}$. As a subset of 
$\GL_n(k)$, this is defined by the empty set of polynomial equations. 
\end{example}

\begin{example}[Special linear]
Let $\SL_n(k)=\{g\in \GL_n(k):\det(g)=1\}$. This is defined by the equation 
$\det(g)=1$. 
\end{example}

\begin{example}[Orthogonal]\label{eg:1st-orthogonal}
Let $\operatorname{O}_n(k)=\{g\in \GL_n(k):g \transpose g=1\}$. This is cut 
out by the equations 
\[
  \sum_{j=1}^n g_{i j} g_{k j} = \delta_{i k}
\]
for $1\leqslant i,k\leqslant n$. 
\end{example}

\begin{example}[Special orthogonal]
This is $\SO_n(k)=\operatorname{O}_n(k)\cap \SL_n(k)$. 
\end{example}

\begin{example}
This group doesn't have a special name, but for the moment we will write 
$U_n(k)$ for the group of strictly upper triangular matrices: 
\[
  U_n(k) = \begin{pmatrix} 1 & \cdots & \ast \\ & \ddots & \vdots \\ & & 1 \end{pmatrix} \subset \GL_n(k) .
\]
Recall that a matrix $g\in \GL_n(k)$ is \emph{unipotent} if $(g-1)^m=0$ for 
some $m\geqslant 1$. All elements of $U_n(k)$ are unipotent. The group $U_n(k)$ 
is defined by the equations 
\[
  \{g_{i j}=0\text{ for }j<i, g_{ii}=1\} .
\]
\end{example}

\begin{example}[Multiplicative]
We write $\Gm(k)=k^\times$ for the multiplicative group of $k$ with its obvious 
group structure. Note that $\Gm=\GL(1)$. 
\end{example}

\begin{example}[Additive]\label{eg:additive}
Write $\Ga(k)=k$, with its additive group structure. There is a natural 
isomorphism $\varphi:\Ga\iso U_2$ given by $\varphi(x)=\begin{pmatrix} 1 & x \\ & 1 \end{pmatrix}$. Since 
\[
  \begin{pmatrix} 1 & x \\ & 1 \end{pmatrix} \begin{pmatrix} 1 & y \\ & 1 \end{pmatrix} = \begin{pmatrix} 1 & x+y \\ & 1 \end{pmatrix} ,
\]
this is a group homomorphism, and it is easy to see that $\varphi$ is an 
isomorphism of the underlying varieties. 
\end{example}

\begin{example}
For any $n\geqslant 0$, $\Ga^n(k)=k^n$ with the usual addition is a linear 
algebraic group. We could embed it into $\GL_{2n}(k)$ via $2\times 2$ blocks 
and the isomorphism $\Ga\iso U_2$ in \autoref{eg:additive}.
\end{example}

\begin{example}[Tori]
For any $n\geqslant 0$, we have a \emph{torus} of rank $n$, namely 
\[
  T(k) = \begin{pmatrix} \ast \\ & \ddots \\ & & \ast \end{pmatrix}\subset \GL_n(k) .
\]
This is clearly isomorphic to $\Gm^n(k)$. We call any linear algebraic group 
isomorphic to some $\Gm^n$ a torus.  
\end{example}

This list almost exhausts the class of simple algebraic groups over an 
algebraically closed field. All of these groups make sense over 
an arbitrary field. But over non-algebraically closed fields (or even base 
rings that are not fields) thinking of algebraic groups in terms of their sets 
of points doesn't work very well. 





\subsection{Coordinate rings}

The standard references \cite{borel-1991,humphreys-1975,springer-2009} all treat 
algebraic groups in terms of their sets of points in an algebraically closed 
field. This leads to convoluted arguments, and (sometimes) theorems that are 
actually wrong. It is better to use schemes. Since we will (almost) never use 
non-affine schemes, we can study algebraic groups via their coordinate rings. 

\begin{example}
Consider $G=\GL_n(k)$. For $g=(g_{i j})\in \matrices_n(k)$, we have 
$g\in G$ if and only if $\det(g)\ne 0$. But this isn't an honest algebraic 
equation. We can remedy this by noting that $\det(g)\ne 0$ if and only if 
there exists $y\in k$ such that $\det(g)\cdot y=1$. Thus we can define the 
\emph{coordinate ring} $k[G]$ of $G$, as 
\[
  k[G] = k[x_{i j},y] / (\det(x_{i j})y-1) .
\]
For $k$-algebras $A,B$, write $\hom_k(A,B)$ for the set of $k$-algebra 
homomorphisms $A\to B$. There is a natural identification 
\[
  \hom_k(k[G],k) = \GL_n(k) .
\]
For $\varphi:k[G]\to k$, put $g_{i j} = \varphi(x_{i j})$ and 
$b=\varphi(y)$. Then $\varphi$ is well-defined exactly if 
$\det(g)\cdot b=1$. So $\varphi$ is uniquely determined by the choice of an 
invertible matrix $g\in \GL_n(k)$. Since $b$ is determined by $g$ and $g$ can 
be chosen arbitrarily in $\GL_n(k)$, this correspondence is a bijection. So in 
some sense, $k[G]$ recovers $\GL_n(k)$. 
\end{example}

Now let $k$ be an arbitrary field. For concreteness, you could think of one of 
$\dQ$, $\dR$, $\dC$, or $\dF_p$. Put $A=k[x_{i j},y]/(\det(x_{i j}) y-1)$, and 
define $\GL_n(R)=\hom_k(A,R)$ for any $k$-algebra $R$. We will think of 
``$\GL(n)_{/k}$'' as $\spectrum(A)$, which is a topological space with structure 
sheaf (essentially) $A$. The punchline is that the coordinate ring 
$k[G]$ of an algebraic group $G$ determines everything we need to know about 
$G$. 

\begin{example}
Let $k$ be a field not of characteristic $2$. For $d\in k^\times$, let 
$G_d\subset \dA_{/k}^2$ be the subscheme cut out by $x^2-d y^2=1$. In 
other words, $k[G_d]=k[x,y]/(x^2-d y^2-1)$. The group operation is 
\[
  (x_1,y_1)\cdot (x_2,y_2) = (x_1 x_2+d y_1 y_2, x_1 y_2+x_2 y_1) .
\]
We would like to realize $G_d$ as a matrix group. Consider the map 
$\varphi_d:G_d\to \GL(2)_{/k}$ given by 
$(x,y)\mapsto \mat{x}{dy}{y}{x}$. As an exercise, 
check that this is an isomorphism between $G_d$ and the subgroup 
$\{g_{11}=g_{22},x_{12}=d x_{21}\}$ of $\GL(2)_{/k}$. 

If $k=\bar k$, then consider the composite of $G_d\monic \GL(2)_{/k}\iso \GL(2)_{/k}$, 
the second map being given by 
\[
  g\mapsto \begin{pmatrix} & \sqrt d \\ 1 \end{pmatrix} g \begin{pmatrix} & \sqrt d \\ 1 \end{pmatrix}^{-1} .
\]
It sends $(x,y)\in G_d$ to $\begin{pmatrix} a & b\sqrt d \\ b\sqrt d & a \end{pmatrix}$. 
This has the same image as $\varphi_1:G_1\to \GL(2)_{/k}$. So if $k=\bar k$, then 
$G_d\simeq G_1$. 
\end{example}

\begin{hard}
Write $A_d=k[x]/(x^2-d)$. A more conceptual definition of $G_d$ is that it is 
the restriction of scalars $G_d=\weil_{A_d/k} \Gm$. That is, for any 
$k$-algebra $A$, we have $G_d(A) = \Gm(A\otimes_k A_d)$. 
\end{hard}

\begin{example}
Set $k=\dR$. We claim that $G_1\not\simeq G_{-1}$. We give a topological 
proof. The group $G_1(\dR)\subset \dA^2(\dR)$ is cut out by $x^2-y^2=1$, hence 
non-compact. But $G_{-1}(\dR)=\{x^2+y^2=1\}$ is compact. Thus 
$G_1\not\simeq G_{-1}$ over $\dR$. 
But we have seen that $G_{-1}\simeq G_1$ ``over $\dC$.'' As an exercise, 
convince yourself that $G_1\simeq \Gm$. 
\end{example}

It turns out that ``twists of $\Gm$ over $k$ up to isomorphism'' are in 
bijective correspondence with $k^\times/2$, via the correspondence 
$d\mapsto G_d$. 

\begin{hard}
This is easy to see. The $k$-forms of $\Gm$ is in natural bijection with
$\h^1(k,\automorphisms \Gm)=\h^1(k,\dZ/2)$. Kummer theory tells us that 
$\h^1(k,\dZ/2)=k^\times/2$. 
\end{hard}





\subsection{Structure theory for linear algebraic groups}\label{sec:str-thry-lag}

Let $G$ be linear algebraic group over $k$. There is a filtration of normal 
subgroups 
\begin{center}
\begin{tikzpicture}
  \matrix (m) [
    matrix of nodes,
    nodes={anchor=west}
  ] {
    & $G$ \\
    & $\cup$ & finite\\
    connected & $G^\circ$ \\
    & $\cup$ & semisimple \\
    solvable & $\rad G$ \\
    & $\cup$ & torus \\
    unipotent & $\urad G$ \\
    & $\cup$ \\
    & $1$ \\
  };
  
  \draw [decoration={brace,amplitude=0.5em},decorate]
        (m-3-2.north -| m.east) -- (m-7-2.south -| m.east) 
        node [midway,xshift=1cm] {reductive};
\end{tikzpicture}
\end{center}

Here, $G^\circ$ is the \emph{identity component} (for the Zariski topology) of 
$G$. The quotient $\pi_0(G)=G/G^\circ$ is a finite group. We write 
$\rad G$ for the \emph{radical} of $G$, namely the maximal smooth connected 
solvable normal subgroup of $G$. Finally, $\urad G$ is the \emph{unipotent 
radical} of $G$, namely the largest smooth connected unipotent subgroup of 
$G$. The quotient $G^\circ/\rad G$ is semisimple, and the quotient 
$G^\circ/\urad G$ is reductive. In general, we say a connected algebraic 
group $G$ is \emph{semisimple} if $\rad G=1$, and \emph{reductive} if 
$\urad G=1$. 

We've seen examples of tori and unipotent groups, and everybody knows plenty of 
finite groups. Here is a more direct definition of semisimple groups that 
allows us to give some basic examples. 

\begin{example}[Semisimple]
Let $k=\dC$, and let $G$ be a connected linear 
algebraic group over $k$. We say that $G$ is \emph{simple} if it is 
non-commutative, and has no proper nontrivial closed normal subgroups. We say 
$G$ is \emph{almost simple} if the only such subgroups are finite. 

For example, $\SL(2)$ is almost simple, because its only nontrivial 
closed normal subgroup is $\{\pm 1\}$. 

A group $G$ is \emph{semisimple} if we have an isogeny 
$G_1\times \cdots \times G_r\to G$ with the $G_i$ almost simple. Here an 
\emph{isogeny} is a surjection with finite kernel. 
\end{example}

Over $\dC$, the almost simple groups (up to isogeny) are: 
\begin{center}
\begin{tabular}{c|c|c}
label & group & dimension \\ \hline
$\typeA_n$ $(n\geqslant 1)$ & $\SL(n+1)$ & $n^2-1$ \\
$\typeB_n$ $(n\geqslant 2)$ & $\SO(2n+1)$ & $n(2n+1)$\\
$\typeC_n$ $(n\geqslant 3)$ & $\Sp(2n)$ & $n(2n+1)$ \\
$\typeD_n$ $(n\geqslant 4)$ & $\SO(2n)$ & $n(2n-1)$
\end{tabular}
\end{center}
We make requirements on the index in these families to prevent degenerate 
cases (e.g.~$\typeB_1=1$) or matching (e.g.~$\typeA_2=\typeC_2$). There are 
five \emph{exceptional groups} 
\begin{center}
\begin{tabular}{c|l}
label & dimension \\ \hline
$\typeE_6$ & $78$ \\
$\typeE_7$ & $133$ \\
$\typeE_8$ & $248$ \\
$\typeF_4$ & $52$ \\
$\typeG_2$ & $14$
\end{tabular}
\end{center}
Later on, we'll be able to understand why this list is complete.
