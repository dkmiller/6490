% !TEX root = 6490.tex









\section{Reductive groups}

Let $k$ be an algebraically closed field of characteristic zero. Let 
$G_{/k}$ be a reductive algebraic group. That is, $G$ is connected and 
$\urad G=1$. We will associate to $G$ some combinatorial data, called the 
\emph{root datum}. The root datum of $G$ will characterize $G$ up to 
isomorphism, will not depend on the field $k$, and will completely determine 
the representation theory of $G$. Much of the theory will depend on the 
notions of \emph{maximal torus} and \emph{Borel subgroup}. To prove basic 
properties of these, we need to be able to take quotients, and it is to this 
that we now turn. The theory of quotients requires quite a bit of, 
sophistication, and may safely be skipped at first reading. 





\subsection{Quotients \texorpdfstring{$\star$}{*}}

The ``morally correct'' place to take quotients is in the category of 
fppf sheaves. 

\begin{definition}
A finite family $\{f_i:X_i\to X\}$ of morphisms of schemes is an 
\emph{fppf cover} if it is jointly surjective (on points), and each 
$X_i\to X$ is flat, of finite presentation, and quasi-finite. 
\end{definition}

We write $\fppf$ for the topology generated by fppf covers. Recall that a 
topology is called \emph{subcanonical} if each representable functor is a 
sheaf. Also, a \emph{sieve} on $X$ is a subfunctor of $h_X=\hom(-,X)$. 
We reproduce the following theorem from SGA. 

\begin{theorem}
\leavevmode
\begin{enumerate}
\item
A sieve $S$ on $X$ is a cover for the fppf topology if and only if there is 
an open affine cover $\{X_i\}$ of $X$ together with affine fppf covers 
$\{X_{i j}\to X_i\}$ such that each $X_{i j}\to X$ is in $S$. 

\item A presheaf $F$ on $\schemes{}$ is an fppf sheaf if and only if: 
  \begin{enumerate}
  \item $F$ is a Zariski sheaf.  
  \item For each fppf cover $X\to Y$, where $X$ and $Y$ are affine, the 
    following diagram is an equalizer: 
    \[\begin{tikzcd}
      F(X) \ar[r] 
        & F(Y) \ar[r, shift left=.5ex] \ar[r, shift right=.5ex]
        & F(Y\times_X Y) .
    \end{tikzcd}\]
  \end{enumerate}
  
\item The fppf topology is subcanonical. 

\item If $\{X_i\to X\}$ is jointly surjective and each $X_i\to X$ is 
faithfully flat and of finite presehtation, then $\{X_i\to X\}$ is an fppf 
cover. 
\end{enumerate}
\end{theorem}
\begin{proof}
This is \cite[IV 6.3.1]{sga3-i}. 
\end{proof}

A good general source for topologies and sheaves is the book 
\cite{maclane-moerdijk-1994}. From it, we get that the category 
$\sheaves_\fppf(\schemes S)$ is a \emph{topos}. That is, limits, colimits 
and more exist in the category of sheaves on $\schemes S$. However, it can 
be quite tricky to check whether a given colimit of schemes (taken in the 
fppf topology) is actually represented by a scheme. 

Let $k$ be a field. For the remainder, we'll work in the category of 
fppf sheaves on $\schemes k$. We'll call such sheaves \emph{spaces}. 

\begin{definition}
Let $G$ be a $k$-group space. A $k$-space $X$ is called a \emph{$G$-space} if 
it is equipped with a morphism $G\times X\to X$, such that for each 
$S\in \schemes k$, the map $G(S)\times X(S)\to X(S)$ gives an action of the 
group $G(S)$ on the set $X(S)$. 
\end{definition}

If $G_{/k}$ is an algebraic group and $X_{/k}$ is a variety, we call $X$ a 
\emph{$G$-variety}, or \emph{variety with $G$-action}. (Recall that for us, 
a \emph{variety} over $k$ is a separated scheme of finite type over $k$.) The 
representability of quotients $G\backslash X$ is a subtle one, but fortunately 
we only need to take quotients $G/H$, where $H\subset G$ is an algebraic 
subgroup. 

\begin{theorem}
Let $G_{/k}$ be an algebraic group, $H\subset G$ an algebraic subgroup. Then 
the fppf quotient $G/H$ is a variety over $k$. If $G$ is smooth, so is $G/H$, 
and if $H$ is normal, then $G/H$ has a unique structure of a group scheme for 
which $G\to G/H$ is a homomorphism. 
\end{theorem}
\begin{proof}
This is \cite[VI\textsubscript{A} 3.2]{sga3-i}. 
\end{proof}

A major theorem whose proof uses quotients is the Borel fixed point theorem. 
To state it, we need some terminology. Following \cite[II 5.4.1]{ega2}, we say 
a morphism $f:X\to Y$ of schemes is \emph{proper} if it is separated, of finite 
type, and if for all $Y'\to Y$, the induced morphism $X_{Y'}\to Y$ is closed. 
We call a variety $X_{/k}$ proper if the structure map $X\to \spectrum(k)$ is 
proper. There is a valuative criterion for properness \cite[II 7.3.8]{ega2}. 
One considers pairs $(A,K)$, where $A$ is a valuation ring with morphism 
$\spectrum(A)\to Y$, and $K$ is the field of fractions of $Y$. A morphism 
$X\to Y$ is proper if and only if for all such pairs, the map 
$X_{/Y}(A)\to X_{/Y}(K)$ is a bijection. If $f:X\to Y$ is a morphism of 
varieties over $\dC$, then by \cite[XII 3.2]{sga1}, $f$ is proper if and 
$f:X(\dC)\to Y(\dC)$ is proper in the topological sense. (Following 
\cite[I \S10.2 th.1]{bourbaki-topology-1-4}, a map $f:X\to Y$ between 
Hausdorff spaces is proper if it is closed, and the preimage of a compact 
set is compact.)

\begin{theorem}\label{thm:borel-fixed}
Let $k$ be an algebraically closed field, $G_{/k}$ a connected solvable group, 
and $X_{/k}$ a proper non-empty $G$-variety. Then $X^G\ne\varnothing$. 
\end{theorem}
\begin{proof}
We induct on the dimension of $G$. If $G$ is one-dimensional, then 
$G$ is one of $\{\Ga,\Gm\}$, so in particular neither $G$ nor any of its 
nontrivial quotients are proper. For $x\in X(k)$, let 
$G_x=\stabilizer_G(x)$. Either $G_x=G$, in which case $x\in X^G(k)$, or 
$G_x$ is finite, in which case we have an embedding $G/G_x\monic X$ given 
by $g\mapsto g x$. By passing to the closure of the image of this map, 
we may assume the image of $G/G_x$ in $X$ is dense. Thus 
$X\smallsetminus G/G_x$ is a $G$-stable finite nonempty (because $G$ 
isn't proper) variety on which $G$ acts, hence 
$X^G\supset X\smallsetminus G/G_x$. 

In the general case, choose a one-dimensional normal subgroup 
$H\subset G$. By induction, $X^H\ne\varnothing$, and by the first part of 
the proof, $X^G=(X^H)^{G/H}\ne\varnothing$. 

For a more careful proof, see \cite[18.1]{milne-iAG}. 
\end{proof}

Checking the valuative (or topological) criteria for properness can be pretty 
difficult. There is a special class of morphisms, called \emph{projective 
morphisms}, that are automatically proper. Let $S$ be a base scheme, and 
$\sE$ a quasi-coherent sheaf on $S$. Define a functor $\dP(\sE)$ on 
$\schemes S$ by 
\[
  \dP(\sE)(X) = \{(\sL,\varphi):\text{$\sL$ is invertible and }\varphi:\sE_X\epic \sL\} /\sim .
\]
That is, $\dP(\sE)(X)$ is the set of isomorphism classes of pairs 
$(\sL,\varphi)$, where $\sL$ is a line bundle on $X$ and 
$\varphi:\sE_X\epic \sL$ is a surjection. By \cite[II 4.2.3]{ega2}, this 
functor is representable. Following \cite[II 5.5.2]{ega2}, a morphism 
$f:X\to Y$ is called \emph{projective} if it factors as 
$X\monic \dP(\sE)\epic Y$, where $\sE$ is a coherent $\sO_Y$-module, 
$X\monic \dP(\sE)$ is a closed immersion and $\dP(\sE)\epic Y$ is the 
canonical morphism. Equivalently, $X$ is isomorphic to $\proj(\sA)$ for 
some sheaf $\sA=\sA_\bullet$ of graded $\sO_Y$-modules, which is generated 
by $\sA_1$ and for which $\sA_1$ is of finite type. By 
\cite[II 5.5.3]{ega2}, projective morphisms are proper. In particular, 
if we work over a base field $k$, all closed subschemes of 
$\dP(V)$ for a finite-dimensional $k$-vector space $V$ are proper. 





\subsection{Borel subgroups}

For this section, $k$ is an algebraically closed field of characteristic 
zero. 

\begin{definition}
Let $G_{/k}$ be a linear algebraic group. A \emph{Borel subgroup} of $G$ is 
a connected solvable subgroup $B\subset G$ that is maximal with respect to 
those properties. 
\end{definition}

\begin{theorem}\label{thm:borel-conjugate}
Let $G_{/k}$ be a linear algebraic group. Then all Borel subgroups of $G$ 
are conjugate. 
\end{theorem}
\begin{proof}
Let $B_1,B_2$ be two Borel subgroups. By \cite[18.11.a]{milne-iAG}, the 
variety $G/B_1$ is proper. By \autoref{thm:borel-fixed}, the left-action 
of $B_2$ on $G/B_1$ admits a fixed point $g B_1$. One has 
$B_2\subset\adjoint(g)(B_1)$; by maximality $B_2=\adjoint(g)(B_1)$. 
\end{proof}

In general, one calls the quotient $G/B$ of $G$ by a Borel subgroup $B$ the 
\emph{flag variety} of $G$. 

\begin{example}
If $G=\GL(n)$, then by \autoref{thm:lie-kolchin}, the subgroup $B$ of 
upper-triangular matrices is a Borel subgroup. We will see that 
$G/B$ is proper. Indeed, we will work in much greater generality. Let $S$ be a 
base scheme and $\sE$ a locally free $\sO_S$-module. If $X_{/S}$ is a scheme, 
write $\sE_X$ for the pullback of $\sE$ to $X$. Write $\GL(\sE)$ for the 
functor $X\mapsto \automorphisms_{\sO_X}(\sE_X)$. This is an open subscheme of 
$\dV(\sE^\vee\otimes \sE)$, so it is representable. 

Let $n=\rank(\sE)$; for $r<n$, let $\grassmannian(\sE,r)$ be the functor given 
by 
\[
  \grassmannian(\sE,r)(X) = \{\sE_X\epic \sF:\text{$\sF$ is a locally free $\sO_X$-module of rank $r$}\} /\simeq .
\]
By \cite[ex.2]{nitsure-2005}, this is representable. note that 
$\grassmannian(\sE,1) = \dP(\sE)$, so $\grassmannian(\sE,1)$ is projective. 
More generally, for each $r$ the operation ``take $r$-th wedge power'' gives 
a closed embedding $\grassmannian(\sE,r) \monic \dP(\bigwedge^r\sE)$, so all 
the $\grassmannian(\sE,r)$ are projective. Put 
$\grassmannian(\sE)=\prod_{r<n} \grassmannian(\sE,r)$; this is 
projective via the Segre embedding \cite[II 4.3.3]{ega2}
\[
  \dP(\sE)\times \cdots \times \dP(\textstyle\bigwedge^{n-1}\sE)\monic \dP(\sE\otimes \cdots \otimes \textstyle\bigwedge^{n-1} \sE) .
\]

Suppose $\sE$ admits a descending filtration $\filtration^\bullet\sE$ such that 
each quotient $\sE_r=\sE/\filtration^r$ has rank $r$. Define a closed subgroup 
of $\GL(\sE)$ by 
\[
  B(X) = \{g\in \GL(\sE)(X):g\text{ preserves }\filtration^\bullet\} .
\]
This is a Borel subgroup of $\GL(\sE)$. We will realize the relative 
flag variety $G/B$ as a closed subscheme of $\grassmannian(\sE)$. 
If $X_{/S}$ is a scheme, a \emph{flag (relative to $\sE$)} on $X$ is a diagram 
$\sE_X=\sV_n\epic \cdots \epic \sV_0=0$ of locally free quotients of $\sE_X$, 
where each $\sV_r$ has rank $r$. Write $\sV_\bullet$ for such a flag. Define a 
functor $\flag(\sE)$ by 
\[
  \flag(\sE)(X) = \{\text{flags relative to $\sE$ on $X$}\} / \simeq .
\]
The rule $\sV_\bullet\mapsto (\sV_r)_r$ gives a closed embedding 
$\flag(\sE)\monic \grassmannian(\sE)$. See 
\cite[I \S 2 6.3]{demazure-gabriel-1980} for a proof when $S=\spectrum(\dZ)$ 
and $\sE=\dZ^n$. 

Our filtration $\filtration^\bullet\sE$ induces a canonical element of 
$\flag(\sE)(S)$, namely $\{\sE\epic\sE/\filtration^r\}_r$. The action of 
$\GL(\sE)$ on $\sE$ induces an action of $\GL(\sE)$ on $\flag(\sE)$. An 
element $g\in \GL(\sE)(X)$ acts on a flag $\sE\epic \sV_\bullet$ by 
$\sE\xrightarrow{g^{-1}} \sE\epic \sV_\bullet$. Moreover, $B\subset \GL(\sE)$ 
acts trivially. Assume $S$ is locally noetherian. Then by 
\cite[V 10.1.1]{sga3-i}, the quotient $\GL(\sE)/B$ exists and comes with 
an embedding $\GL(\sE)/B\monic \flag(\sE)$. We claim that this map is 
surjective, hence an isomorphism. Indeed, Zariski-locally, it comes down to 
checking that $\GL(A^n)/B(A)\to \flag(A^n)$ is surjective, which is a basic 
fact from linear algebra. Thus $\GL(\sE)/B\iso \flag(\sE)$. 
\end{example}

We will be mainly interested in when $S=\spectrum(k)$ and 
$V=k^n$. In this case, we write $\grassmannian(n,r)=\grassmannian(k^n,r)$ 
and $\flag(n)=\flag(k^n)$. Thus $\GL(n)/B\iso \flag(n)\monic \grassmannian(n)$. 
More generally, we have the following fact. 

\begin{theorem}
Let $G_{/k}$ be a linear algebraic group, $B\subset G$ a Borel subgroup. Then 
$G/B$ is a projective variety. 
\end{theorem}






\subsection{Maximal tori}

For this section, $k$ is an algebraically closed field of characteristic 
zero. Let $G_{/k}$ be a linear algebraic group. A \emph{maximal torus} in 
$G$ is a subgroup scheme $T\subset G$ that is a torus, and is maximal with 
respect to this property. In our setting, any torus in $G$ is contained in a 
maximal torus. 


\begin{example}
Let $G=\GL(n)$. Then we can choose $T$ to be the subgroup consisting of 
diagonal matrices. This is a maximal torus because all tori are diagonalizable, 
so any torus in $\GL(n)$ is conjugate to a subgroup of $T$. A Borel subgroup 
consists of upper-triangular matrices. 
\end{example}

\begin{theorem}
Let $G_{/k}$ be a linear algebraic group. Then all maximal tori are conjugate. 
\end{theorem}
\begin{proof}
If $G$ is solvable, this is \cite[17.40]{milne-iAG}. Essentially, one uses 
the fact that $G=T\ltimes \urad G$. 

If $G$ is a general group, then different maximal tori $T_1,T_2$ will live in 
Borel subgroups $B_1,B_2$. By \autoref{thm:borel-conjugate}, the subgroups 
$B_1$ and $B_2$ are conjugate, so we may as well assume $T_1$ and $T_2$ are 
maximal tori in the same Borel subgroup $B$. But then we can use the fact that 
the theorem holds for solvable groups. 
\end{proof}

\begin{example}
Let $\rho:G\to \GL(V)$ be a representation, where $G$ is a connected 
solvable group. Then $G$ acts on $\dP(V)$ via $\GL(V)$, so 
\autoref{thm:borel-fixed} tells us that $G$ fixes a vector, i.e.~that 
$V$ contains a $G$-invariant line. 
\end{example}





