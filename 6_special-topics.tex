% !TEX root = 6490.tex

\section{Special topics}





\subsection[Borel-Weil theorem]{Borel-Weil theorem\footnote{Balazs Elek}}

Let $k$ be a field of characteristic zero, $G_{/k}$ an algebraic group. We 
write $\modules(G)$ for the category whose objects are (not necessarily 
finite-dimensional) vector spaces $V$ together with $\rho:G\to \GL(V)$. Here, 
$\GL(V)$ is the fppf sheaf (not representable unless $V$ is finite-dimensional) 
$S\mapsto \GL(V_{\sO(S)})$. The action action of $G$ on itself by 
multiplication induces an action of $G$ on the ($k$-vector spaces) $\sO(G)$, 
which is not finite-dimensional unless $G$ is finite. By 
\cite[I 3.9]{jantzen-2003}, the category $\modules(G)$ has enough injectives; 
this enables us to define derived functors in the usual way. 

Let $H\subset G$ be an algebraic subgroup. There is an obvious functor 
$\restrict_H^G:\modules(G)\to \modules(H)$ which sends $(V,\rho:G\to \GL(V))$ 
to $(V,\rho|_H)$. It has a right adjoint, the \emph{induction functor}, 
determined by 
\[
  \hom_G(V,\induce_H^G U) = \hom_H(\restrict_H^G V,U) .
\]
We are especially interested in this when $U$ is finite-dimensional, in which 
case we have 
\[
  \induce_H^G U = \{f:G\to \dV(U):f(g h) = h^{-1} f(g)\text{ for all }g\in G\} .
\]
Since the induction functor is a right adjoint, it is left-exact, so it 
makes sense to talk about its derived functors $\eR^\bullet\induce_H^G$. It 
turns out that these can be computed as the sheaf cohomology of certain 
locally free sheaves on the quotient $G/H$. Let $\pi:G\epic G/H$ be the 
quotient map; for a representation $V$ of $H$, define an $\sO_{G/H}$-module 
$\sL(V)$ by 
\[
  \sL(V)(U) = (V\otimes \sO(\pi^{-1} U))^G .
\]
By \cite[I 5.9]{jantzen-2003}, the functor $\sL(-)$ is exact, and sends 
finite-dimensional representations of $H$ to coherent $\sO_{G/H}$-modules. 
Moreover, by \cite[I 5.12]{jantzen-2003} there is a canonical isomorphism 
\begin{equation*}\tag{$\ast$}\label{eq:induce-cohomology}
  \eR^\bullet\induce_H^G V = \h^\bullet(G/H,\sL(V)) .
\]

In general, both the vector spaces in \eqref{eq:induce-cohomology} will not be 
finite-dimensional. However, if $G/H$ is proper, then finiteness theorems 
for proper pushforward \cite[3.2.1]{ega3-i} tell us that 
$\h^\bullet(G/H,\sF)$ is finite-dimensional whenever $\sF$ is coherent. So 
we can use $\eR^i \induce_H^G$ to produce finite-dimensional representations of 
$G$ from finite-dimensional representations of $H$. 

\emph{Note}: for the remainder of this section, ``representation'' means 
\emph{finite-dimensional} representation, while ``module'' means possibly 
infinite-dimensional representation. 

Let $G_{/k}$ be a split reductive group, $B\subset G$ a Borel subgroup and 
$T\subset B$ a maximal torus. Let $N=\urad B$; one has $B\simeq T\rtimes N$. 
In particular, we can extend $\chi\in \characters^\ast(T)$ to a one-dimensional 
representation of $B$ by putting $\chi(t n) = \chi(t)$ for $t\in T$, $n\in N$. 

\begin{theorem}
Every irreducible representation of $G$ is a quotient of $\induce_B^G\chi$ for 
a unique $\chi\in\characters^\ast(T)$. 
\end{theorem}
\begin{proof}
Let $V$ be an irreducible representation of $G$. The group $B$ acts on the 
projective variety $\dP(V)$; by \ref{thm:borel-fixed} there is a fixed point 
$v$, i.e.~$\restrict_B^G V$ contains a one-dimensional subrepresentation. 
This corresponds to a $B$-equivariant map $\chi\monic \restrict_B^G V$ for some 
$\chi\in \characters^\ast(T)$. By the definition of induction functors, we get 
a nonzero map $\induce_B^G \chi \to V$. Since $V$ is simple, it must be 
surjective. Uniqueness of $\chi$ is a bit trickier. 
\end{proof}

One calls $\chi$ the \emph{highest weight} of $V$. A natural question is: for 
which $\chi$ is $\induce_B^G\chi$ irreducible? By \cite[II 2.3]{jantzen-2003}, 
$\induce_B^G\chi$ (if nonzero) contains a unique simple subrepresentation, 
which we denote by $L(\chi)$. 

Recall that our choice of Borel $B\subset T$ induces a base 
$S\subset \roots(G,T)\subset \characters^\ast(T)$. We put an ordering on 
$\characters^\ast(T)$ by saying that $\lambda \leqslant \mu$ if 
$\mu-\lambda\in \dN\cdot S$. It is known that there exists 
$w_0\in W=\weyl(G,T)$ such that $w_0(R^+)=R^-$. Finally, the set of 
\emph{dominant weights} is: 
\[
  \characters^\ast(T)_+ = \{\chi\in \characters^\ast(T):\langle \chi,\check\alpha\rangle \geqslant 0\text{ for all }\alpha\in R^+\} .
\]
We can now classify irreducible representations of $G$. 

\begin{theorem}
Any irreducible representation of $G$ is of the form $L(\chi)$ for a unique 
$\chi\in \characters^\ast(T)_+$. 
\end{theorem}
\begin{proof}
This is \cite[II 2.7]{jantzen-2003}. 
\end{proof}

The Borel-Weil theorem completely describes $\eR^\bullet\induce_B^G(\chi)$ for 
dominant $\chi$. First we need some definitions. If $w\in W$, the \emph{length} 
of $w$, denoted $l(w)$, is the minimal $n$ such that $w$ can be written as a 
product $s_1\dotsm s_n$ of simple reflections. Let $S$ be a set of simple roots, 
$\rho=\frac 1 2 \sum_{\alpha\in R^+} \alpha$. The ``dot action'' of $W$ on 
character is: 
\[
  w\bullet \chi = w(\chi+\rho)-\rho .
\]
Define 
\[
  C = \{\chi\in \characters^\ast(T):\langle \chi+\rho,\check\alpha\rangle\text{ for all }\alpha\in R^+\} .
\]
It turns out that all $\chi\in \characters^\ast(T)$ are of the form 
$w\bullet \chi_1$ for some $\chi_1\in C$. Thus the following theorem describes 
$\eR^\bullet\induce_B^G\chi$ for all $\chi$. 

\begin{theorem}[Borel-Weil]
Let $\chi\in C$. If $c\notin \characters^\ast(T)_+$, then 
$\eR^\bullet\induce_B^G(w\bullet\chi)=0$ for all $w\in W$. If 
$\chi\in \characters^\ast(T)_+$, then for all $w\in W$, 
\[
  \eR^i\induce_B^G(w\bullet\chi) = \begin{cases} L(\chi)& i=l(w) \\ 0 & \text{otherwise} \end{cases}
\]
\end{theorem}
\begin{proof}
This is \cite[II 5.5]{jantzen-2003}. 
\end{proof}

Putting $w=1$, we see that $\induce_B^G(\chi) = L(\chi)$. 





\subsection{Tannakian categories}

Throughout, $k$ is an arbitrary field of characteristic zero. We will work over 
$k$, so all maps are tacitly assumed to be $k$-linear and all tensor product 
will be over $k$. Consider the following categories. 

For $G_{/k}$ an algebraic group, the category $\rep(G)$ has as objects pairs 
$(V,\rho)$, where $V$ is a finite-dimensional $k$-vector space and 
$\rho:G\to \GL(V)$ is a homomorphism of $k$-groups. A morphism 
$(V_1,\rho_1)\to (V_2,\rho_2)$ in $\rep(G)$ is a $k$-linear map 
$f:V_1\to V_2$ such that for all $k$-algebras $A$ and $g\in G(A)$, one has 
$f \rho_1(g) = \rho_2(g)  f$, i.e.~the following diagram commutes:
\[
\begin{tikzcd}
  V_1\otimes A \ar[r, "f"] \ar[d, "\rho_1(g)"] 
    & V_2\otimes A \ar[d, "\rho_2(g)"] \\
  V_1\otimes A \ar[r, "f"]
    & V_2\otimes A .
\end{tikzcd}
\]

\begin{example}[Representations of a Hopf algebra]
Let $H$ be a co-commutative Hopf algebra. The category $\rep(H)$ has as objects 
$H$-modules that are finite-dimensional over $k$, and morphisms are 
$k$-linear maps. The algebra $H$ acts on a tensor product $U\otimes V$ via 
its comultiplication $\Delta:H\to H\otimes H$. 
\end{example}

\begin{example}[Representations of a Lie algebra]
Let $\fg$ be a Lie algebra over $k$. The category $\rep(\fg)$ has as objects 
$\fg$-representations that are finite-dimensional as a $k$-vector space. There 
is a canonical isomorphism $\rep(\fg)=\rep(\cU \fg)$, where $\cU \fg$ is the 
universal enveloping algebra of $\fg$. 
\end{example}

\begin{example}[Continuous representations of a compact Lie group]
Let $K$ be a compact Lie group. The category $\rep_\dC(K)$ has as objects 
pairs $(V,\rho)$, where $V$ is a finite-dimensional complex vector space and 
$\rho:K\to \GL(V)$ is a continuous (hence smooth, by Cartan's theorem) 
homomorphism. Morphisms $(V_1,\rho_1)\to (V_2,\rho_2)$ are $K$-equivariant 
$\dC$-linear maps $V_1\to V_2$. 
\end{example}

\begin{example}[Graded vector spaces]
Consider the category whose objects are finite-dimensional $k$-vector spaces 
$V$ together with a direct sum decomposition $V=\bigoplus_{n\in \dZ} V_n$. 
Morphisms $U\to V$ are $k$-linear maps $f:U\to V$ such that 
$f(U_n)\subset V_n$. 
\end{example}

\begin{example}[Hodge structures]
Let $V$ be a finite-dimensional $\dR$-vector space. A \emph{Hodge structure} 
on $V$ is a direct sum decomposition $V_\dC=\bigoplus V_{p,q}$ such that 
$\overline{V_{p,q}}=V_{q,p}$. If $U,V$ are vector spaces with Hodge structures, 
a morphism $U\to V$ is a $\dR$-linear map $f:U\to V$ such that 
$f(U_{p,q})\subset V_{p,q}$. Write $\mathsf{Hdg}$ for the category of vector spaces 
with Hodge structure. 
\end{example}

Let $\mathsf{Vect}(k)$ be the category of finite-dimensional $k$-vector spaces. For 
$\cC$ any of the categories above, there is a faithful functor 
$\omega:\cC\to \mathsf{Vect}(k)$. In our examples, it is just the forgetful functor. 
The main theorem will be that for $\pi=\automorphisms(\omega)$, the functor $\omega$ 
induces an equivalence of categories $\cC\iso \rep(\pi)$. We proceed to make 
sense of the undefined terms in this theorem. 


Our definitions follow \cite{deligne-milne-1982}. As before, $k$ is an 
arbitrary field of characteristic zero. 

\begin{definition}
A \emph{$k$-linear category} is an abelian category $\cC$ such that each 
$V_1,V_2$, the group $\hom(V_1,V_2)$ has the structure of a $k$-vector space 
in such a way that the composition map 
$\hom(V_2,V_3)\otimes \hom(V_1,V_2)\to \hom(V_1,V_3)$ is $k$-linear. For us, 
a \emph{rigid $k$-linear tensor category} is a $k$-linear category $\cC$ 
together with the following data:
\begin{enumerate}
\item An exact faithful functor $\omega:\cC\to \mathsf{Vect}(k)$. 
\item A bi-additive functor $\otimes:\cC\times \cC\to \cC$. 
\item Natural isomorphisms 
$\omega(V_1\otimes V_2)\iso \omega(V_1)\otimes \omega(V_2)$. 
\item Isomorphisms $V_1\otimes V_2\iso V_2\otimes V_1$ for all $V_i\in \cC$. 
\item Isomorphisms $(V_1\otimes V_2)\otimes V_3\iso V_1\otimes (V_2\otimes V_3)$
\end{enumerate}
These data are required to satisfy the following conditions:
\begin{enumerate}
\item There exists an object $1\in \cC$ such that $\omega(1)$ is 
one-dimensional and such that the natural map $k\to \hom(1,1)$ is an 
isomorphism. 
\item If $\omega(V)$ is one-dimensional, there exists $V^{-1}\in \cC$ such 
that $V\otimes V^{-1}\simeq 1$. 
\item Under $\omega$, the isomorphisms 3 and 4 are the obvious ones. 
\end{enumerate}
\end{definition}

By \cite[Pr.~1.20]{deligne-milne-1982}, this is equivalent to the standard 
(more abstract) definition. Note that all our examples 
are rigid $k$-linear tensor categories. One calls the 
functor $\omega$ a \emph{fiber functor}. 

Let $(\cC,\otimes)$ be a rigid $k$-linear tensor category. In this setting, 
define a functor $\automorphisms(\omega)$ from $k$-algebras to groups by setting: 
\begin{align*}
  \automorphisms^\otimes(\omega)(A) 
    &= \automorphisms^\otimes\left(\omega:\cC\otimes A\to \rep(A)\right) \\
    &= \left\{(g_V)\in \prod_{V\in \cC} \GL(\omega(V)\otimes A):\begin{array}{c}g_{V_1\otimes V_2} = g_{V_1}\otimes g_{V_2}\text{, and } \\ f g_{V_1} = g_{V_1} f\text{ for all }f,V_1,V_2\end{array} \right\} .
\end{align*}
In other words, an element of $\automorphisms(\omega)(A)$ consists of a collection 
$(g_V)$ of $A$-linear automorphisms 
$g_V:\omega(V)\otimes A\iso \omega(V)\otimes A$, where $V$ ranges over objects 
in $\cC$. This collection must satisfy: 
\begin{enumerate}
  \item $g_1 = 1_{\omega(1)}$
  \item $g_{V_1\otimes V_2} = g_{V_1}\otimes g_{V_2}$ for all $V_1,V_2\in \cC$, and 
  \item whenever $f:V_1\to V_2$ is a morphism in $\cC$, the following diagram 
    commutes: 
    \[
    \begin{tikzcd}
      \omega(V_1)_A \ar[r, "f"] \ar[d, "g_{V_1}"] 
        & \omega(V_2)_A \ar[d, "g_{V_2}"] \\
      \omega(V_1)_A \ar[r, "f"] 
        & \omega(V_2)_A .
    \end{tikzcd}
    \]
\end{enumerate}

Typically one only considers affine group schemes $G_{/k}$ that are 
\emph{algebraic}, i.e.~whose coordinate ring $\sO(G)$ is a finitely generated 
$k$-algebra, or equivalently that admit a finite-dimensional faithful 
representation. Let $G_{/k}$ be an arbitrary affine group scheme, $V$ an 
arbitrary representation of $G$ over $k$. By 
\cite[Cor.~2.4]{deligne-milne-1982}, one has $V=\varinjlim V_i$, where $V_i$ 
ranges over the finite-dimensional subrepresentations of $V$. Applying this to 
the regular representation $G\to \GL(\sO(G))$, we see that 
$\sO(G)=\varinjlim\sO(G_i)$, where $G_i$ ranges over the algebraic quotients of 
$G$. That is, an arbitrary affine group scheme $G_{/k}$ can be written as a 
filtered projective limit $G=\varprojlim G_i$, where each $G_i$ is an affine 
algebraic group over $k$. So we will speak of pro-algebraic groups instead of 
arbitrary affine group schemes. 

If $V$ is a finite-dimensional $k$-vector space and $G=\varprojlim G_i$ is a 
pro-algebraic $k$-group, representations $G\to \GL(V)$ factor through some 
algebraic quotient $G_i$. That is, 
$\hom(G,\GL(V))=\varinjlim \hom(G_i,\GL(V))$. As a basic example of this, 
let $\Gamma$ be a profinite group, i.e.~a projective limit of finite groups. If 
we think of $\Gamma$ as a pro-algebraic group, then algebraic representations 
$\Gamma\to \GL(V)$ are exactly those representations that are continuous when 
$V$ is given the discrete topology. 

First, suppose $\cC=\rep(G)$ for a pro-algebraic group $G$, and that 
$\omega:\rep(G)\to \mathsf{Vect}(k)$ is the forgetful functor. Then the Tannakian 
fundamental group $\automorphisms^\otimes(\omega)$ carries no new information 
\cite[Pr.~2.8]{deligne-milne-1982}: 

\begin{theorem}\label{thm:reconst}
Let $G_{/k}$ be a pro-algebraic group, $\omega:\rep(G)\to \mathsf{Vect}(k)$ the 
forgetful functor. Then $G\iso \automorphisms^\otimes(G)$. 
\end{theorem}

The main theorem is the following, taken essentially verbatum from 
\cite[Th.~2.11]{deligne-milne-1982}. 

\begin{theorem}\label{thm:main}
Let $(\cC,\otimes,\omega)$ be a rigid $k$-linear tensor category. Then 
$\pi=\automorphisms^\otimes(\omega)$ is represented by a pro-algebraic group, and 
$\omega:\cC\to \rep(\pi)$ is an equivalence of categories. 
\end{theorem}

Often, the group $\pi_1(\cC)$ is ``too large'' to handle directly. For 
example, if $\cC$ contains infinitely many simple objects, probably 
$\pi_1(\cC)$ will be infinite-dimensional. For $V\in \cC$, let 
$\cC(V)$ be the Tannakian subcategory of $\cC$ generated by $V$. One 
puts $\pi_1(\cC/V)=\pi_1(\cC(V))$. It turns out that 
$\pi_1(\cC/V)\subset \GL(\omega V)$, so $\pi_1(\cC/V)$ is finite-dimensional. 
One has $\pi_1(\cC)=\varprojlim \pi_1(\cC/V)$. 

\begin{example}[Pro-algebraic groups]
If $G_{/k}$ is a pro-algebraic group, then \autoref{thm:reconst} tells us that if 
$\omega:\rep(G)\to \mathsf{Vect}(k)$ is the forgetful functor, then 
$G=\automorphisms^\otimes(G)$. That is, $G=\pi_1(\rep G)$. 
\end{example}

\begin{example}[Hopf algebras]
Suppose $H$ is a co-commutative Hopf algebra over $k$. Then 
$\pi_1(\rep H)=\spectrum(H^\circ)$, where $H^\circ$ is the \emph{reduced dual} 
defined in \cite{cartier-2007}. Namely, for any $k$-algebra $A$, $A^\circ$ is 
the set of $k$-linear maps $\lambda:A\to k$ such that $\lambda(\fa)=0$ for some 
two-sided ideal $\fa\subset A$ of finite codimension. The key fact here is that 
$(A\otimes B)^\circ=A^\circ\otimes B^\circ$, so that we can use multiplication 
$m:H\otimes H\to H$ to define comultiplication 
$m^\ast:H^\circ\to (H\otimes H)^\circ=H^\circ\otimes H^\circ$. 
From \cite[II \S 6 1.1]{demazure-gabriel-1980}, if $G$ is a linear algebraic 
group over an algebraically closed field $k$ of characteristic zero, we get an 
isomorphism $\sO(G)^\circ= k[G(k)]\otimes \cU(\fg)$. Here $k[G(k)]$ is the 
usual group algebra of the abstract group $G(k)$, and $\cU(\fg)$ is the 
universal enveloping algebra of $\fg=\lie(G)$, both with their standard Hopf 
structures. 
\end{example}

[Note: one often calls $\sO(G)^\circ$ the ``space of distributions on $G$.'' 
If instead $G$ is a real Lie group, then one often writes $\sH(G)$ for the 
space of distributions on $G$. Let $K\subset G$ be a maximal compact subgroup, 
$M(K)$ the space of finite measures on $K$. Then convolution 
$D\otimes \mu\mapsto D\ast\mu$ induces an isomorphism 
$\cU(\fg)\otimes M(K)\iso \sH(G)$. In the algebraic setting, 
$k[G(k)]$ is the appropriate replacement for $M(K)$.]

\begin{example}[Lie algebras]
Let $\fg$ be a semisimple Lie algebra over $k$. Then by \cite{milne-2007}, 
$G=\pi_1(\rep\fg)$ is the unique connected, simply connected algebraic group 
with $\lie(G)=\fg$. If $\fg$ is not semisimple, e.g.~$\fg=k$, then things get a 
lot nastier. See the above example. 
\end{example}

\begin{example}[Compact Lie groups]
By definition, the \emph{complexification} of a real Lie group $K$ is a complex 
Lie group $K_\dC$ such that all morphisms $K\to \GL(V)$ factor uniquely through 
$K_\dC\to\GL(V)$. It turns out that $K_\dC$ is a complex algebraic group, and 
so $\pi_1(\rep K)=K_\dC$. 
\end{example}

\begin{example}[Graded vector spaces]
To give a grading $V=\bigoplus_{n\in \dZ} V_n$ on a vector space is equivalent 
to giving an action of the split rank-one torus $\Gm$. On each $V_n$, $\Gm$ 
acts via the character $g\mapsto g^n$. Thus 
$\pi_1(\text{graded vector spaces})=\Gm$. 
\end{example}

\begin{example}[Hodge structures]
Let $\dS=\weil_{\dC/\dR}\Gm$; this is defined by $\dS(A)=(A\otimes\dC)^\times$ 
for $\dR$-algebras $A$. One can check that the category $\mathsf{Hdg}$ of Hodge 
structures is equivalent to $\rep_\dR(\dS)$. Thus $\pi_1(\mathsf{Hdg})=\dS$. 
\end{example}





\subsection[Automorphisms of semisimple Lie algebras]{Automorphisms of semisimple Lie algebras\footnote{Sasha Patotski}}

Let $k$ be a field of characteristic zero, $\fg$ a split semisimple Lie algebra 
over $k$. We want to describe the group $\automorphisms(\fg)$. It turns out 
$\automorphisms(\fg)=\automorphisms(G^\mathrm{sc})$, where $G^\mathrm{sc}$ is 
the unique simply connected semisimple group with Lie algebra $\fg$. We can give 
$\automorphisms(\fg)$ the structure of a linear algebraic group by putting 
\[
  \automorphisms(\fg)(A) = \automorphisms(\fg_A) .
\]
Clearly $\automorphisms(\fg)\subset \GL(\fg)$. Let $\ft\subset \fg$ be a maximal 
abelian subalgebra. Let $R\subset \ft^\vee$ be the set of roots, 
$S\subset R$ a base of $\Delta$. We can define some subgroups of 
$\automorphisms(\fg)$:
\begin{align*}
  \automorphisms(\fg,\ft) 
    &= \{\theta\in \automorphisms(\fg):\theta(\ft)= \ft\} \\
  \automorphisms(\fg,\ft,S) 
    &= \{\theta\in \automorphisms(\fg,\ft):\transpose\theta(S)= S\} .
\end{align*}

\begin{lemma}
If $\theta\in \automorphisms(\fg,\ft)$, then $\transpose\theta(R)=R$. 
\end{lemma}
\begin{proof}
The definition of $R$ is invariant under automorphisms of $\ft$. 
\end{proof}

Let $\automorphisms(R)$ be the set of automorphisms of the Dynkin diagram 
associated to $R$. 

\begin{lemma}
The natural map
$\varepsilon:\automorphisms(\fg,\ft,S)\to \automorphisms(R)$ is surjective. 
\end{lemma}
\begin{proof}
Choose, for each $\alpha\in R$, a non-zero element $x_\alpha\in \fg_\alpha$. 
The existence theorem \cite[VIII \S 4.4 th.2]{bourbaki-lie-alg-7-9} tells us 
that to each automorphism $\varphi$ of the Dynkin diagram of $R$, there exists 
a unique $\tilde\varphi\in \automorphisms(\fg,\ft,S,\{x_\alpha\})$ inducing 
$\varphi$. Thus $\varepsilon:\automorphisms(\ft,\ft,S)\to \automorphisms(R)$ 
splits. 
\end{proof}

We want to describe the kernel of $\varepsilon$. Since $\fg$ is semisimple, 
the adjoint map $\fg\to \derivations(\fg)=\lie(\automorphisms\fg)$ is an 
embedding, and thus $\ft\iso \adjoint(\ft)$. Let $T\subset \inner(\fg)$ be the 
subgroup with Lie algebra $\adjoint(\ft)$. 

\begin{lemma}
$\ker(\varepsilon) = T$. 
\end{lemma}
\begin{proof}
Clearly $T\subset \ker(\varepsilon)$. Let $\theta\in \ker(\varepsilon)$. For 
any $\alpha\in R$, we have $\theta x_\alpha\in \fg_\beta$ for some $\beta$. 
For $t\in\ft$, we compute: 
\begin{align*}
  [t,\theta x_\alpha] 
    &= [\theta t,\theta x_\alpha] \\
    &= \theta[t,x_\alpha] \\
    &= \alpha(t)\theta x_\alpha ,
\end{align*}
so $\alpha=\beta$. Moreover, $\theta$ acts on $\fg$ exactly like an element of 
$\ft$, so $\theta\in T$. 
\end{proof}

\begin{lemma}
$\automorphisms(\fg)=\automorphisms(\fg,\ft,S)\cdot \inner(\fg)$. 
\end{lemma}
\begin{proof}
Use the well-known facts that all Cartan subalgebras of $\fg$ are conjugate, 
and that moreover the Weyl group acts transitively on sets of simple roots. 
\end{proof}

We have arrived at the main result. 

\begin{theorem}
$\automorphisms(\fg)=\inner(\fg)\rtimes \automorphisms(R)$. In 
particular, $\mathrm{Out}(\fg) = \automorphisms(R)$. 
\end{theorem}

This allows us to recover the table of automorphisms in 
\autoref{sec:root-classify}. 





\subsection[Exceptional isomorphisms]{Exceptional isomorphisms\footnote{Theodore Hui}}

Terence Tao's blog post at 
\url{http://terrytao.wordpress.com/2011/03/11/}
is an excellent reference for this section. From 
\autoref{thm:classify-semisimple}, we know that the set of isogeny classes of 
split semisimple algebraic groups is the same as the set of isomorphism classes 
of Dynkin diagrams. In \autoref{thm:classify-root}, we classified the Dynkin 
diagrams. In this classification, we just included, e.g.~$\typeD_n$ for 
$n\geqslant 4$. If we include all the $\typeD_n$ etc., the classification is 
no longer unique -- we have to account for the ``exceptional isomorphisms'' 
\begin{align*}
  \typeA_1 &\simeq \typeB_1\simeq \typeC_1\simeq \typeD_2\simeq \typeE_1 \\
  \typeB_2 &\simeq \typeC_2 \\
  \typeD_3 &\simeq \typeA_3 \\
  \typeD_2 &\simeq \typeA_1\times \typeA_1 .
\end{align*}

The Dynkin diagram for $\typeD_2$ and $\typeA_1\times \typeA_1$ is disconnected 
-- it is a disjoint union of two points. We'll explicitly construct the induced 
isogenies between algebraic groups of different types. 

\begin{example}[$\typeA_1\simeq\typeC_1$]
Since $\Sp(2)=\SL(2)$, there is nothing to prove.  
\end{example}

\begin{example}[$\typeA_1\simeq\typeB_1$]
Define a pairing on $\Sl_2$ by $\langle x,y\rangle = \trace(x y)$ (the 
\emph{Killing form}). One easily verifies that the adjoint action of 
$\SL(2)$ on $\Sl_2$ preserves this form. Moreover, a general theorem of linear 
algebra tells us that any non-degenerate bilinear symmetric pairing on a 
three-dimensional vector space is isomorphic to the orthogonal pairing. It 
follows that $\automorphisms(\Sl_2,\langle\cdot,\cdot\rangle)\simeq \Or(3)$, 
at least over an algebraically closed field. Since $\SL(2)$ is connected, 
$\adjoint(\SL(2))\supset \SO(3)$. A dimension count tells us that 
$\SL(2)/\dmu_2\simeq \SO(3)$. Again, this only works over an algebraically 
closed field. 
\end{example}


\begin{example}[$\typeD_2\simeq \typeA_1\times \typeA_1$]
We need to show that $\SO(4)$ and $\SL(2)\times \SL(2)$ are isogenous. 
Let $\mathrm{std}:\SL(2)\monic \GL(2)$ be the standard representation, and 
consider the representation $\mathrm{std}\boxtimes\mathrm{std}$ of 
$\SL(2)\times \SL(2)$. There is an obvious bilinear form form: 
\[
  \langle u_1\otimes u_2,v_1\otimes v_2\rangle = \omega(u_1,v_1)\omega(u_2,v_2) ,
\]
where $\omega$ is the determinant pairing 
$k^2\times k^2\to \bigwedge^2 k^2\simeq k$ given by $(x,y)\mapsto x\wedge y$. 
The group $\SL(2)\times \SL(2)$ acts on $\mathrm{std}\boxtimes\mathrm{std}$ by 
$(g,h)(v\otimes w) = (g v)\otimes (g w)$. Thus we have a representation 
$\SL(2)\times \SL(2)\to \GL(4)$. The image preserves 
$\langle\cdot,\cdot\rangle$, hence (by linear algebra) lies inside $\Or(4)$, 
By connectedness and a dimension count, we see that this is an isogeny 
$\SL(2)\times \SL(2)\epic \SO(4)$. 
\end{example}





\subsection[Constructing some exceptional groups]
{Constructing some exceptional groups\footnote{Gautam Gopal}}

We roughly follow \cite{springer-veldkamp-2000}. Let $k$ be a field of 
characteristic not $2$ or $3$. Recall that if $V$ is a $k$-vector space and 
$\langle\cdot,\cdot\rangle:V\times V\to k$ is a symmetric bilinear form, we can 
define a quadratic form $q:V\to k$ by $q(v)=\langle v,v\rangle$. This 
correspondence is bijective; we can go backwards via the familiar identity 
\[
  \langle u,v\rangle = \frac 1 2 (q(u+v)-q(u)-q(v)) .
\]

\begin{definition}
A \emph{composition algebra} is a pair $(C,q)$, where $C$ is a unital, 
not-necessarily associative $k$-algebra and $q$ is a multiplicative 
non-degenerate quadratic on $C$.
\end{definition}

In other words, we require $q(x y)=q(x)q(y)$ for all $x,y\in C$. Since by 
\cite[1.2.4]{springer-veldkamp-2000}, $q$ is determined by the multiplicative 
strucure of $C$, we will just refer to ``a composition algebra $C$.'' Write $e$ 
for the unit of $C$. Every composition algebra comes with a natural involution 
$x\mapsto \bar x$, defined by $\bar x=\langle x,e\rangle-x$. 

\begin{theorem}
Let $C$ be a composition algebra. Then $\dim(C)\in \{1,2,4,8\}$. 
\end{theorem}
\begin{proof}
This is \cite[1.6.2]{springer-veldkamp-2000}. 
\end{proof}

We call an $8$-dimensional composition algebra an \emph{octobian algebra}. 
If $C$ is an octonian $k$-algebra, we define an algebraic group 
$\automorphisms(C)$ by putting 
\[
  \automorphisms(C)(A) = \{g\in \GL(C\otimes A):g\text{ is a morphism of normed $A$-algebras}\} ,
\]
for all (commutative, unital) $k$-algebras $A$. There is an obvious embedding 
$\automorphisms(C)\monic \Or(C,q)$. 

\begin{theorem}
Let $C$ be an octonian $k$-algebra. Then $\automorphisms(C)$ is a connected 
algebraic group of type $\typeG_2$. 
\end{theorem}
\begin{proof}
We mean that after base-change to an algebraic closure of $k$, 
$\automorphisms(C)$ becomes isomorphic to $\typeG_2$. This is 
\cite[2.3.5]{springer-veldkamp-2000}.
\end{proof}

If $C$ is a composition algebra and 
$\gamma=(\gamma_1,\gamma_2,\gamma_3)\in (k^\times)^3$, we define a new algebra 
$H_{C,\gamma}$ to be as a set the collection of matrices 
\[
  \begin{pmatrix} 
    z_1 & c_3 & \gamma_1^{-1} \gamma_3 \bar c_2 \\ 
    \gamma_2^{-1} \gamma_1 \bar c_3 & z_2 & c_1 \\ 
    c_2 & \gamma_3^{-1} \gamma_2 \bar c_1 & z_3 \end{pmatrix} \qquad c_i\in k^\times\text{ and }z_i\in k .
\]
Give $H_{C,\gamma}$ the product $x y = \frac 1 2 (x\cdot y+y\cdot x)$ and 
quadratic form $q(x)=\frac 1 2 \trace(x^2)$. 

\begin{definition}
An \emph{Albert algebra} is a commutative, non-unital, associative $k$-algebra 
$A$ such that $A\otimes \bar k$ is isomorphic to a $\bar k$-algebra of the form 
$H_{C,\gamma}$ for some octobian algebra $C$ and $\gamma\in \bar k^\times$. 
\end{definition}

If $A$ is an Albert algebra, we can define an algebraic group 
$\automorphisms(A)$ just as above. 

\begin{theorem}
If $A$ is an Albert algebra, then $\automorphisms(A)$ is a connected simple 
algebraic group of type $\typeF_4$. 
\end{theorem}
\begin{proof}
Put $G=\automorphisms(A)$. 
Call an element $u\in A$ \emph{idempotent} if $u2=u$. It turns out that if $u$ 
is idempotent, then either $u\in \{0,e\}$, or $q(u)\in \{1/2,1\}$. Call the 
idempotents with $q(u)=1/2$ \emph{primitive}. Let $V\subset A$ be the set of 
primitive idempotents; this is naturally a variety over $k$. It turns out that 
$V$ is $16$-dimensional closed and irreducible and has a transitive $G$-action. 
For some $v\in V$, the group $G_v=\stabilizer_G(v)$ is the spin group of a 
nine-dimensional quadratic form, so $\dim(G_v)=36$. It follows 
that $\dim(G)=16+36=52$. Since $G_v$ and $V$ are irreducible, $G$ is connected. 
From the action of $G$ on $e^\bot\subset A$, we see that $G$ is semisimple 
algebraic. The only $52$-dimensional semisimple algebraic group over an 
algebraically closed field is $\typeF_4$. For a more careful proof, see 
\cite[7.2.1]{springer-veldkamp-2000}. 
\end{proof}

It turns out that any group of type $\typeF_4$ can be obtained as 
$\automorphisms(A)$ for some Albert algebra $A$. Consider the cubic form 
\[
  \det(x) = z_1 z_2 z_3-\gamma_3^{-1} \gamma_2 z_1 q(c_1) - \gamma_2^{-1} \gamma_3 z_2 q(c_2) - \gamma_2^{-1} \gamma_1 z_3 q(c_3)+\langle c_1 c_2,\bar c_3\rangle .
\]
Let $\GL(A,\det)$ be the subgroup of $\GL(A)$ consisting of those linear 
maps which preserve $\det$. 

\begin{theorem}
Let $A$ be an Albert algebra. Then $\GL(A,\det)$ is a connected simple 
algebraic group of type $\typeE_6$. 
\end{theorem}
\begin{proof}
This is \cite[7.3.2]{springer-veldkamp-2000}. 
\end{proof}

Unlike the case with groups of type $\typeF_4$, not all groups of type 
$\typeE_6$ can be obtained this way. 





\subsection[Spin groups]{Spin groups\footnote{Benjamin ?}}

The motivation for spin groups is as follows. Recall that up to isogeny, the 
simple algebraic groups are $\SL(n)$, $\Sp(2n)$, $\SO(n)$, or one of the 
exceptional groups. Recall that each isogeny class has two distinguished 
elements, the simply connected and adjoint. For $\SO(n)$, we should expect 
there to be a simply connected group $\Spin(n)=(\typeD_n)^\mathrm{sc}$, which 
is a double cover of $\SO(n)$. 

Work over a field $k$ of characteristic not $2$. Let $V$ be a $k$-vector space, 
$q$ a quadratic form on $V$. The \emph{Clifford algebra} $\clifford(V,q)$ is 
the quotient of the tensor algebra $T(V)$ by the ideal generated by 
$\{v\otimes v-q(v):v\in V\}$. There is an obvious injection 
$V\monic \clifford(V,q)$, and $k$-linear maps $f:V\to A$ into associative 
$k$-algebras lift to $\tilde f:\clifford(V,q)\to A$ if and only if 
$f(v) f(v)=q(v)$ for all $v\in V$. This univeral property clearly 
characterizes $\clifford(V,q)$. For brevity, write 
$\clifford(V)=\clifford(V,q)$. 

\begin{example}
Let $V=\dR^4$, $q$ be the indefinite form of signature $(3,1)$, 
i.e.~$q(v)=-v_1^2+v_2^2+v3^2+v_4^2$. We write $\clifford_{(1,3)}(\dR)$ for 
the Clifford algebra $\clifford(V,q)$; it has presentation 
\[
  \clifford_{(1,3)}(\dR) = \dR\langle e_1,e_2,e_3,e_4\rangle / (e_1^2=-1, e_2^2=e_3^2=e_4^2=1) .
\]
This is used in the Dirac equation, which unifies special relativity and 
quantum mechanics. 
\end{example}

There is a clear action $\Or(V,q)\to \automorphisms \clifford(V)$. In 
particular, the involution $\alpha(v)=-v$ induces an involution (also denoted 
$\alpha$) of $\clifford(V)$. This induces a grading 
$\clifford(V)=\clifford^0(V)\oplus \clifford^1(V)$, where 
\begin{align*}
  \clifford^0(V) &= \{x\in\clifford(V):\alpha(x)=x\} \\
  \clifford^1(V) &= \{x\in \clifford(V):\alpha(x)=-x\} .
\end{align*}

We define some algebraic groups via their functors of points: 
\begin{align*}
  \Pin(V,q)(A) &= \{g\in \clifford(V,q)_A:q(g)\in \dmu_2(A)\} \\
  \Spin(V,q) &= \Pin(V,q)\cap \clifford^0(V) .
\end{align*}
There is a ``twisted adjoint map'' $\tilde\adjoint:\Pin(V,q)\to \Or(V,q)$, 
given by $\tilde\adjoint(g)(v) = \alpha(g) v g^{-1}$. 

\begin{theorem}
There is a natural exact sequence 
\[
  1 \to \dmu_2 \to \Spin(V,q) \to \SO(V,q) \to 1 .
\]
\end{theorem}
\begin{proof}
This is \cite[IV.10.21]{berhuy-2010}. 
\end{proof}





\subsection[Differential Galois theory]
{Differential Galois theory\footnote{Ian Pendleton}}

A good source for differential Galois theory is \cite{vanderput-singer-2003}. 
Recall that if $A$ is a ring, a \emph{derivation} on $A$ is an additive map 
$\partial:A\to A$ satisfying the \emph{Liebniz rule}: 
$\partial(a b)=a\partial(b)+\partial(a)b$. We write $\derivations(A)$ for the 
group of derivations $A\to A$. 

\begin{definition}
A \emph{differential ring} is a pair $(R,\Delta)$, where 
$\Delta\subset \derivations(R)$ is such that 
$\partial_1\partial_2 = \partial_2\partial_1$ for all 
$\partial_1,\partial_2\in \Delta$. 
\end{definition}

If $\Delta=\{\partial\}$, we write $r'=\partial r$ for $r\in R$. The ring 
$C=\{c\in R:\partial c=0\text{ for all }\partial\in\Delta\}$ is called the 
\emph{ring of constants}. If $R$ is a field, we call $(R,\Delta)$ a 
\emph{differential field}. 

\begin{example}
Let $R=C^\infty(\dR^n)$ and 
$\Delta=\{\frac{\partial}{\partial x_i}:1\leqslant i\leqslant n\}$. Then 
$(R,\Delta)$ is a differential ring with $\dR$ as ring of constants. 
\end{example}

\begin{example}
Let $k$ be a field, $R=k(x_1,\dots,x_n)$, and 
$\Delta=\{\frac{\partial}{\partial x_i}:1\leqslant i\leqslant n\}$. Then 
$(R,\Delta)$ is a differential field with field of constants $k$. 
\end{example}

We are interested in solving \emph{matrix differential equations}, that is, 
equations of the form $y'=A y$ for $A\in \matrices_n(k)$. A solution would be a 
tuple $y=(y_1,\dots,y_n)\in k^n$ such that $(y_1',\dots,y_n') = A y$. To a 
matrix differential equation we will associate two objects: a Picard-Vessiot ring 
$R$, and a linear algebraic group $\dgalois(R/k)$. 

\begin{definition}
Let $(k,\partial)$ be a differential field, $(R,\partial)$ a differential 
$k$-algebra (so $\partial_R|_k=\partial_k$). Let $A\in \matrices_n(k)$. A 
\emph{fundamental solution matrix} to the equation $y'=A y$ is an element 
$Z\in \GL_n(R)$ such that $Z'=A Z$. 
\end{definition}

It is easy to construct a (universal) fundamental solution matrix for the 
equation $y'=A y$. Let $S=\sO(\GL_n)=k[y_{i j},\det(y_{i j})^{-1}]$, and 
define a differential $\partial:S\to S$ by $\partial(y_{i j}) = (A y)_{i j}$. 
It is easy to see that $S$ represents the functor that sends a differential 
$k$-algebra $R$ to the set of fundamental solution matrices in $R$. 

If $(R,\Delta)$ is a differential ring, a \emph{differential ideal} is an 
ideal $\fa\subset R$ such that $\partial(\fa)\subset \fa$ for all 
$\partial\in \Delta$. If $\fa$ is a differential ideal, then $R/\fa$ naturally 
has the structure of a differential ring. Call a differential ring 
\emph{simple} if it has no nontrivial differential ideals. Note that simple 
differential rings need not be fields, 
e.g.~$(k[t],\frac{\partial}{\partial t})$. 

\begin{definition}
Let $A\in \matrices_n(k)$. A \emph{Picard-Vessiot ring} for the equation 
$y'=A y$ is pair $(R,Z)$, where $R$ is a simple differential $k$-algebra which 
has a fundamental solution matrix $Z$ for $y'=A y$, such that $R$ is generated 
as a $k$-algebra by the entries of $Z$ and $\frac{1}{\det Z}$. 
\end{definition}

It is easy to prove that Picard-Vessiot rings exist. Let $S$ be the 
differential ring constructed above. For any maximal differential ideal 
$\fm\subset S$, the quotient $R=S/\fm$ is a Picard-Vessiot ring. By 
\cite[1.20]{vanderput-singer-2003}, Picard-Vessiot rings (for a given 
equation $y'=A y$) are unique. 

\begin{definition}
Let $(k,\partial)$ be a differential field, $A\in \matrices_n(k)$. The 
\emph{differential Galois group} of the equation $y'= A y$ is 
$\dgalois(R/k)=\automorphisms_{(k,\partial})(R)$ for any Picard-Vessiot ring 
$R$ (for the equation $y'=A y$).  
\end{definition}

\begin{lemma}
The group $\dgalois(R/k)$ is linear algebraic. 
\end{lemma}
\begin{proof}
See \cite[1.26]{vanderput-singer-2003}. 
\end{proof}

\begin{example}
If $k=\dC(x)$ and we consider the equation $y'=\frac{\alpha}{x} y$ for 
$\alpha=\frac n m\in \dQ$, then the Picard-Vessiot ring is 
$R=\dC(x^{n/m})$ and $\dgalois(R/k)=\dZ/m$. 
\end{example}

It is shown in \cite{tretkoff-tretkoff-1979} that over $\dC$, all linear 
algebraic groups arise as differential Galois groups. The ``modern'' approach 
to differential Galois theory uses $\mathscr{D}$-modules and Tannakian 
categories. 





\subsection[Universal enveloping algebras and the Poincar\'e-Birkhoff-Witt theorem]
{Universal enveloping algebras and the Poincar\'e-Birkhoff-Witt theorem\footnote{Daoji Huang}}

Let $k$ be a field, $\mathsf{Lie}$ be the category of Lie algebras over $k$, 
and $\mathsf{Ass}$ be the category of unital associative $k$-algebras. There 
is an easy functor $\cL:\mathsf{Ass}\to \mathsf{Lie}$, that sends a $k$-algebra 
$A$ to the Lie algebra $\cL A$ whose underlying vector space is $A$, with 
bracket 
\[
  [a,b] = a\cdot b-b\cdot a .
\]
This has a left adjoint, denoted $\cU$. That is, for each Lie algebra $\fg$, 
there is an associative algebra $\cU\fg$ with a $k$-linear map 
$i:\fg\to \cU\fg$ satisfying $i[x,y]=[i(x),i(y)]$, such that for any 
algebra $A$ and linear map $f:\fg\to A$ satisfying $f[x,y]=[f(x),f(y)]$, there 
is a unique extension $\tilde f:\cU\fg\to A$ such that $f=\tilde f\circ i$. 

We can construct the universal enveloping algebra $\cU\fg$ directly. Let 
$\cT\fg=\bigoplus_{n\geqslant 0} \fg^{\otimes n}$ be the tensor algebra of 
$\fg$. It is easy to see that $\cU\fg$ is the quotient of $\cT\fg$ by the 
relations $\{x\otimes y-y\otimes x - [x,y]:x,y\in \fg\}$. 

There is an obvious filtration on $\cT\fg$, for which 
$\cT_m\fg=\bigoplus_{m\geqslant n} \fg^{\otimes m}$. It induces a filtration on 
$\cU\fg$. The Poincar\'e-Birkhoff-Witt theorem is an explicit description of 
the graded ring $\graded\cU(\fg)$. 

Define a map $\varphi:\cT\fg\to \graded(\cU\fg)$ by the obvious surjections 
\[
  \cT^m\fg = \fg^{\otimes m} \to \cU^m\fg \epic \graded^m\cU(\fg) .
\]
This is a homomorphism of graded $k$-algebras. Moreover, since, for 
$x\in \cU_2\fg$, we have 
\[
  U_2\ni x \otimes y - y\otimes x = [x,y]\in U_1 ,
\]
it follows that the map $\varphi$ factors through the symmetric algebra 
$\cS\fg$. 

\begin{theorem}[Poincar\'e-Birkhoff-Witt]
Let $\fg$ be a Lie algebra over $k$. The map 
$\varphi:\cS(\fg)\to \graded\cU(\fg)$ is an isomorphism of graded $k$-agebras. 
\end{theorem}
\begin{proof}
This is \cite[I \S 2.7 th.1]{bourbaki-lie-alg-1-3}. 
\end{proof}

More concretely, suppose $\fg$ has a basis $x_1,\dots,x_n$ of $\fg$. The 
PBW theorem tells us that every element of $\cU\fg$ can be written uniquely as 
a sum 
\[
  \sum_{\bm i} \lambda_{\bm i} x_{\bm i}^{e_{\bm i}} ,
\]
where $\bm i$ ranges over all tuples $(i_1,\dots,i_r)$ for which  
$1\leqslant i_1< \cdots < i_r\leqslant n$. Here we write 
\[
  x^{e_{\bm i}} = x_{i_1}^{e_1} \dotsm x_{i_r}^{e_r} .
\]
The PBW theorem is used in many places -- one application is a simpler (though 
less transparent) construction of free Lie algebras than the one in 
\cite[II \S 2]{bourbaki-lie-alg-1-3}. 





\subsection[Forms of algebraic groups]{Forms of algebraic groups\footnote{Tao Ran Chen}}

We start by defining non-abelian $\h^1$. Let $G$ be a group, $A$ another 
(possibly non-abelian) on which $G$ acts by automorphisms. We define a set 
$Z^1(G,A)$ to consist of functions $f:G\to A$ such that 
$f(\sigma\tau) = f(\sigma)^\tau\cdot f(\tau)$ for all $\sigma,\tau\in G$. Two 
cocyclces $f,f'$ are equivalent if there exists $b\in A$ such that 
$f'(\sigma) = b^\sigma f(\sigma) b^{-1}$ for all $\sigma$. [action by 
exponents on the right.] Let $\h^1(G,A)=Z^1(G,A)/\sim$ be the set of 
equivalence classes of $1$-cocycles. 

Now let $k$ be a field, $K/k$ an extension, $G_{/k}$ an algebraic group. An 
\emph{$K/k$-form of $G$} is an algebraic group $G'_{/k}$ such that 
$G_K\simeq G'_K$. Two $K/k$-forms $G',G''$ are equivalent if 
$G'\simeq G''$ over $k$. Let $E(K/k,G)$ be the set of equivalence classes of 
such forms. 

Example. If $k=\dR$, $K=\dC$, we could let 
\begin{align*}
  G &= \{\mat{x}{y}{-y}{x} : x^2+y^2=1\} \\
  G' &= \{\mat{x}{}{}{y} : x y=1\} .
\end{align*}
These are not isomorphic over $\dR$, but they are isomorphic over $\dC$ via 
\[
  \varphi\mat{x}{y}{-y}{x} = \mat{x+i y}{}{}{x-i y} .
\]

Example. Once again, $k=\dR$ and $K=\dC$. Let $G=\SL(2)_{/\dR}$. What are the 
$\dC/\dR$-forms of $G$? They are: 

$\SL(2)_{/\dR}$

$\SL(\mathbf H/\dR)$. Recall that $\mathbf H$ is the unique division algebra 
over $\dR$, it comes with a norm map $N:\mathbf H\to \dR$. Put 
$\SL(\dH/\dR)=\{x\in \dH:\norm(x)=1\}$. Formally, 
$\SL(\dH/\dC)=\{x\in \dH_\dC:\norm(x)=1\}$. One has 
$\dH_\dC\simeq \matrices_2(\dC)$ via 
\[
  \varphi(a+b i+c j+d k) = \mat{a+b\sqrt{-1}}{c+d\sqrt{-1}}{-c+d\sqrt{-1}}{a-b\sqrt{-1}} .
\]
This induces $\SL(\dH/\dC)\simeq \SL(2)_{/\dC}$. It turns out that these 
are the only $\dC/\dR$-forms of $\SL(2)$, but this is far from obvious. 

We'd like to describe $E(K/k,G)$ using Galois cohomology. Assume $K/k$ is 
Galois with group $\galois(K/k)=\Gamma$. Then $\Gamma$ acts on the set of 
isomorphisms $G'_K\to G_K$ in the obvious way [write this out]. In 
particular, $\Gamma$ acts on $\automorphisms_K(G)$. Let 
$\varphi:G'\iso G$ over $K$. Define a map 
$c:\Gamma\to \automorphisms_K(G)$ by 
$\sigma\mapsto c_\sigma = \varphi^\sigma\circ \varphi^{-1}$. It turns out that 
$c\in Z^1(\Gamma,\automorphisms_K(G))$. To see this, note that 
\[
  c_{\sigma\tau} = \varphi^{\sigma\tau}\circ\varphi^{-1} = (\varphi^{\sigma\tau}\circ\varphi^{-1})^\tau \circ(\varphi^\tau\circ \varphi^{-1}) = c_\sigma^\tau c_\tau .
\]
The class of $c$ in $\h^1(\Gamma,\automorphisms_K G)$ does not depend on the 
particular choice of $\varphi$. Indeed, suppose we have a commutative diagram of 
automorphisms: 
\[\begin{tikzcd}
  G' \ar[r, "\phi"] 
    & G \ar[d, "f"] \\
  & G
\end{tikzcd}\]
We compute 
\[
  \phi^\sigma\circ \phi^{-1} = (f\phi)^\sigma\circ(f\varphi)^{-1} = f^\sigma\circ(\varphi^\sigma\circ\varphi^{-1})\circ f^{-1} \sim \varphi^\sigma \circ \varphi^{-1} .
\]
[Here $\varphi$ and $\phi$ are different: this is very bad!]

We have defined a map $E(K/k,G)\to \h^1(\Gamma,\automorphisms_K G)$. It is a 
theorem that this map is a bijection. 

Returning to our example $\SL(2)$, we see that 
$E(\dC/\dR,\SL_2)=\h^1(\dZ/2,\PSL_2)$, which classifies quaternion algebras 
over $\dR$. There are two such algebras, namely $\matrices_2$ and $\dH$. This 
yields our example above. 

[work out for any field: forms of $\SL_2$ are in bijection with quaternion 
algebras over the field.] 





\subsection{Finite simple groups of Lie type}

Let $G_{/k}$ be a split reductive group, 
$\fg=\ft\oplus \bigoplus_{\alpha\in R} \fg_\alpha$ the root decomposition of 
its Lie algebra. Recall that there are $T$-equivariant embeddings 
$\fg_\alpha\iso U_\alpha\simeq \Ga\subset G$. The group $G$ is generated by 
$T$ and $\{U_\alpha:\alpha\in R\}$. We constructed a bijection bewteen 
isomorphism classes of split connected reductive groups over $k$ and 
root data. 

Let $\sR=(X,R,\check X,\check R)$ be a root datum corresponding to an 
adjoint simple group. For any field $k$, there is a unique split reductive 
group $G_{\sR/k}$ that is adjoint, with root datum $\sR$. So $\sR$ could be 
any of 
$\typeA_n,\typeB_n,\typeC_n,\typeD_n,\typeE_6,\typeE_7,\typeE_8,\typeF_4,\typeG_2$. 

Let $Q=\dZ\cdot R$, $\check Q=\dZ \check R$, $n=\rank(X)$. Define 
$\fg_\dZ = \ft\oplus \bigoplus_{\alpha\in R} \dZ e_\alpha$, where 
$\ft=\check Q\simeq \dZ^n$. Give $\fg_\dZ$ a Lie bracket: 
\begin{align*}
  [\ft,\ft] &= 0 \\
  [x,e_\alpha] &= \langle \alpha,x\rangle e_\alpha \qquad (x\in \ft,\alpha\in R) \\
  [e_\alpha,e_{-\alpha}] &= \check\alpha (\alpha\in R) \\
  [e_\alpha,e_\beta] &= 0 (\alpha+\beta\notin R) \\
  [e_\alpha,e_\beta] &= c_{\alpha,\beta} e_{\alpha+\beta} (\alpha+\beta\in R) .
\end{align*}
The ``mystery constants'' $c_{\alpha,\beta}$ are up to sign 
$c_{\alpha,\beta}=\pm(r+1)$, where the $r,r'$ are maximal integers such 
$\beta-r\alpha,\dots,\beta-\alpha,\beta,\beta+\alpha,\dots,\beta+r'\alpha$ are 
all in $R$. [$r=r_{\alpha,\beta}$; $r'$ not really needed here -- could have 
stopped with $\beta$.] There is a recipe for the signs (different choices). 
In [?], we gave a recipe for the $c_{\alpha,\beta}$ for simply laced diagrams. 

The key point is: we end up with a simple Lie algebra $\fg_\dZ$ over $\dZ$. 
For any field $k$, $\fg_\dZ\otimes k$ is a Lie algebra over $k$ with the same 
root datum. We want to recover a group from $\fg_k=\fg_\dZ\otimes k$. 

We'll construct $G$ such that $\adjoint:G\to \GL(\fg)$ is an embedding. First, 
we'll construct the $U_\alpha\subset \GL(\fg)$. Define a homomorphism 
$u_\alpha:\Ga\to \GL(\fg)$ by $x\mapsto \exp(x\cdot \adjoint(e_\alpha)) = \sum_{n\geqslant 0} \frac{x^n\adjoint(e_\alpha)^n}{n!}$. One has: 
\[
  u_\alpha(x) = 1+x\adjoint(e_\alpha) + x^2\frac{\adjoint(e_\alpha)^2}{2} .
\]
(except in $\typeG_2$. This makes sense if $2$ is invertible. The whole thing 
$\frac 1 2 \adjoint(e_\alpha)^2$ works over $\dZ$\ldots the whole power series 
actually lives over $\dZ$.) So we've defined an injective homomorphism 
$u_\alpha:\Ga\monic \GL(\fg)$; put $U_\alpha=\image(u_\alpha)$. 

We'll construct a split torus $T=\Gm^n$. Identify $X(T)$ with $Q=\dZ R$. Then 
$T\monic \GL(\fg)$, where $T$ acts on $\ft$ trivially, and on 
$\fg_\alpha$ by the character $\alpha\in Q=X(T)$. Let $G$ be the algebraic 
subgroup of $\GL(\fg)$ generated by $T$ and the $\{U_\alpha:\alpha\in R\}$. Then 
$G$ is a connected reductive group over $k$, $T\subset G$ is a split maximal 
torus, we have the same decomposition of $\fg$, and $G$ has the same root 
datum as we started out with. 

[See SGA 3 for this done more carefully.]

[next time for construction of some finite simple groups.]




