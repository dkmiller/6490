% !TEX root = 6490.tex

\section{Canonical filtration}

Recall that if $G_{/k}$ is an algebraic group, there is a canonical filtration 
\[
  1\supset \urad G\subset \rad G\subset G^\circ \subset G .
\]
Each subgroup in the filtration is normal in $G$. The \emph{neutral 
component} $G^\circ$ of $G$, is defined as a functor on $k$-schemes by 
\[
  G^\circ(S) = \{g:S\to G:g(|S|)\subset |G|^\circ\} ,
\]
where $|G|^\circ$ is the connected component of $1$ in the topological space 
underlying $G$. By \cite[VI\textsubscript{A} 2.3.1, 2.4]{sga3-i}, the functor $G^\circ$ 
represents an open, geometrically irreducible, subgroup scheme of $G$.By 
\cite[VI\textsubscript{A} 5.5.1]{sga3-i}, the quotient $\pi_0(G)=G/G^\circ$ is 
\'etale over $k$. 

One calls $\rad G$ the \emph{radical} of $G$, an $\urad G$ the \emph{unipotent 
radical}. All possible quotients in the filtration have names: 
\begin{itemize}
  \item $G^\circ/G$ is \emph{finite}
  \item $G^\circ/\rad G$ is \emph{semisimple}
  \item $G^\circ/\urad G$ is \emph{reductive}
  \item $\rad G/\urad G$ is a \emph{torus}
  \item $\urad G$ is \emph{unipotent} .
\end{itemize}





\subsection{One-dimensional groups}

For simplicity, assume $k$ is algebraically closed. Let $G_{/k}$ be a 
one-dimensional smooth connected linear algebraic group. Currently, we have 
two candidates for $G$, the additive group $\Ga$ and the multiplicative 
group $\Gm$, given by 
\begin{align*}
  \Ga(A) &= (A,+) \\
  \Gm(A) &= A^\times 
\end{align*}
for all $k$-algebras $A$. 

\begin{theorem}\label{thm:1d-class}
Any one-dimensional connected smooth linear algebraic group over an 
algebraically closed field is isomorphic to a unique member of 
$\{\Ga,\Gm\}$. 
\end{theorem}
\begin{proof}
As a variety over $k$, $G$ is smooth and one-dimensional. Thus there is a 
unique smooth proper curve $C_{/k}$ with 
an open embedding $G\hookrightarrow C$. The set $S=C(k)\smallsetminus G(k)$ is 
finite. Take $g\in G(k)$. Then $\phi_g:G\iso G$ 
given by $x\mapsto g\cdot x$ is an automorphism of curves. We can think of 
$\phi_g$ as a rational map $C\to C$; by Zariski's main theorem, this extends 
uniquely to an automorphism $\phi_g:C\iso C$. 
We obtain a group homomorphism $\phi:G(k)\monic \automorphisms(C)$. 

Let 
$g$ be the genus of $C$ (if $k=\dC$, this is just the number of ``holes'' in 
the closed surface $C(\dC)$). By \cite[IV ex 5.2]{hartshorne-1977}, if $g\geqslant 2$, then 
$\automorphisms(C)$ is finite. It follows that $g\leqslant 1$. Since $S$ is 
finite, there is an infinite subgroup $H\subset G(k)$ that acts trivially on 
$S$. Write $\automorphisms(C,S)$ for the group of automorphisms of $C$ that 
are trivial on $S$; there is an injection $H\monic \automorphisms(C,S)$. If 
$g=1$, then $\automorphisms(C,S)$ is finite \cite[IV cor 4.7]{hartshorne-1977}. 

We've reduced to the case $g=0$. We can assume 
$G\subset C=\dP^1_{/k} = \dA^1_{/k}\cup \{\infty\}$. It is known 
that $\automorphisms(\dP^1)=\PGL(2)$ as schemes, so in particular 
$\automorphisms(\dP_{/k}^1) = \PGL_2(k)$ \cite[IV 7.1.1]{hartshorne-1977}. 
The action of $\PGL_2(k)$ is via 
fractional linear transformations: 
\[
  \begin{pmatrix} a & b \\ c & d \end{pmatrix} x = \frac{a x+b}{c x+d} .
\]
For any distinct $\alpha,\beta,\gamma\in \dP^1(k)$, there is a unique 
$g\in \PGL_2(k)$ such that $g(\alpha)=0$, $g(\beta)=1$, and $g(\gamma)=\infty$ 
[this is equivalent to $M_{0,3}=\ast$, it can be verified by a direct 
computation]. If $\# S\geqslant 3$, then this shows that 
$\automorphisms(\dP^1_{/k},S)=1$, which doesn't work. We now get two cases, 
$\# S\in \{1,2\}$. So without loss of generality, 
$G=\dP^1_{/k}\smallsetminus \{\infty\}$ or 
$G=\dP^1_{/k}\smallsetminus \{0,\infty\}$. So as a variety, $G=\Ga$ or $\Gm$. 

We'll treat the case $G=\dP^1_{/k}\smallsetminus \{0,\infty\}$. We can assume 
$1\in \dP^1$ is the identity of $G$. Pick $g\in G(k)\subset k^\times$; then 
$\phi_g:x\mapsto g x$ must be of the form $x\mapsto \frac{a x+b}{c x+d}$. 
Moreover, $\phi_g\{0,\infty\}=\{0,\infty\}$. Either $\phi_g(x)=a x$ 
for $a\in k^\times$, or $\phi_g(x)=a/x$. In the latter case, $\phi_g$ has a 
fixed point, namely $\sqrt a$. But translation has no fixed points, so 
$\phi_g(x)=a x$. Since $g=\phi_g(1)=a$, it follows that $G=\Gm$. 
\end{proof}

The group $\Gm$ is reductive (better, a torus), and $\Ga$ is unipotent. For a 
general (i.e., not necessarily linnear) connected one-dimensional algebraic 
group, there are many possibilities, namely elliptic curves. Even over $\dC$, 
the collection of elliptic curves is one-dimensional when interpreted as a 
variety in the appropriate sense. 

\autoref{thm:1d-class} is a very useful result. It will arise many times in 
the proof of various important theorems. 





\subsection{Solvable groups}

Let $G_{/k}$ be an algebraic group, and let $\fg=\lie(G)$. It turns out that 
$\lie(G^\circ)=\fg$. There is a canonical sub-Lie algebra 
$\lierad(\fg)\subset \fg$; in characteristic zero, this will determine a 
connected linear algebraic subgroup $\rad G\subset G$. 

For the moment, let $\fg$ be an arbitrary $k$-Lie algebra. 
Let $\derived\fg=[\fg,\fg]$ be the subspace 
of $\fg$ generated by $\{[x,y]:x,y\in \fg\}$.
The subspace $\derived\fg$ is actually an ideal, and the quotient 
$\fg/\derived\fg$ is commutative. Moreover, $\derived\fg$ is the smallest 
ideal in $\fg$ with this property. 

Every Lie algebra comes with a canonical descending filtration 
$\derived^\bullet\fg$, called the \emph{derived series}. It is defined by 
\begin{align*}
  \derived^1\fg &= \derived \fg \\
  \derived^{n+1}\fg &=\derived(\derived^n\fg) .
\end{align*}

\begin{definition}
A Lie algebra $\fg$ is \emph{solvable} if the filtration $\derived^\bullet\fg$ 
is separated, that is if $\derived^n\fg=0$ for some $n$. 
\end{definition}

\begin{lemma}
A Lie algebra $\fg$ is solvable if and only if there exists a decreasing 
filtration $\fg=\fg_0\supset \cdots \supset \fg_n=0$ with each $\fg_{i+1}$ an 
ideal in $\fg_i$, and with $\fg_i/\fg_{i+1}$ commutative. 
\end{lemma}
\begin{proof}
$\Rightarrow$. This follows from the fact that 
$\derived^i\fg / \derived^{i+1}\fg$ is abelian for each $i$. 

$\Leftarrow$. Since $\fg_0/\fg_1$ is commutative, we get 
$\derived\fg\supset\fg_1$. More generally, we get $\derived^i\fg\supset \fg_i$ 
by induction, so $\fg_i=0$ for $i\gg 0$ implies $\derived^i\fg=0$ for $i\gg 0$. 
\end{proof}

\begin{lemma}
Let $\fg$ be a finite-dimensional Lie algebra. Then $\fg$ has a unique maximal 
solvable ideal; it is called the \emph{radical} of $\fg$, and denoted 
$\lierad(\fg)$. 
\end{lemma}
\begin{proof}
Let $\fa$ be an ideal of $\fg$ that is solvable, and has $\dim(\fa)$ maximal. 
Let $\fb$ be any solvable ideal. Then $\fa+\fb$ is an ideal and 
$(\fa+\fb)/\fa$ is solvable. From the general fact that solvable Lie algebras 
are closed under extensions, we get that $\fa+\fb$ is solvable, so 
$\fa+\fb=\fa$, whence $\fb\subset \fa$. 
\end{proof}

\begin{definition}
A Lie algebra $k$ is \emph{semisimple} if $\lierad(\fg)=0$. 
\end{definition}

\begin{example}
Consider the Lie algebra 
\[
  \fb_n = \{x\in \gl_n : x_{i,j}=0\text{ for all }i>j\} .
\]
We claim that $\fb_n$ is solvable. This follows from the fact that 
\[
  \derived^r \fb_n = \{x\in \gl_n : x_{i,j=0}\text{ for all }i>j-r\} .
\]
To see that this is true, note that it is trivially true for $r=0$, so 
assume it is true for some $r$, and let $x,y\in \derived^r \fb_n$. Note that 
\begin{align*}
  [x,y]_{i,j} 
    &= \sum_k (x_{i,k}y_{k,j}-y_{i,k}x_{k,j}) \\ \tag{$\ast$}\label{eq:bn}
    &= \sum_{i-r\leqslant k \leqslant j+r} (x_{i,k}y_{k,j}-y_{i,k}x_{k,j})
\end{align*}
Moreover, when $i>j+r+1$, then all of the terms in \eqref{eq:bn} are 
zero, whence the result. In some sense, $\fb_n$ is the 
``only'' example of a solvable Lie algebra over an algebraically closed 
field. 
\end{example}

\begin{theorem}[Lie-Kolchin]
Let $k$ be an algebraically closed field, $\fg$ a finite-dimensional 
solvable $k$-Lie algebra. Then there is an injective Lie homomorphism 
$\fg\monic \fb_n$ for some $n$. 
\end{theorem}
\begin{proof}
In characteristic zero, this follows directly from Corollary 2 of 
\cite[I \S 5.3]{bourbaki-lie-alg-1-3} applied to the adjoint representation. 
\end{proof}

\begin{example}
One can check that: 
\begin{align*}
  \lierad(\gl_2) &= \left\langle \begin{pmatrix} 1 \\ & 1 \end{pmatrix}\right\rangle \\
  \derived^n(\gl_2) &= \Sl_2 && \text{for all }n\geqslant 1 
\end{align*}
This is because $\Sl_2$ is simple (has no non-trivial ideals). 
\end{example}

\begin{definition}
Let $G_{/k}$ be a linear algebraic group. We say $G$ is \emph{solvable} if 
there is a sequence of algebraic subgroups 
$G=G_0\supset G_1\supset \cdots \supset G_n=1$ such that 
\begin{enumerate}
  \item each $G_{i+1}$ is normal in $G_i$, 
  \item each $G_i/G_{i+1}$ is commutative. 
\end{enumerate}
\end{definition}

\begin{example}
Let $G=B(n)\subset \GL(n)$ be the subgroup of upper-triangular matrices (the 
letter $B$ represents ``Borel''). This is solvable, as is witnessed by the 
filtration 
\begin{align*}
  G_1 &= \{g\in \GL(n):g_{i i}=1\text{ for all }i\} \\
  G_r &= \{g\in B(n)_2:g_{i j}=0\text{ for all }i<j+r\} && r\geqslant 1 .
\end{align*}
The map $G_0\to \Gm^n$ defined by $(a_{i j})\mapsto (a_{11},\dots,a_{nn})$ 
induces an isomorphism $G_0/G_1\iso \Gm^n$. Similarly 
$G_1\to \Ga^{n-1}$ defined by $(a_{i j})\mapsto (a_{1,2},\dots,a_{n-1,n})$ 
induces an isomorphism $G_1/G_2\iso \Ga^{n-1}$. In general, for $i\geqslant 0$, 
we have $G_{i}/G_{i+1} \iso \Ga^{n-i}$. 

We could have chosen our filtration in such a way that 
$G_i/G_{i+1}\in \{\Ga,\Gm\}$. 
\end{example}

\begin{definition}
Let $G_{/k}$ be a linear algebraic group. The \emph{derived} group 
$\derived G$ (or $G'$, or $G^\mathrm{der}$) is the smallest normal subgroup of 
$G$ such that $G/\derived G$ is commutative. 
\end{definition}

The basic idea is that $\derived G$ is the algebraic group generated by 
$\{x y x^{-1} y^{-1}:x,y\in G\}$. 

\begin{theorem}\label{thm:derived-nice}
Let $G_{/k}$ be a smooth linear algebraic group. Then $\derived G$ exists and 
is smooth, and if $G$ is connected then so is $\derived G$. 
\end{theorem}
\begin{proof}
This essentially follows from \cite[VI\textsubscript{B} 7.1]{sga3-i}. 
\end{proof}

Just as for Lie algebras, we can define a filtration 
\begin{align*}
  \derived^1 G &= \derived G \\
  \derived^{n+1} G &= \derived(\derived^n G) .
\end{align*}

\begin{theorem}
A linear algebraic group $G$ is solvable if and only if $\derived^n G=1$ for 
some $n$. 
\end{theorem}

\begin{example}
If $G\subset B(n)\subset \GL(n)$, then $G$ is solvable. Indeed, this 
follows from $\derived^\bullet G\subset \derived^\bullet B(n)$. 
\end{example}

\begin{theorem}[Lie-Kolchin]\label{thm:lie-kolchin}
Suppose $k$ is algebraically closed. Let $G\subset \GL(n)_{/k}$ be a 
connected solvable algebraic group. Then there exists $x\in \GL_n(k)$ such that 
$x G x^{-1}\subset B(n)$. 
\end{theorem}
\begin{proof}
Let $V=k^{\oplus n}$, and consider $V$ as a representation of $G$ via the 
inclusion $G\hookrightarrow \GL(V)$. It is sufficient to prove that $V$ 
contains a one-dimensional subrepresentation, for then we could induct on 
$\dim(V)$. For simplicity, we assume $G$ is smooth. If $G$ is commutative, 
then the set $G(k)$ is a family of mutually commuting endomorphisms of $V$. 
It is known that such sets are mutually triangularisable, i.e.~can be 
conjugated to lie within $B(n)$. (This follows from the Jordan decomposition.) 

In the general case, we may assume the claim is true for $\derived G$. 
Recall that $\characters^\ast(\derived G)=\hom(\derived G,\Gm)$ is the group of 
\emph{characters} of $\derived G$. The group $G$ acts on 
$\characters^\ast(\derived G)$ by conjugation: 
\[
  (g\cdot \chi)(h) = \chi(g h g^{-1}) .
\]
Even better, we can define $\characters^\ast(\derived G)$ as a group functor: 
\[
  \characters^\ast(\derived G)(A)=\hom_{\groups_{/A}}((\derived G)_{/A},(\Gm)_{/A}) .
\]
By \cite[11.4.2]{sga3-ii}, this is represented by a smooth separated scheme 
over $k$. We define a subscheme 
\[
  \Delta(A) = \{\chi\in \characters^\ast(\derived G)(A):\chi\text{ factors through }G_{/A}\hookrightarrow \GL(V)_{/A}\} .
\]
Note that $\Delta$ is finite and nonempty, and $G$ is connected. Thus the 
induced action of $G$ on $\Delta$ is trivial, so $G$ fixes some 
$\chi\in \Delta(k)$. Let $V_\chi$ be the suprepresentation 
of $V$ generated by all $\chi$-typical vectors, i.e. 
\[
  V_\chi(A) = \langle v\in V_{/A}:g\cdot v=\chi(g) v\text{ for all }g\in G(A)\} .
\]
After replacing $V$ by $V_\chi$, we may assume $\derived G$ acts on $V$ by a 
character $\chi$. That is, as a subgroup of $\GL(n)$, $\derived G\subset \Gm$. 
Since the determinant map kills commutators, 
$\derived G\subset \Gm\cap \SL(n)=\dmu_n$, so $\derived G$ is finite. Since $G$ 
connected, \autoref{thm:derived-nice} tells us that $\derived G=1$, so $G$ is 
abelian, and we're done. 
\end{proof}

\begin{example}
If $k$ is algebraically closed and $G_{/k}\subset \GL(n)_{/k}$ is solvable, 
then there exists a filtration $G=G_0\supset \cdots \supset G_n=1$ such that 
$G_0/G_1\simeq \Gm^r$, and all higher $G_i/G_{i+1}\simeq \Ga$. We will see 
that this property determines $G_1$. We know there is $v\in k^{\oplus n}$ which 
is a common eigenvector of all $g\in \derived G(k)$. This gives us a 
character $\chi:\derived G\to \Gm$. 
\end{example}

\begin{example}
This shows that \autoref{thm:lie-kolchin} does not hold over non-algebraically closed 
fields. Let 
\[
  G_{/\dR} = \left\{\begin{pmatrix} a & -b \\ b & a \end{pmatrix}\right\}\subset \GL(2)_{/\dR} .
\]
Note that $G(\dR)\simeq \dC^\times$ (in fact, $G=\weil_{\dC/\dR} \Gm$) via 
\[
  \begin{pmatrix} a & -b \\ b & a \end{pmatrix} \leftrightarrow a+ b i .
\]
The group $G$ is commutative (hence solvable), but it is not conjugate to 
$B(2)$ by an element of $\GL_2(\dR)$. Indeed, note that 
$\begin{pmatrix} & -1 \\ 1 \end{pmatrix}$ is not diagonalizable in $\GL_2(\dR)$ 
because it has eigenvalues $\pm i$. 
\end{example}

For a general group, we'll have a subgroup $\rad G$, the \emph{radical} of 
$G$. It will be the largest connected normal solvable subgroup of $G$. It 
will turn out that $\lierad(\fg) = \lie(\rad G)$. 





\subsection{Quotients}

Let $G_{/k}$ be an algebraic group, $N\subset G$ a sub-algebraic group. As 
functors of points, $N(A)$ is a normal subgroup of $G(A)$ for all $k$-algebras 
$A$. 

\begin{example}
Let $k=\dR$, and consider $\dmu_2=\ker(\Gm\xrightarrow{(-)^2}\Gm)$ over $k$. 
Should we consider the sequence 
\[
  1\to \dmu_2\to \Gm\xrightarrow 2 \Gm \to 1 
\]
to be exact? On $\dC$-points, this is the sequence 
\[
  1\to \{\pm 1\} \to \dC^\times \xrightarrow 2 \dC^\times \to 1
\]
which is certainly exact. So we will write $\Gm=\Gm/\dmu_2$. Note however 
that the sequence for $\dR$-points is 
$1\to \{\pm 1\}\to \dR^\times \xrightarrow 2 \dR^\times$, which is \emph{not} 
exact on the right. 
\end{example}

In general, if $1\to N\to G\to H\to 1$ is an ``exact sequence'' of algebraic 
groups, we will not necessarily have surjections from the $A$-points of $G$ to 
the $A$-points of $H$. 

\begin{definition}
Let $N_{/k}\subset G_{/k}$ be a sub-algebraic group. A \emph{quotient} of $G$ 
by $N$ is a homomorphism $G\xrightarrow q Q$ of algebraic groups over $k$ with kernel $N$, 
such that if $G\xrightarrow\phi G'$ vanishes on $N$, then there is a unique 
$\psi:Q\to G'$ such that the following diagram commutes:
\[\begin{tikzcd}
  1 \ar[r] 
    & N \ar[r] 
    & G \ar[r, "q"] \ar[dr, "\phi"]
    & Q \ar[d, "\psi", dashrightarrow] \\
  & & & G'
\end{tikzcd}\]
\end{definition}

\begin{theorem}
Quotients exist in the category of (possibly non-smooth) affine group schemes 
of finite type over $k$. 
\end{theorem}
\begin{proof}
This is \cite[VII 8.1]{milne-AGS}. 
\end{proof}

If $G/H$ is the quotient of $G$ by a subgroup $H$, it is generally true that 
$(G/H)(\bar k)=G(\bar k)/H(\bar k)$. 

\begin{hard}
There is a more abstract, but powerful approach to defining quotients. 
Let $\schemes k$ be the category of schemes over $\spectrum(k)$. We can regard 
any scheme over $k$ as an fppf sheaf on $\schemes k$ via the Yoneda embedding. 
If $H\subset G$ is a closed subgroup-scheme, we write $G/H$ for the quotient 
sheaf of $S\mapsto G(S)/H(S)$ in the fppf topology. By 
\cite[VI\textsubscript{A} 3.2]{sga3-i}, this quotient sheaf is representable. 
By general nonsense, it will satisfy more elementary definition of quotient. 
\end{hard}





\subsection{Unipotent groups}

Recall that for the moment, $B(n)\subset \GL(n)$ is the subgroup of upper 
triangular matrices and $U(n)\subset B(n)$ is the subgroup of \emph{strictly} 
upper-triangular matrices. 

Recall that in our filtration of $B(n)\subset \GL(n)$, we had 
$B(n)/U(n)\simeq \Gm^n$, and all further quotients were $\Ga^r$ for varying 
$r$. We'd like to generalize this to an arbitrary connected solvable groups 
over algebraically closed fields. For the moment though, $k$ need not be 
algebraically closed. 

\begin{definition}
Let $G_{/k}$ be an algebraic group. We say $G$ is \emph{unipotent} if it admits 
a filtration $G=G_0\supset \cdots \supset G_n=1$ of closed subgroups defined 
over $k$ such that 
\begin{enumerate}
  \item each $G_{i+1}$ is normal in $G_i$, 
  \item each $G_i/G_{i+1}$ is isomorphic to a closed subgroup of $\Ga$. 
\end{enumerate}
\end{definition}

Clearly, unipotent groups are solvable. If $k$ has characteristic zero, we can 
assume $G_i/G_{i+1}\simeq \Ga$. If $k$ has characteristic $p>0$, then we have 
to worry about things like $\dalpha_p=\ker(\Ga\xrightarrow p \Ga)$. 
Fortunately, by \cite[XVII 1.5]{sga3-ii}, the only possible closed subgroups of 
$\Ga$ over a field of characteristic $p$ are $0$, $\Ga$ and extensions of 
$(\dZ/p)^r$ by $\dalpha_{p^e}$. 

\begin{theorem}
Let $G_{/k}$ be a connected linear algebraic group. Then $G$ is unipotent 
if and only if it is isomorphic to a closed subgroup of some $U(n)_{/k}$. 
\end{theorem}
\begin{proof}
This follows directly from \cite[XV 2.2]{milne-AGS}. 
\end{proof}

If $G_{/k}$ is unipotent and $\rho:G\to \GL(n)$ is a representation, then for 
all $g\in G(\bar k)$, the matrix $\rho(g)$ is unipotent, i.e.~$(\rho(g)-1)^n=0$ 
for some $n\geqslant 1$. Indeed, the result in \cite{milne-AGS}

See \cite[XVII 3.5]{sga3-ii}. 

[find source] The converse holds as long as the group 
is smooth [maybe need characteristic zero: find source]

\begin{theorem}
Let $G_{/k}$ be a connected solvable smooth group over a perfect field $k$. 
Then there exists a unique connected normal $G_\unipotent\subset G$ such that 
\begin{enumerate}
  \item $G_\unipotent$ is unipotent, 
  \item $G/G_\unipotent$ is of multiplicative type .
\end{enumerate}
\end{theorem}

[mention pseudo-reductive groups for non-perfect fields, cite 
CGP.]

[Define ``of multiplicative type.'']

In particular, if $k$ has characteristic zero, $G/G_\unipotent$ will be a 
torus. If moreover $k=\bar k$, then $G/G_\unipotent\simeq \Gm^r$. In general, 
$G/G_\unipotent$ will be a closed subgroup of a torus. 

[This construction works well with base-change: cite SGA.]

\begin{example}
If $G=B(n)\subset \GL(n)$, then $G_\unipotent=U(n)$, the subgroup of strictly 
upper-triangular matrices. 
\end{example}





\subsection{Review of canonical filtration}

For this section, assume $k$ is a perfect field. 

Let $G_{/k}$ be a smooth connected linear algebraic group. In [cite], we defined 
$\rad G$, the radical of $G$, to be the largest normal subgroup of $G$ that is 
connected and solvable. 

The \emph{unipotent radical} $\urad G$ of $G$, is by definition 
$(\rad G)_\unipotent$. As in \autoref{sec:str-thry-lag}, we have a diagram 
\begin{center}
\begin{tikzpicture}
  \matrix (m) [
    matrix of nodes,
    nodes={anchor=west}
  ] {
    & $G$ \\
    & $\cup$ & finite\\
    connected & $G^\circ$ \\
    & $\cup$ & semisimple \\
    solvable & $\rad G$ \\
    & $\cup$ & torus \\
    unipotent & $\urad G$ \\
    & $\cup$ \\
    & $1$ \\
  };
  
  \draw [decoration={brace,amplitude=0.5em},decorate]
        (m-3-2.north -| m.east) -- (m-7-2.south -| m.east) 
        node [midway,xshift=1cm] {reductive};
\end{tikzpicture}
\end{center}

So the group $\urad G$ is unipotent, $\rad G$ is solvable. The quotient 
$\rad G/\urad G$ is a torus (we could define a torus to be a solvable group 
with trivial unipotent radical.) 

The quotient $G/\rad G$ is \emph{semisimple}, and $G/\urad G$ is 
\emph{reductive}. There is a good structure theory for semisimple groups. 

\begin{definition}
Let $k$ be a perfect field. A connected linear algebraic group $G_{/k}$ 
is \emph{reductive} if $\urad G=1$. 
\end{definition}

[mention $\urad(G_{\bar k})$... pseudo-reductive groups]

\begin{example}
The group $\GL(n)$ is reductive, even though $\rad \GL(n)=\Gm$ is nontrivial. 
\end{example}

\begin{definition}
A connected linear algebraic group $G$ is \emph{semisimple} if 
$\rad G=1$. 
\end{definition}

\begin{example}
The group $\SL(n)$ is semisimple. The subgroup $\dmu_n= \zentrum(\SL_n)$ is 
a solvable normal subgroup, but it's not connected (or not smooth, if the 
base characteristic divides $n$). 
\end{example}





\subsection{Jordan decomposition}

Let $k$ be a field, $V$ a finite-dimensional $k$-vector space. Choose some 
$g\in \GL(V)=\automorphisms_k(V)$. Recall the \emph{characteristic 
polynomial} of $g$ will factor (over $\bar k$) as
\[
  \det(t\cdot 1-g) = \prod_i (t-\lambda_i)^{n_i}
\]
where the $\lambda_i$ are distinct. Let 
\[
  V_{\lambda_i} = \{v\in V:(g-\lambda_i\cdot 1)^{n_i} v=0\} .
\]
Consider the action of $k[t]$ on $V$ in which $t$ acts via $g$. Then $V$ is 
a finitely generated torsion $k[t]$-module, so we can write it as 
$k[t]/f_1\oplus \cdots \oplus k[t]/f_r$. (This is just the fundamental theorem 
of finitely generated modules over a principal ideal domain.) We can take the 
$f_i$ to be powers of irreducible polynomials. If $k=\bar k$, then we get 
$V=\bigoplus_i k[t]/(t-\lambda_i)^{n_i}$, i.e.~$V=\bigoplus V_{\lambda_i}$. 

As a $k$-vector space, $k[t]/(t-\lambda)^n$ has basis 
$\{v_i=(t-\lambda)^{n-i}\}_{i=1}^n$. Moreover, 
\begin{align*}
  t v_1 &= \lambda v_1 \\
  t v_2 &= \lambda v_2 + v_2 \
  \cdots 
\end{align*}
When we write $g$ with respect to this basis, we get the $n\times n$ matrix 
$J_n(\lambda)$ which is $\lambda$ along the diagonal, $1$ just above the 
diagonal, and zeros everywhere else. We end up with a direct sum decomposition 
(with respect to some basis) $g=\bigoplus J_{n_i}(\lambda_i)$. 

[go back and clarify]

In all of this, we used $k=\bar k$. Over a general field $k$, we'll be able to 
write a matrix $g$ as the sum of a diagonal matrix and a strictly upper 
triangular (hence nilpotent) matrix. Suppose we have written 
$g=g_\semisimple+g_\nilpotent$, where $g_\semisimple$ is diagonal and 
$g_\nilpotent$ is nilpotent. It turns out that $g_\semisimple$ is uniquely 
determined, and is a polynomial in $g$. That is, there exists a polynomial 
$Q\in k[x]$ such that $g_\semisimple=Q(g)$ for all $g$. [cite source]
Similarly $g_\nilpotent=g-g_\semisimple$ is a polynomial in $g$. [all this only 
over $k=\bar k$.]

Now let $k$ be an arbitrary perfect field, $V$ a finite-dimensional $k$-vector 
space. 

\begin{definition}
An element $g\in \gl(V)$ is \emph{semisimple} if it is diagonalizable after 
base-change to an extension of $k$.  
\end{definition}

\begin{definition}
An element $g\in \gl(V)$ is \emph{nilpotent} if $g^n=0$ for some $n\geqslant 1$. 
\end{definition}

\begin{definition}
An element $g\in \GL(V)$ is \emph{unipotent} if $g-1$ is nilpotent. 
\end{definition}

\begin{theorem}[Additive Jordan decomposition]
For any $g\in \gl(V)$, there exists unique elements $g_\semisimple\in \GL(V)$, 
$g_\nilpotent\in \gl(V)$, such that 
\begin{enumerate}
  \item $g=g_\semisimple+g_\nilpotent$, 
  \item $g_\semisimple$ is semisimple, 
  \item $g_\nilpotent$ is nilpotent, and 
  \item $[g_\semisimple,g_\nilpotent]=0$. 
\end{enumerate}
\end{theorem}
\begin{proof}
Case 1: the eigenvalues of $g$ lie in $k$. Existence you've already done. 
Suppose $g=g_\semisimple+g_\nilpotent = h_\semisimple+h_\nilpotent$ are two 
distinct decompositions. Then 
$g_\semisimple-h_\semisimple=-g_\nilpotent+h_\nilpotent$, and 
$g_\semisimple$, $g_\nilpotent$ commute with $h_\semisimple$, $h_\nilpotent$. 
This is because $g_\semisimple$ and $g_\nilpotent$ are polynomials in $g$. It 
follows that $-g_\nilpotent+h_\nilpotent$ is nilpotent. The matrices 
$g_\semisimple$, $h_\semisimple$ commute, so they are simultaneously 
diagonalizable. So $g_\semisimple-h_\semisimple$ is still semisimple. But 
$g_\semisimple-h_\semisimple$ is also nilpotent, so 
$g_\semisimple=h_\semisimple$. 

Case 2: the eigenvalues of $g$ may not lie in $k$. Take a Galois extension 
$K/k$ containing all the eigenvalues of $g$. (For example, we could let $K$ be 
the extension of $k$ generated by the eigenvalues of $g$.) We can write 
$g=g_\semisimple+g_\nilpotent$, where 
$g_\semisimple,g_\nilpotent\in \gl(V)\otimes K$. For any 
$\sigma\in \galois(K/k)$, we have $\sigma(g)=g$, because $g\in \gl(V)$. But 
then $g=\sigma(g_\semisimple)+\sigma(g_\nilpotent)$, and this is another 
additive Jordan decomposition of $g$. By uniqueness in the first case, we 
see that $g_\semisimple$ and $g_\nilpotent$ are fixed by $\sigma$. Since 
$\sigma$ was arbitrary, $g_\semisimple$ and $g_\nilpotent$ lie in 
$\gl(V)$. 
\end{proof}

[change notation to $x\in \gl(V)$]

\begin{example}
This shows that the perfectness hypothesis on $k$ is necessary. Let 
$k=\dF_2(t)$. Consider the matrix $x=\begin{pmatrix} & 1 \\ t \end{pmatrix}$. 
This has characteristic polynomial $x^2-t$, so its only eigenvalue is 
$\sqrt 2$ appearing with multiplicity two. So the only way we could 
give $x$ a Jordan decomposition is 
$\begin{pmatrix} \sqrt t \\ & \sqrt t\end{pmatrix} + \begin{pmatrix} -\sqrt t & 1 \\ t & -\sqrt t\end{pmatrix}$. 
The problem is, this doesn't work after a separable base change. More 
concretely, this doesn't descend back to $k$. 
\end{example}

\begin{theorem}[Multiplicative Jordan decomposition]
Let $k$ be a perfect field, $V$ a finite-dimensional $k$-vector space. For any 
$g\in \GL(V)$, there exists unique $g_\semisimple, g_\unipotent\in \GL(V)$ such 
that 
\begin{enumerate}
  \item $g=g_\semisimple g_\unipotent$, 
  \item $g_\semisimple$ is semisimple, 
  \item $g_\unipotent$ is unipotent, and 
  \item $g_\semisimple$ and $g_\unipotent$ commute. 
\end{enumerate}
\end{theorem}
\begin{proof}
Recall that we can write $g=g_\semisimple+g_\nilpotent$. Just rewrite it as 
$g_\semisimple(1+g_\semisimple^{-1} g_\nilpotent)$, and note that 
$g_\semisimple^{-1} g_\nilpotent$ is nilpotent because 
$g_\nilpotent$ is nilpotent and $g_\semisimple$ commutes with $g_\nilpotent$. 
So $g_\unipotent = 1+g_\semisimple^{-1} g_\nilpotent$. Uniqueness is similarly 
easy. 
\end{proof}

We'd like to connect the theory of Jordan decomposition with algebraic groups. 
Let $G_{/k}$ be a linear algebraic group, $\rho:G\to \GL(n)$ a representation. 
For $g\in G(k)$, we can write $\rho(g)=h_\semisimple h_\unipotent$. It is a 
beautiful fact that in fact $h_\semisimple=\rho(g_\semisimple)$, 
$h_\unipotent=\rho(g_\unipotent)$ for uniquely determined 
$g_\semisimple, g_\unipotent\in G(k)$. 

Similarly, let $\fg$ be a $k$-Lie algebra, $\rho:\fg\to \gl(V)$ a 
representation. For $x\in \fg$, the decomposition 
$\rho(x)=\rho(x)_\semisimple+\rho(x)_\nilpotent$ comes from a uniquely 
determined decomposition $x=x_\semisimple+x_\nilpotent$. [find assumptions, 
cite source.]

Unfortunately, not all Lie algebras have this property. For example, this fails 
for Lie algebras which don't come from algebraic groups. 

Let $f:G\to H$ be a homomorphism of linear algebraic groups defined over a 
perfect field $k$. Let $g\in G(k)$. Then 
$f(g)_\semisimple=f(g_\semisimple)$ and $f(g)_\unipotent=f(g_\unipotent)$. 
In other words, the Jordan decomposition is functorial. 
It follows that if $G_{/k}$ is an affine algebraic group (without a choice of 
embedding $G\hookrightarrow \GL(n)$), then the multiplicative Jordan 
decomposition within $G$ is well-defined, independent of any choice of 
embedding. 




