% !TEX root = 6490.tex

\section{Canonical filtration}

Recall that if $G_{/k}$ is an algebraic group, there is a canonical filtration 
\[
  1\supset \urad G\subset \rad G\subset G^\circ \subset G .
\]
Each subgroup in the filtration is normal in $G$. The \emph{neutral 
component} $G^\circ$ of $G$, is defined as a functor on $k$-schemes by 
\[
  G^\circ(S) = \{g:S\to G:g(|S|)\subset |G|^\circ\} ,
\]
where $|G|^\circ$ is the connected component of $1$ in the topological space 
underlying $G$. By \cite[VI\textsubscript{A} 2.3.1, 2.4]{sga3-i}, the functor $G^\circ$ 
represents an open, geometrically irreducible, subgroup scheme of $G$.By 
\cite[VI\textsubscript{A} 5.5.1]{sga3-i}, the quotient $\pi_0(G)=G/G^\circ$ is 
\'etale over $k$. 

One calls $\rad G$ the \emph{radical} of $G$, an $\urad G$ the \emph{unipotent 
radical}. All possible quotients in the filtration will have names, like 
semisimple (for $G^\circ/\rad G$) or reductive (for $G^\circ/\urad G$). 





\subsection{One-dimensional groups}

For simplicity, assume $k$ is algebraically closed. Let $G_{/k}$ be a 
one-dimensional smooth connected linear algebraic group. Currently, we have 
two candidates for $G$, the additive group $\Ga$ and the multiplicative 
group $\Gm$, given by 
\begin{align*}
  \Ga(A) &= (A,+) \\
  \Gm(A) &= A^\times 
\end{align*}
for all $k$-algebras $A$. 

\begin{theorem}\label{thm:1d-class}
Any one-dimensional connected smooth linear algebraic group over an 
algebraically closed field is isomorphic to a unique member of 
$\{\Ga,\Gm\}$. 
\end{theorem}
\begin{proof}
As a variety over $k$, $G$ is smooth and one-dimensional. Thus there is a 
unique smooth proper curve $C_{/k}$ with 
an open embedding $G\hookrightarrow C$. The set $S=C(k)\smallsetminus G(k)$ is 
finite. Take $g\in G(k)$. Then $\phi_g:G\iso G$ 
given by $x\mapsto g\cdot x$ is an automorphism of curves. We can think of 
$\phi_g$ as a rational map $C\to C$; by Zariski's main theorem, this extends 
uniquely to an automorphism $\phi_g:C\iso C$. 
We obtain a group homomorphism $\phi:G(k)\monic \automorphisms(C)$. 

Let 
$g$ be the genus of $C$ (if $k=\dC$, this is just the number of ``holes'' in 
the closed surface $C(\dC)$). By \cite[IV ex 5.2]{hartshorne-1977}, if $g\geqslant 2$, then 
$\automorphisms(C)$ is finite. It follows that $g\leqslant 1$. Since $S$ is 
finite, there is an infinite subgroup $H\subset G(k)$ that acts trivially on 
$S$. Write $\automorphisms(C,S)$ for the group of automorphisms of $C$ that 
are trivial on $S$; there is an injection $H\monic \automorphisms(C,S)$. If 
$g=1$, then $\automorphisms(C,S)$ is finite \cite[IV cor 4.7]{hartshorne-1977}. 

We've reduced to the case $g=0$. We can assume 
$G\subset C=\dP^1_{/k} = \dA^1_{/k}\cup \{\infty\}$. It is known 
that $\automorphisms(\dP^1)=\PGL(2)$ as schemes, so in particular 
$\automorphisms(\dP_{/k}^1) = \PGL_2(k)$ \cite[IV 7.1.1]{hartshorne-1977}. 
The action of $\PGL_2(k)$ is via 
fractional linear transformations: 
\[
  \begin{pmatrix} a & b \\ c & d \end{pmatrix} x = \frac{a x+b}{c x+d} .
\]
For any distinct $\alpha,\beta,\gamma\in \dP^1(k)$, there is a unique 
$g\in \PGL_2(k)$ such that $g(\alpha)=0$, $g(\beta)=1$, and $g(\gamma)=\infty$ 
[this is equivalent to $M_{0,3}=\ast$, it can be verified by a direct 
computation]. If $\# S\geqslant 3$, then this shows that 
$\automorphisms(\dP^1_{/k},S)=1$, which doesn't work. We now get two cases, 
$\# S\in \{1,2\}$. So without loss of generality, 
$G=\dP^1_{/k}\smallsetminus \{\infty\}$ or 
$G=\dP^1_{/k}\smallsetminus \{0,\infty\}$. So as a variety, $G=\Ga$ or $\Gm$. 

We'll treat the case $G=\dP^1_{/k}\smallsetminus \{0,\infty\}$. We can assume 
$1\in \dP^1$ is the identity of $G$. Pick $g\in G(k)\subset k^\times$; then 
$\phi_g:x\mapsto g x$ must be of the form $x\mapsto \frac{a x+b}{c x+d}$. 
Moreover, $\phi_g\{0,\infty\}=\{0,\infty\}$. Either $\phi_g(x)=a x$ 
for $a\in k^\times$, or $\phi_g(x)=a/x$. In the latter case, $\phi_g$ has a 
fixed point, namely $\sqrt a$. But translation has no fixed points, so 
$\phi_g(x)=a x$. Since $g=\phi_g(1)=a$, it follows that $G=\Gm$. 
\end{proof}

The group $\Gm$ is reductive (better, a torus), and $\Ga$ is unipotent. For a 
general (i.e., not necessarily linnear) connected one-dimensional algebraic 
group, there are many possibilities, namely elliptic curves. Even over $\dC$, 
the collection of elliptic curves is one-dimensional when interpreted as a 
variety in the appropriate sense. 

\autoref{thm:1d-class} is a very useful result. It will arise many times in 
the proof of various important theorems. 





\subsection{Radical of an algebraic group}

Let $G_{/k}$ be an algebraic group, and let $\fg=\lie(G)$. It turns out that 
$\lie(G^\circ)=\fg$. There is a canonical sub-Lie algebra 
$\lierad(\fg)\subset \fg$; in characteristic zero, this will determine a 
connected linear algebraic subgroup $\rad G\subset G$. 

Let $\derived\fg=[\fg,\fg]$ be the subspace 
of $\fg$ generated by $\{[x,y]:x,y\in \fg\}$.
The subspace $\derived\fg$ is actually an ideal, and the quotient 
$\fg/\derived\fg$ is commutative. Moreover, $\derived\fg$ is the smallest 
ideal in $\fg$ with this property. 

Every Lie algebra comes with a canonical descending filtration 
$\derived^\bullet\fg$, called the \emph{derived series}. It is defined by 
\begin{align*}
  \derived^1\fg &= \derived \fg \\
  \derived^{n+1}\fg &=\derived(\derived^n\fg) .
\end{align*}

\begin{definition}
A Lie algebra $\fg$ is \emph{solvable} if the filtration $\derived^\bullet\fg$ 
is separated, that is if $\derived^n\fg=0$ for some $n$. 
\end{definition}

\begin{lemma}
A Lie algebra $\fg$ is solvable if and only if there exists a decreasing 
filtration $\fg=\fg_0\supset \cdots \supset \fg_n=0$ with each $\fg_{i+1}$ an 
ideal in $\fg_i$, and with $\fg_i/\fg_{i+1}$ commutative. 
\end{lemma}
\begin{proof}
$\Rightarrow$. This follows from the fact that 
$\derived^i\fg / \derived^{i+1}\fg$ is abelian for each $i$. 

$\Leftarrow$. Since $\fg_0\fg_1$ is commutative, we get 
$\derived\fg\supset\fg_1$. More generally, we get $\derived^i\fg\supset \fg_i$ 
by induction, so $\fg_i=0$ for $i\gg 0$ implies $\derived^i\fg=0$ for $i\gg 0$. 
\end{proof}

\begin{lemma}
Let $\fg$ be a finite-dimensional Lie algebra. Then $\fg$ has a unique maximal 
solvable ideal; it is called the \emph{radical} of $\fg$, and denoted 
$\lierad(\fg)$. 
\end{lemma}
\begin{proof}
Let $\fa$ be an ideal of $\fg$ that is solvable, and has $\dim(\fa)$ maximal. 
Let $\fb$ be any solvable ideal. Then $\fa+\fb$ is an ideal and 
$(\fa+\fb)/\fa$ is solvable. From the general fact that solvable Lie algebras 
are closed under extensions, we get that $\fa+\fb$ is solvable, so 
$\fa+\fb=\fa$, whence $\fb\supset \fa$. 
\end{proof}







