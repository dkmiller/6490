% !TEX root = 6490.tex

\section{Solvable groups, unipotent groups, and tori}

Recall that if $G_{/k}$ is an algebraic group, there is a canonical filtration 
\[
  1\supset \urad G\subset \rad G\subset G^\circ \subset G .
\]
Each subgroup in the filtration is normal in $G$. The \emph{neutral 
component} $G^\circ$ of $G$, is defined as a functor on $k$-schemes by 
\[
  G^\circ(S) = \{g:S\to G:g(|S|)\subset |G|^\circ\} ,
\]
where $|G|^\circ$ is the connected component of $1$ in the topological space 
underlying $G$. By \cite[VI\textsubscript{A} 2.3.1, 2.4]{sga3-i}, the functor $G^\circ$ 
represents an open, geometrically irreducible, subgroup scheme of $G$.By 
\cite[VI\textsubscript{A} 5.5.1]{sga3-i}, the quotient $\pi_0(G)=G/G^\circ$ is 
\'etale over $k$. 

One calls $\rad G$ the \emph{radical} of $G$, an $\urad G$ the \emph{unipotent 
radical}. All possible quotients in the filtration have names: 
\begin{itemize}
  \item $G^\circ/G$ is \emph{finite}
  \item $G^\circ/\rad G$ is \emph{semisimple}
  \item $G^\circ/\urad G$ is \emph{reductive}
  \item $\rad G/\urad G$ is a \emph{torus}
  \item $\urad G$ is \emph{unipotent} .
\end{itemize}





\subsection{One-dimensional groups}

For simplicity, assume $k$ is algebraically closed. Let $G_{/k}$ be a 
one-dimensional smooth connected linear algebraic group. Currently, we have 
two candidates for $G$, the additive group $\Ga$ and the multiplicative 
group $\Gm$, given by 
\begin{align*}
  \Ga(A) &= (A,+) \\
  \Gm(A) &= A^\times 
\end{align*}
for all $k$-algebras $A$. 

\begin{theorem}\label{thm:1d-class}
Any one-dimensional connected smooth linear algebraic group over an 
algebraically closed field is isomorphic to a unique member of 
$\{\Ga,\Gm\}$. 
\end{theorem}
\begin{proof}
As a variety over $k$, $G$ is smooth and one-dimensional. Thus there is a 
unique smooth proper curve $C_{/k}$ with 
an open embedding $G\hookrightarrow C$. The set $S=C(k)\smallsetminus G(k)$ is 
finite. Take $g\in G(k)$. Then $\phi_g:G\iso G$ 
given by $x\mapsto g\cdot x$ is an automorphism of curves. We can think of 
$\phi_g$ as a rational map $C\to C$; by Zariski's main theorem, this extends 
uniquely to an automorphism $\phi_g:C\iso C$. 
We obtain a group homomorphism $\phi:G(k)\monic \automorphisms(C)$. 

Let 
$g$ be the genus of $C$ (if $k=\dC$, this is just the number of ``holes'' in 
the closed surface $C(\dC)$). By \cite[IV ex 5.2]{hartshorne-1977}, if $g\geqslant 2$, then 
$\automorphisms(C)$ is finite. It follows that $g\leqslant 1$. Since $S$ is 
finite, there is an infinite subgroup $H\subset G(k)$ that acts trivially on 
$S$. Write $\automorphisms(C,S)$ for the group of automorphisms of $C$ that 
are trivial on $S$; there is an injection $H\monic \automorphisms(C,S)$. If 
$g=1$, then $\automorphisms(C,S)$ is finite \cite[IV cor 4.7]{hartshorne-1977}. 

We've reduced to the case $g=0$. We can assume 
$G\subset C=\dP^1_{/k} = \dA^1_{/k}\cup \{\infty\}$. It is known 
that $\automorphisms(\dP^1)=\PGL(2)$ as schemes, so in particular 
$\automorphisms(\dP_{/k}^1) = \PGL_2(k)$ \cite[IV 7.1.1]{hartshorne-1977}. 
The action of $\PGL_2(k)$ is via 
fractional linear transformations: 
\[
  \begin{pmatrix} a & b \\ c & d \end{pmatrix} x = \frac{a x+b}{c x+d} .
\]
For any distinct $\alpha,\beta,\gamma\in \dP^1(k)$, there is a unique 
$g\in \PGL_2(k)$ such that $g(\alpha)=0$, $g(\beta)=1$, and $g(\gamma)=\infty$ 
[this is equivalent to $M_{0,3}=\ast$, it can be verified by a direct 
computation]. If $\# S\geqslant 3$, then this shows that 
$\automorphisms(\dP^1_{/k},S)=1$, which doesn't work. We now get two cases, 
$\# S\in \{1,2\}$. So without loss of generality, 
$G=\dP^1_{/k}\smallsetminus \{\infty\}$ or 
$G=\dP^1_{/k}\smallsetminus \{0,\infty\}$. So as a variety, $G=\Ga$ or $\Gm$. 

We'll treat the case $G=\dP^1_{/k}\smallsetminus \{0,\infty\}$. We can assume 
$1\in \dP^1$ is the identity of $G$. Pick $g\in G(k)\subset k^\times$; then 
$\phi_g:x\mapsto g x$ must be of the form $x\mapsto \frac{a x+b}{c x+d}$. 
Moreover, $\phi_g\{0,\infty\}=\{0,\infty\}$. Either $\phi_g(x)=a x$ 
for $a\in k^\times$, or $\phi_g(x)=a/x$. In the latter case, $\phi_g$ has a 
fixed point, namely $\sqrt a$. But translation has no fixed points, so 
$\phi_g(x)=a x$. Since $g=\phi_g(1)=a$, it follows that $G=\Gm$. 
\end{proof}

The group $\Gm$ is reductive (better, a torus), and $\Ga$ is unipotent. For a 
general (i.e., not necessarily linnear) connected one-dimensional algebraic 
group, there are many possibilities, namely elliptic curves. Even over $\dC$, 
the collection of elliptic curves is one-dimensional when interpreted as a 
variety in the appropriate sense. 

\autoref{thm:1d-class} is a very useful result. It will arise many times in 
the proof of various important theorems. 





\subsection{Solvable groups}

Let $G_{/k}$ be an algebraic group, and let $\fg=\lie(G)$. It turns out that 
$\lie(G^\circ)=\fg$. There is a canonical sub-Lie algebra 
$\lierad(\fg)\subset \fg$; in characteristic zero, this will determine a 
connected linear algebraic subgroup $\rad G\subset G$. 

For the moment, let $\fg$ be an arbitrary $k$-Lie algebra. 
Let $\derived\fg=[\fg,\fg]$ be the subspace 
of $\fg$ generated by $\{[x,y]:x,y\in \fg\}$.
The subspace $\derived\fg$ is actually an ideal, and the quotient 
$\fg/\derived\fg$ is commutative. Moreover, $\derived\fg$ is the smallest 
ideal in $\fg$ with this property. 

Every Lie algebra comes with a canonical descending filtration 
$\derived^\bullet\fg$, called the \emph{derived series}. It is defined by 
\begin{align*}
  \derived^1\fg &= \derived \fg \\
  \derived^{n+1}\fg &=\derived(\derived^n\fg) .
\end{align*}

\begin{definition}
A Lie algebra $\fg$ is \emph{solvable} if the filtration $\derived^\bullet\fg$ 
is separated, that is if $\derived^n\fg=0$ for some $n$. 
\end{definition}

\begin{lemma}
A Lie algebra $\fg$ is solvable if and only if there exists a decreasing 
filtration $\fg=\fg_0\supset \cdots \supset \fg_n=0$ with each $\fg_{i+1}$ an 
ideal in $\fg_i$, and with $\fg_i/\fg_{i+1}$ commutative. 
\end{lemma}
\begin{proof}
$\Rightarrow$. This follows from the fact that 
$\derived^i\fg / \derived^{i+1}\fg$ is abelian for each $i$. 

$\Leftarrow$. Since $\fg_0/\fg_1$ is commutative, we get 
$\derived\fg\supset\fg_1$. More generally, we get $\derived^i\fg\supset \fg_i$ 
by induction, so $\fg_i=0$ for $i\gg 0$ implies $\derived^i\fg=0$ for $i\gg 0$. 
\end{proof}

\begin{lemma}
Let $\fg$ be a finite-dimensional Lie algebra. Then $\fg$ has a unique maximal 
solvable ideal; it is called the \emph{radical} of $\fg$, and denoted 
$\lierad(\fg)$. 
\end{lemma}
\begin{proof}
Let $\fa$ be an ideal of $\fg$ that is solvable, and has $\dim(\fa)$ maximal. 
Let $\fb$ be any solvable ideal. Then $\fa+\fb$ is an ideal and 
$(\fa+\fb)/\fa$ is solvable. From the general fact that solvable Lie algebras 
are closed under extensions, we get that $\fa+\fb$ is solvable, so 
$\fa+\fb=\fa$, whence $\fb\subset \fa$. 
\end{proof}

\begin{definition}
A Lie algebra $k$ is \emph{semisimple} if $\lierad(\fg)=0$. 
\end{definition}

\begin{example}
Consider the Lie algebra 
\[
  \fb_n = \{x\in \gl_n : x_{i,j}=0\text{ for all }i>j\} .
\]
We claim that $\fb_n$ is solvable. This follows from the fact that 
\[
  \derived^r \fb_n = \{x\in \gl_n : x_{i,j=0}\text{ for all }i>j-r\} .
\]
To see that this is true, note that it is trivially true for $r=0$, so 
assume it is true for some $r$, and let $x,y\in \derived^r \fb_n$. Note that 
\begin{align*}
  [x,y]_{i,j} 
    &= \sum_k (x_{i,k}y_{k,j}-y_{i,k}x_{k,j}) \\ \tag{$\ast$}\label{eq:bn}
    &= \sum_{i-r\leqslant k \leqslant j+r} (x_{i,k}y_{k,j}-y_{i,k}x_{k,j})
\end{align*}
Moreover, when $i>j+r+1$, then all of the terms in \eqref{eq:bn} are 
zero, whence the result. In some sense, $\fb_n$ is the 
``only'' example of a solvable Lie algebra over an algebraically closed 
field. 
\end{example}

\begin{theorem}[Lie-Kolchin]
Let $k$ be an algebraically closed field, $\fg$ a finite-dimensional 
solvable $k$-Lie algebra. Then there is an injective Lie homomorphism 
$\fg\monic \fb_n$ for some $n$. 
\end{theorem}
\begin{proof}
In characteristic zero, this follows directly from Corollary 2 of 
\cite[I \S 5.3]{bourbaki-lie-alg-1-3} applied to the adjoint representation. 
\end{proof}

\begin{example}
One can check that: 
\begin{align*}
  \lierad(\gl_2) &= \left\langle \begin{pmatrix} 1 \\ & 1 \end{pmatrix}\right\rangle \\
  \derived^n(\gl_2) &= \Sl_2 && \text{for all }n\geqslant 1 
\end{align*}
This is because $\Sl_2$ is simple (has no non-trivial ideals). 
\end{example}

\begin{definition}
Let $G_{/k}$ be a linear algebraic group. We say $G$ is \emph{solvable} if 
there is a sequence of algebraic subgroups 
$G=G_0\supset G_1\supset \cdots \supset G_n=1$ such that 
\begin{enumerate}
  \item each $G_{i+1}$ is normal in $G_i$, 
  \item each $G_i/G_{i+1}$ is commutative. 
\end{enumerate}
\end{definition}

\begin{example}
Let $G=B(n)\subset \GL(n)$ be the subgroup of upper-triangular matrices (the 
letter $B$ represents ``Borel''). This is solvable, as is witnessed by the 
filtration 
\begin{align*}
  G_1 &= \{g\in \GL(n):g_{i i}=1\text{ for all }i\} \\
  G_r &= \{g\in B(n)_2:g_{i j}=0\text{ for all }i<j+r\} && r\geqslant 1 .
\end{align*}
The map $G_0\to \Gm^n$ defined by $(a_{i j})\mapsto (a_{11},\dots,a_{nn})$ 
induces an isomorphism $G_0/G_1\iso \Gm^n$. Similarly 
$G_1\to \Ga^{n-1}$ defined by $(a_{i j})\mapsto (a_{1,2},\dots,a_{n-1,n})$ 
induces an isomorphism $G_1/G_2\iso \Ga^{n-1}$. In general, for $i\geqslant 0$, 
we have $G_{i}/G_{i+1} \iso \Ga^{n-i}$. 

We could have chosen our filtration in such a way that 
$G_i/G_{i+1}\in \{\Ga,\Gm\}$. 
\end{example}

\begin{definition}
Let $G_{/k}$ be a linear algebraic group. The \emph{derived} group 
$\derived G$ (or $G'$, or $G^\mathrm{der}$) is the smallest normal subgroup of 
$G$ such that $G/\derived G$ is commutative. 
\end{definition}

The basic idea is that $\derived G$ is the algebraic group generated by 
$\{x y x^{-1} y^{-1}:x,y\in G\}$. 

\begin{theorem}\label{thm:derived-nice}
Let $G_{/k}$ be a smooth linear algebraic group. Then $\derived G$ exists and 
is smooth, and if $G$ is connected then so is $\derived G$. 
\end{theorem}
\begin{proof}
This essentially follows from \cite[VI\textsubscript{B} 7.1]{sga3-i}. 
\end{proof}

Just as for Lie algebras, we can define a filtration 
\begin{align*}
  \derived^1 G &= \derived G \\
  \derived^{n+1} G &= \derived(\derived^n G) .
\end{align*}

\begin{theorem}
A linear algebraic group $G$ is solvable if and only if $\derived^n G=1$ for 
some $n$. 
\end{theorem}

\begin{example}
If $G\subset B(n)\subset \GL(n)$, then $G$ is solvable. Indeed, this 
follows from $\derived^\bullet G\subset \derived^\bullet B(n)$. 
\end{example}

\begin{theorem}[Lie-Kolchin]\label{thm:lie-kolchin}
Suppose $k$ is algebraically closed. Let $G\subset \GL(n)_{/k}$ be a 
connected solvable algebraic group. Then there exists $x\in \GL_n(k)$ such that 
$x G x^{-1}\subset B(n)$. 
\end{theorem}
\begin{proof}
Let $V=k^{\oplus n}$, and consider $V$ as a representation of $G$ via the 
inclusion $G\hookrightarrow \GL(V)$. It is sufficient to prove that $V$ 
contains a one-dimensional subrepresentation, for then we could induct on 
$\dim(V)$. For simplicity, we assume $G$ is smooth. If $G$ is commutative, 
then the set $G(k)$ is a family of mutually commuting endomorphisms of $V$. 
It is known that such sets are mutually triangularisable, i.e.~can be 
conjugated to lie within $B(n)$. (This follows from the Jordan decomposition.) 

In the general case, we may assume the claim is true for $\derived G$. 
Recall that $\characters^\ast(\derived G)=\hom(\derived G,\Gm)$ is the group of 
\emph{characters} of $\derived G$. The group $G$ acts on 
$\characters^\ast(\derived G)$ by conjugation: 
\[
  (g\cdot \chi)(h) = \chi(g h g^{-1}) .
\]
Even better, we can define $\characters^\ast(\derived G)$ as a group functor: 
\[
  \characters^\ast(\derived G)(A)=\hom_{\groups_{/A}}((\derived G)_{/A},(\Gm)_{/A}) .
\]
By \cite[11.4.2]{sga3-ii}, this is represented by a smooth separated scheme 
over $k$. We define a subscheme 
\[
  \Delta(A) = \{\chi\in \characters^\ast(\derived G)(A):\chi\text{ factors through }G_{/A}\hookrightarrow \GL(V)_{/A}\} .
\]
Note that $\Delta$ is finite and nonempty, and $G$ is connected. Thus the 
induced action of $G$ on $\Delta$ is trivial, so $G$ fixes some 
$\chi\in \Delta(k)$. Let $V_\chi$ be the suprepresentation 
of $V$ generated by all $\chi$-typical vectors, i.e. 
\[
  V_\chi(A) = \langle v\in V_{/A}:g\cdot v=\chi(g) v\text{ for all }g\in G(A)\} .
\]
After replacing $V$ by $V_\chi$, we may assume $\derived G$ acts on $V$ by a 
character $\chi$. That is, as a subgroup of $\GL(n)$, $\derived G\subset \Gm$. 
Since the determinant map kills commutators, 
$\derived G\subset \Gm\cap \SL(n)=\dmu_n$, so $\derived G$ is finite. Since $G$ 
connected, \autoref{thm:derived-nice} tells us that $\derived G=1$, so $G$ is 
abelian, and we're done. 
\end{proof}

\begin{example}
If $k$ is algebraically closed and $G_{/k}\subset \GL(n)_{/k}$ is solvable, 
then there exists a filtration $G=G_0\supset \cdots \supset G_n=1$ such that 
$G_0/G_1\simeq \Gm^r$, and all higher $G_i/G_{i+1}\simeq \Ga$. We will see 
that this property determines $G_1$. We know there is $v\in k^{\oplus n}$ which 
is a common eigenvector of all $g\in \derived G(k)$. This gives us a 
character $\chi:\derived G\to \Gm$. 
\end{example}

\begin{example}
This shows that \autoref{thm:lie-kolchin} does not hold over non-algebraically closed 
fields. Let 
\[
  G_{/\dR} = \left\{\begin{pmatrix} a & -b \\ b & a \end{pmatrix}\right\}\subset \GL(2)_{/\dR} .
\]
Note that $G(\dR)\simeq \dC^\times$ (in fact, $G=\weil_{\dC/\dR} \Gm$) via 
\[
  \begin{pmatrix} a & -b \\ b & a \end{pmatrix} \leftrightarrow a+ b i .
\]
The group $G$ is commutative (hence solvable), but it is not conjugate to 
$B(2)$ by an element of $\GL_2(\dR)$. Indeed, note that 
$\begin{pmatrix} & -1 \\ 1 \end{pmatrix}$ is not diagonalizable in $\GL_2(\dR)$ 
because it has eigenvalues $\pm i$. 
\end{example}

For a general group, we'll have a subgroup $\rad G$, the \emph{radical} of 
$G$. It will be the largest connected normal solvable subgroup of $G$. It 
will turn out that $\lierad(\fg) = \lie(\rad G)$. 





\subsection{Quotients}

Let $G_{/k}$ be an algebraic group, $N\subset G$ a sub-algebraic group. As 
functors of points, $N(A)$ is a normal subgroup of $G(A)$ for all $k$-algebras 
$A$. 

\begin{example}
Let $k=\dR$, and consider $\dmu_2=\ker(\Gm\xrightarrow{(-)^2}\Gm)$ over $k$. 
Should we consider the sequence 
\[
  1\to \dmu_2\to \Gm\xrightarrow 2 \Gm \to 1 
\]
to be exact? On $\dC$-points, this is the sequence 
\[
  1\to \{\pm 1\} \to \dC^\times \xrightarrow 2 \dC^\times \to 1
\]
which is certainly exact. So we will write $\Gm=\Gm/\dmu_2$. Note however 
that the sequence for $\dR$-points is 
$1\to \{\pm 1\}\to \dR^\times \xrightarrow 2 \dR^\times$, which is \emph{not} 
exact on the right. 
\end{example}

In general, if $1\to N\to G\to H\to 1$ is an ``exact sequence'' of algebraic 
groups, we will not necessarily have surjections from the $A$-points of $G$ to 
the $A$-points of $H$. 

\begin{definition}
Let $N_{/k}\subset G_{/k}$ be a sub-algebraic group. A \emph{quotient} of $G$ 
by $N$ is a homomorphism $G\xrightarrow q Q$ of algebraic groups over $k$ with kernel $N$, 
such that if $G\xrightarrow\phi G'$ vanishes on $N$, then there is a unique 
$\psi:Q\to G'$ such that the following diagram commutes:
\[\begin{tikzcd}
  1 \ar[r] 
    & N \ar[r] 
    & G \ar[r, "q"] \ar[dr, "\phi"]
    & Q \ar[d, "\psi", dashrightarrow] \\
  & & & G'
\end{tikzcd}\]
\end{definition}

\begin{theorem}
Quotients exist in the category of (possibly non-smooth) affine group schemes 
of finite type over $k$. 
\end{theorem}
\begin{proof}
This is \cite[10.16]{milne-iAG}. 
\end{proof}

If $G/H$ is the quotient of $G$ by a subgroup $H$, it is generally true that 
$(G/H)(\bar k)=G(\bar k)/H(\bar k)$. 

\begin{hard}
There is a more abstract, but powerful approach to defining quotients. 
Let $\schemes k$ be the category of schemes over $\spectrum(k)$. We can regard 
any scheme over $k$ as an fppf sheaf on $\schemes k$ via the Yoneda embedding. 
If $H\subset G$ is a closed subgroup-scheme, we write $G/H$ for the quotient 
sheaf of $S\mapsto G(S)/H(S)$ in the fppf topology. By 
\cite[VI\textsubscript{A} 3.2]{sga3-i}, this quotient sheaf is representable. 
By general nonsense, it will satisfy more elementary definition of quotient. 
\end{hard}





\subsection{Unipotent groups}

Recall that for the moment, $B(n)\subset \GL(n)$ is the subgroup of upper 
triangular matrices and $U(n)\subset B(n)$ is the subgroup of \emph{strictly} 
upper-triangular matrices. 

Recall that in our filtration of $B(n)\subset \GL(n)$, we had 
$B(n)/U(n)\simeq \Gm^n$, and all further quotients were $\Ga^r$ for varying 
$r$. We'd like to generalize this to an arbitrary connected solvable groups 
over algebraically closed fields. For the moment though, $k$ need not be 
algebraically closed. 

\begin{definition}
Let $G_{/k}$ be an algebraic group. We say $G$ is \emph{unipotent} if it admits 
a filtration $G=G_0\supset \cdots \supset G_n=1$ of closed subgroups defined 
over $k$ such that 
\begin{enumerate}
  \item each $G_{i+1}$ is normal in $G_i$, 
  \item each $G_i/G_{i+1}$ is isomorphic to a closed subgroup of $\Ga$. 
\end{enumerate}
\end{definition}

Clearly, unipotent groups are solvable. If $k$ has characteristic zero, we can 
assume $G_i/G_{i+1}\simeq \Ga$. If $k$ has characteristic $p>0$, then we have 
to worry about things like $\dalpha_p=\ker(\Ga\xrightarrow p \Ga)$. 
Fortunately, by \cite[XVII 1.5]{sga3-ii}, the only possible closed subgroups of 
$\Ga$ over a field of characteristic $p$ are $0$, $\Ga$ and extensions of 
$(\dZ/p)^r$ by $\dalpha_{p^e}$. 

\begin{theorem}
Let $G_{/k}$ be a connected linear algebraic group. Then $G$ is unipotent 
if and only if it is isomorphic to a closed subgroup of some $U(n)_{/k}$. 
\end{theorem}
\begin{proof}
This follows directly from \cite[XVII 3.5]{sga3-ii}. 
\end{proof}

If $G_{/k}$ is unipotent and $\rho:G\to \GL(V)$ is a representation, then for 
all $g\in G(\bar k)$, the matrix $\rho(g)$ is unipotent, i.e.~$(\rho(g)-1)^n=0$ 
for some $n\geqslant 1$. Indeed, by \cite[XVII 3.4]{sga3-ii}, for any such 
representation, there is a 
$G$-stable filtration $\filtration^\bullet V$ for which the action of $G$ 
on $\graded^\bullet(V)$ is trivial. In other words, $\rho$ is conjugate to 
a representation which factors through some $U(n)$, and elements of 
$U(n)(k)$ are all unipotent. Conversely, by \cite[XVII 3.8]{sga3-ii}, if 
$k$ is algebraically closed and every element of $G(k)\subset \GL_n(k)$ is 
unipotent, then $G$ is unipotent when considered as an algebraic group. 

\begin{theorem}
Let $G_{/k}$ be a connected solvable smooth group over a perfect field $k$. 
Then there exists a unique connected normal $G_\unipotent\subset G$ such that 
\begin{enumerate}
  \item $G_\unipotent$ is unipotent, 
  \item $G/G_\unipotent$ is of multiplicative type .
\end{enumerate}
\end{theorem}
\begin{proof}
This is \cite[XVII 17.23]{milne-iAG}. 
\end{proof}

Recall that an algebraic group $G_{/k}$ is \emph{of multiplicative type} 
if it is locally (in the fpqc topology) the form 
$A\mapsto \hom(M,A^\times)$, for $M$ an (abstract) abelian group 
\cite[IX 1.1]{sga3-ii}. Over a perfect field, any group of multiplicative type 
is locally of this form after an \'etale base change. 
In particular, if $k$ has characteristic zero, $G/G_\unipotent$ will be a 
torus. If moreover $k=\bar k$, then $G/G_\unipotent\simeq \Gm^r$. In general, 
$G/G_\unipotent$ will be a closed subgroup of a torus. 

\begin{example}
If $G=B(n)\subset \GL(n)$, then $G_\unipotent=U(n)$, the subgroup of strictly 
upper-triangular matrices. 
\end{example}





\subsection{Review of canonical filtration}

For this section, assume $k$ is a perfect field. 

Let $G_{/k}$ be a smooth connected linear algebraic group. In [cite], we defined 
$\rad G$, the radical of $G$, to be the largest normal subgroup of $G$ that is 
connected and solvable. 

The \emph{unipotent radical} $\urad G$ of $G$, is by definition 
$(\rad G)_\unipotent$. As in \autoref{sec:str-thry-lag}, we have a diagram 
\begin{center}
\begin{tikzpicture}
  \matrix (m) [
    matrix of nodes,
    nodes={anchor=west}
  ] {
    & $G$ \\
    & $\cup$ & finite\\
    connected & $G^\circ$ \\
    & $\cup$ & semisimple \\
    solvable & $\rad G$ \\
    & $\cup$ & torus \\
    unipotent & $\urad G$ \\
    & $\cup$ \\
    & $1$ \\
  };
  
  \draw [decoration={brace,amplitude=0.5em},decorate]
        (m-3-2.north -| m.east) -- (m-7-2.south -| m.east) 
        node [midway,xshift=1cm] {reductive};
\end{tikzpicture}
\end{center}

So the group $\urad G$ is unipotent, $\rad G$ is solvable. The quotient 
$\rad G/\urad G$ is a torus (we could define a torus to be a solvable group 
with trivial unipotent radical.) 

The quotient $G/\rad G$ is \emph{semisimple}, and $G/\urad G$ is 
\emph{reductive}. There is a good structure theory for semisimple groups. 

\begin{definition}
Let $k$ be a perfect field. A connected linear algebraic group $G_{/k}$ 
is \emph{reductive} if $\urad G=1$. 
\end{definition}

Over a non-perfect field $k$, we say $G_{/k}$ is reductive if 
$\urad(G_{\bar k})=1$. If $k$ is perfect, then $\urad(G_{\bar k})$ descends 
uniquely to a group $\urad G$ defined over $k$, but this is not true in 
general. Over a non-perfect field, one calls $G$ \emph{pseudo-reductive} if we 
only have $\urad G=1$. The main example of a pseudo-reductive group which is 
not reductive is any group of the form $\weil_{K/k} G$, where $K/k$ is purely 
inseparable and $G_{/K}$ is reductive. There is a reasonably satisfying 
structure theory for pseudo-reductive groups, worked out in 
\cite{conrad-gabber-prasad-2010}. 

\begin{example}
The group $\GL(n)$ is reductive, even though $\rad \GL(n)=\Gm$ is nontrivial. 
\end{example}

\begin{definition}
A connected linear algebraic group $G$ is \emph{semisimple} if 
$\rad G=1$. 
\end{definition}

\begin{example}
The group $\SL(n)$ is semisimple. The subgroup $\dmu_n= \zentrum(\SL_n)$ is 
a solvable normal subgroup, but it's not connected (or not smooth, if the 
base characteristic divides $n$). 
\end{example}





\subsection{Jordan decomposition}

To begin with, let $k$ be an algebraically closed field, $V$ a 
finite-dimensional $k$-vector space. Let $x\in \gl(V)$. Then we can consider 
the action of a polynomial ring $k[x]$ on $V$ via $x$. By the 
fundamental theorem of finitely generated modules over a principal ideal 
domain \cite[VII \S2.2 thm.1]{bourbaki-algebra-4-7}, we have 
$V=\bigoplus_\lambda V_\lambda$, where for $\lambda\in k$, we define 
\[
  V_\lambda = \{v\in V:(x-\lambda)^n v=0\text{ for some }n\geqslant 1\} .
\]
The $\lambda$ for which $V_\lambda\ne 0$ are called \emph{eigenvalues} of 
$x$, and $\dim(V_\lambda)$ is the \emph{multiplicity} of $\lambda$. By a 
further application of the fundamental theorem for modules over a PID, we 
get that each $V_\lambda\simeq \bigoplus_i k[x]/(x-\lambda)^{n_i}$. 

The $k$-vector space $k[x]/(t-\lambda)^n$ has basis 
$\{v_i=(t-\lambda)^{n-i}\}_{i=1}^n$. Moreover, 
\begin{align*}
  t v_1 &= \lambda v_1 \\
  t v_2 &= \lambda v_2 + v_2 \
  \cdots 
\end{align*}
When we write $x$ with respect to this basis, we get the $n\times n$ matrix 
$J_n(\lambda)$ which is $\lambda$ along the diagonal, $1$ just above the 
diagonal, and zeros everywhere else. 

For a general endomorphism $x\in \gl(V)$, we end up with a direct sum 
decomposition (with respect to some basis) $g=\bigoplus J_{n_i}(\lambda_i)$. 


In all of this, we used $k=\bar k$. Over a general field $k$, we'll be able to 
write a matrix $x$ as the sum of a diagonal matrix and a strictly upper 
triangular (hence nilpotent) matrix. Suppose we have written 
$x=x_\semisimple+x_\nilpotent$, where $x_\semisimple$ is diagonal and 
$x_\nilpotent$ is nilpotent. It turns out that $x_\semisimple$ is uniquely 
determined, and is a polynomial in $x$. That is, there exists a polynomial 
$f\in k[x]$, possibly depending on $x$, such that $x_\semisimple=f(x)$, and 
similarly for $x_\nilpotent$. 

For the remainder of this section, let $k$ be an arbitrary perfect field, $V$ 
a finite-dimensional $k$-vector space. 

\begin{definition}
An element $x\in \gl(V)$ is \emph{semisimple} if it is diagonalizable after 
base-change to an extension of $k$.  
\end{definition}

\begin{definition}
An element $x\in \gl(V)$ is \emph{nilpotent} if $x^n=0$ for some $n\geqslant 1$. 
\end{definition}

\begin{definition}
An element $x\in \GL(V)$ is \emph{unipotent} if $x-1$ is nilpotent. 
\end{definition}

\begin{theorem}[Additive Jordan decomposition]
For any $x\in \gl(V)$, there exists unique elements $x_\semisimple\in \GL(V)$, 
$x_\nilpotent\in \gl(V)$, such that 
\begin{enumerate}
  \item $x=x_\semisimple+g_\nilpotent$, 
  \item $x_\semisimple$ is semisimple, 
  \item $x_\nilpotent$ is nilpotent, and 
  \item $[x_\semisimple,x_\nilpotent]=0$. 
\end{enumerate}
Moreover, there exist polynomials $f,g\in k[x]$ such that 
$x_\semisimple=f(x)$ and $x_\nilpotent=g(x)$. 
\end{theorem}
\begin{proof}
Case 1: the eigenvalues of $x$ lie in $k$. Then by the theory of Jordan 
normal form, we get existence of a decomposition. 
Suppose $x=x_\semisimple+x_\nilpotent = y_\semisimple+y_\nilpotent$ are two 
distinct decompositions. Then 
$y_\semisimple-y_\semisimple=-y_\nilpotent+h_\nilpotent$, and 
$x_\semisimple$, $x_\nilpotent$ commute with $t_\semisimple$, $t_\nilpotent$. 
This is because $x_\semisimple$ and $x_\nilpotent$ are polynomials in $x$. It 
follows that $-x_\nilpotent+y_\nilpotent$ is nilpotent. The matrices 
$x_\semisimple$, $y_\semisimple$ commute, so they are simultaneously 
diagonalizable. So $x_\semisimple-y_\semisimple$ is still semisimple. But 
$x_\semisimple-y_\semisimple$ is also nilpotent, so 
$x_\semisimple=y_\semisimple$. 

Case 2: the eigenvalues of $x$ may not lie in $k$. Take a Galois extension 
$K/k$ containing all the eigenvalues of $x$. (For example, we could let $K$ be 
the extension of $k$ generated by the eigenvalues of $x$.) We can write 
$x=x_\semisimple+x_\nilpotent$, where 
$x_\semisimple,x_\nilpotent\in \gl(V)\otimes K$. For any 
$\sigma\in \galois(K/k)$, we have $\sigma(x)=x$, because $x\in \gl(V)$. But 
then $x=\sigma(x_\semisimple)+\sigma(x_\nilpotent)$, and this is another 
additive Jordan decomposition of $x$. By uniqueness in the first case, we 
see that $x_\semisimple$ and $x_\nilpotent$ are fixed by $\sigma$. Since 
$\sigma$ was arbitrary, $x_\semisimple$ and $x_\nilpotent$ lie in 
$\gl(V)$. 
\end{proof}

For a more careful proof, see \cite[VII \S 5.8 thm.1]{bourbaki-algebra-4-7}. 


\begin{example}
This shows that the perfectness hypothesis on $k$ is necessary. Let 
$k=\dF_2(t)$. Consider the matrix $x=\begin{pmatrix} & 1 \\ t \end{pmatrix}$. 
This has characteristic polynomial $x^2-t$, so its only eigenvalue is 
$\sqrt 2$ appearing with multiplicity two. So the only way we could 
give $x$ a Jordan decomposition is 
$\begin{pmatrix} \sqrt t \\ & \sqrt t\end{pmatrix} + \begin{pmatrix} -\sqrt t & 1 \\ t & -\sqrt t\end{pmatrix}$. 
The problem is, this doesn't work after a separable base change. More 
concretely, this doesn't descend back to $k$. 
\end{example}

\begin{theorem}[Multiplicative Jordan decomposition]
Let $k$ be a perfect field, $V$ a finite-dimensional $k$-vector space. For any 
$g\in \GL(V)$, there exists unique $g_\semisimple, g_\unipotent\in \GL(V)$ such 
that 
\begin{enumerate}
  \item $g=g_\semisimple g_\unipotent$, 
  \item $g_\semisimple$ is semisimple, 
  \item $g_\unipotent$ is unipotent, and 
  \item $g_\semisimple$ and $g_\unipotent$ commute. 
\end{enumerate}
\end{theorem}
\begin{proof}
Recall that we can write $g=g_\semisimple+g_\nilpotent$. Just rewrite it as 
$g_\semisimple(1+g_\semisimple^{-1} g_\nilpotent)$, and note that 
$g_\semisimple^{-1} g_\nilpotent$ is nilpotent because 
$g_\nilpotent$ is nilpotent and $g_\semisimple$ commutes with $g_\nilpotent$. 
So $g_\unipotent = 1+g_\semisimple^{-1} g_\nilpotent$. Uniqueness is similarly 
easy. 
\end{proof}

We'd like to connect the theory of Jordan decomposition with algebraic groups. 
Let $G_{/k}$ be a linear algebraic group, $\rho:G\to \GL(n)$ a representation. 
For $g\in G(k)$, we can write $\rho(g)=h_\semisimple h_\unipotent$. It is a 
beautiful fact that in fact $h_\semisimple=\rho(g_\semisimple)$, 
$h_\unipotent=\rho(g_\unipotent)$ for uniquely determined 
$g_\semisimple, g_\unipotent\in G(k)$. 

Similarly, let $\fg$ be a $k$-Lie algebra, $\rho:\fg\to \gl(V)$ a 
representation. For $x\in \fg$, the decomposition 
$\rho(x)=\rho(x)_\semisimple+\rho(x)_\nilpotent$ comes from a uniquely 
determined decomposition $x=x_\semisimple+x_\nilpotent$. [find assumptions, 
cite source.]

Unfortunately, not all Lie algebras have this property. For example, this fails 
for Lie algebras which don't come from algebraic groups. 

Let $f:G\to H$ be a homomorphism of linear algebraic groups defined over a 
perfect field $k$. Let $g\in G(k)$. Then 
$f(g)_\semisimple=f(g_\semisimple)$ and $f(g)_\unipotent=f(g_\unipotent)$. 
In other words, the Jordan decomposition is functorial. 
It follows that if $G_{/k}$ is an affine algebraic group (without a choice of 
embedding $G\hookrightarrow \GL(n)$), then the multiplicative Jordan 
decomposition within $G$ is well-defined, independent of any choice of 
embedding. 





\subsection{Diagonalizable groups}

Let $k$ be a field; recall that we are working in the category of fppf sheaves 
on $\schemes k$. The affine line $\dA^1(S)=\Gamma(S,\sO_S)$ is such a sheaf. 
We have already written $\Ga$ for the affine line considered as an algebraic 
group. Since $\Gamma(S,\sO_S)$ is naturally a commutative ring, we can consider 
$\dA^1$ as a \emph{ring scheme}, i.e.~a commutative ring object in the category 
of schemes. We will write $\sO$ for $\dA^1$ so considered. 

Thus if $G_{/k}$ is an affine algebraic group, the coordinate ring of $G$ is 
\[
  \sO(G)=\hom(G,\dA^1) .
\]
We will be interested in 
\[
  \characters(G) = \hom_{\groups_{/k}}(G,\Gm) \subset \sO(G) ,
\]
the set of \emph{characters} of $G$. 

\begin{lemma}
The set $\characters(G)$ is linearly independent in $\sO(G)$. 
\end{lemma}
\begin{proof}
We may assume $k$ is algebraically closed. Let $\chi_1,\dots,\chi_n$ be 
distinct characters of $G$, and suppose there is some relation 
$\sum c_i \chi_i=0$ in $\sO(G)$ with the $c_i\in k$. If $n=1$, there is 
nothing to prove. In the general case, we may assume $c_1\ne 0$. There exists 
a $k$-algebra $A$ and $h\in G(A)$ such that $\chi_1(h)\ne \chi_n(h)$. Note 
that 
\[
  \sum c_i \chi_n(h)\chi_i(g) = \sum c_i \chi_i(h) \chi_i(g) = 0 ,
\]
for all $g\in G$. This implies 
\[
  \sum_{i=1}^{n-1} c_i(\chi_n(h)-\chi_i(h)) \chi_i(g) = 0 
\]
for all $g\in G(B)$ for $A$-algebras $B$. 
Thus $\chi_1,\dots,\chi_{n-1}$ are linearly dependent in 
$\sO(G)\otimes_k A$. The only way this can happen is for 
$\chi_1,\dots,\chi_{n-1}$ to be linearly dependent in $\sO(G)$. Induction 
yields a contradiction. 
\end{proof}

The set $\characters(G)$ naturally has the structure of a group, coming from 
the group law on $\Gm$. If $\chi_1,\chi_2\in \characters(G)$, then 
\[
  (\chi_1+\chi_2)(g) = \chi_1(g)\chi_2(g) .
\]
\emph{Warning}: this notation is confusing, because the group law in $\Gm$ is 
written multiplicatively. But the group law on $\characters(G)$ will always be 
written additively. 

\begin{example}
As an exercise, check that $\characters(\Ga)=0$. 
\end{example}

\begin{example}
One has $\characters(\SL_2)=0$. More generally, $\characters(\SL_n)=0$ for all 
$n$. 
\end{example}

\begin{example}\label{eg:chars-Gm}
Let's compute $\characters(\Gm)=\hom_\groups(\Gm,\Gm)$. We claim that 
$\characters(\Gm)\simeq \dZ$ as a group, with $n\in \dZ$ corresponding to the 
character $t\mapsto t^n$. Indeed, $\sO(\Gm)=k[t^{\pm 1}]$, so we have to 
classify elements $f\in k[t^{\pm 1}]$ such that $\Delta(f)=f\otimes f$. It is 
easy to check that these are precisely the powers of $t$. Alternatively, use 
the fact that $\sO(G)$ has a basis consisting of the obvious characters; by 
linear independence of characters, these are the only ones. 
\end{example}

It is easy to check that 
$\characters(G_1\times G_2)=\characters(G_1)\oplus \characters(G_2)$. Thus 
$\characters(\Gm^n) = \dZ^n$. A tuple $a=(a_1,\dots,a_n)\in \dZ^n$ corresponds 
to the character $(t_1,\dots,t_n)\mapsto \prod t_i^{a_i}$. 

\begin{example}\label{eg:chars-mun}
There is a canonical isomorphism $\characters(\dmu_n)=\dZ/n$, even if 
$n$ is not invertible in $k$. There is an obvious inclusion 
$\dZ/n\monic \characters(\dmu_n)$ sending $a\in \dZ/n$ to the character 
$t\mapsto t^a$. This inclusion is an isomorphism. 
\end{example}

In classical texts, if the base field has characteristic $p>0$, then 
$\characters(\dmu_p)=0$ because they only treat $\bar k$-points. In fact, in 
such texts, they have ``$\dmu_p=1$.'' 

From \autoref{eg:chars-Gm} and \autoref{eg:chars-mun}, we see that all finitely 
generated abelian groups arise as $\characters(G)$ for some $G$. 

\begin{theorem}\label{thm:diag-groups}
Let $G_{/k}$ be a linear algebraic group. Then the following are equivalent: 
\begin{enumerate}
\item
$G$ is isomorphic to a closed subgroup of some $\Gm^n$. 

\item
$\characters(G)$ is a finitely generated abelian group, and forms a $k$-basis 
for $\sO(G)$. 

\item 
Any representation $\rho:G\to \GL(n)$ is a direct sum of one-dimensional 
representations. 
\end{enumerate}
\end{theorem}
\begin{proof}
$1\Rightarrow 2$. An embedding $G\monic \Gm^n$ induces a surjection 
$\sO(\Gm^n)\epic \sO(G)$. The image in $\sO(G)$ of a character 
$\chi\in \sO(\Gm^n)$ is a character, and $\sO(\Gm^n)$ has a basis consisting 
of characters. The image of a basis is a generating set, so we're done. 
Alternatively, see \cite[14.8]{milne-iAG}. 

$2\Rightarrow 3$. This is \cite[14.11]{milne-iAG}. 

$3\Rightarrow 1$. By \autoref{thm:groups-linear}, there exists an embedding 
$G\monic \GL(n)$, and by 3, we can assume the image of $G$ lies in the 
subgroup of diagonal matrices, which is isomorphic to $\Gm^n$. 
\end{proof}

\begin{definition}
Let $G$ be a linear algebraic group. If $G$ satisfies any of the 
equivalent conditions of \autoref{thm:diag-groups}, we say that $G$ is 
\emph{diagonalizable}. 
\end{definition}

It is possible to classify diagonalizable groups. Let $M$ be a finitely 
generated abelian group, whose group law is written multiplicatively. The basic 
idea is to construct an algebraic group $\diag(M)$ such that there is a 
natural isomorphism $\characters(\diag(M)) = M$. We first define $\diag(M)$ as 
a functor $\algebras k\to \groups$ by 
\[
  \diag(M)(A) = \hom_\groups(M,A^\times) .
\]
This is representable, with coordinate ring the group ring 
\[
  k[M] = \left\{\sum_{m\in M} c_m\cdot m : c_m\ne 0\text{ for only finitely many }m\right\} .
\]
In other words, $k[M]$ is the free $k$-vector space on $M$. Addition is formal, 
and multiplication is defined by $(c_1 m_1)(c_2 m_2) = c_1 c_2 (m_1 m_2)$, the 
multiplication $m_1 m_2$ taking place in $M$. It is easy to check that 
$\diag(M)(A)=\hom(k[M],A)$. 

There is a canonical isomorphism $\characters(\diag(M))=M$. Given 
$m\in \characters(\diag(M))$, we get a character $\chi_m:\diag(M)\to\Gm$ 
defined on $A$-points by 
\[
  \chi_m(\phi) = \phi(m) \qquad (\phi:M\to A^\times) .
\]
We will see that this is an isomorphism. 

The operation ``take $\diag(-)$'' is naturally a contravariant functor from the 
category of finitely generated abelian groups to the category of diagonalizable 
groups over $k$. By \cite[VIII 1.6]{sga3-ii}, it induces an (anti-) equivalence 
of categories 
\[
  \diag:\{\text{f.g.~ab.~groups}\} \rightleftarrows \{\text{diagonalizable gps.~over $k$}\}: \characters .
\]
So diagonalizable groups are exactly those of the form 
$\diag(M)$ for a finitely generated abelian group $M$. More concretely, every 
diagonalizable group will be of the form 
\[
  \Gm^n \times \dmu_{n_1} \times \cdots \times \dmu_{n_r} .
\]





\subsection{Tori}

Let $k$ be a field. Recall that we have an equivalence of categories: 
\[
  \characters:\{\text{diagonalizable groups }/k\} \rightleftarrows \{\text{f.g.~ab.~groups}\}:\diag .
\]
Here $\characters(G)=\hom(G,\Gm)$ and $\diag(M)(A)=\hom(M,A^\times)$. 

\begin{definition}
An algebraic group $G_{/k}$ has \emph{multiplicative type} if $G_{\bar k}$ is 
diagonalizable. 
\end{definition}

\begin{hard}
The general definition is in \cite[IX 1.1]{sga3-iii}. An affine group scheme 
$G_{/S}$ is said to be of multiplicative type if there is an fpqc cover 
$\widetilde S\to S$ such that $G_{\widetilde S}$ is diagonalizable. One says 
$G$ is \emph{isotrivial} if $\widetilde S\to S$ may be taken to be a finite 
\'etale cover. By \cite[X 5.16]{sga3-ii}, all groups of multiplicative type 
over a field are isotrivial. So if $G_{/k}$ is of multiplicative type, there 
exists a finite separable extension $K/k$ such that $G_K$ is diagonalizable. 
\end{hard}

\begin{definition}
An algebraic group $G_{/k}$ is a \emph{split torus} if 
$G\simeq \Gm^n$ for some $n$. The group $G$ is a \emph{torus} if 
$G_{\bar k}$ is a split torus over $k$. 
\end{definition}

\begin{hard}
As above, the general definition in \cite[IX 1.3]{sga3-ii} is that $G_{/S}$ is 
a torus if there is an fpqc cover $\widetilde S\to S$ such that 
$G_{\widetilde S}\simeq {\Gm^n}_{/\widetilde S}$. As with groups of 
multiplicative type, if $T_{/k}$ is a torus, then there is a finite 
separable extension $K/k$ such that $T_K\simeq {\Gm^n}_{/K}$. 
\end{hard}

\begin{example}[Nonsplit tori]\label{eg:nonsplit-torus}
Recall that if $K/k$ is a field extension and $G_{/K}$ is an algebraic group, 
then the \emph{Weil restriction} $\weil_{K/k} G$ is the algebraic group 
representing the functor $A\mapsto G(A\otimes_k K)$ on $k$-algebras $A$. 
Consider $G=\weil_{K/k}\Gm$. For any $k$-algebra $A$, we have (by definition) 
$G(A)=(A\otimes_k K)^\times$. Note that $G(A)$ acts on $K_A$ via 
multiplication, so there is an embedding 
$\weil_{K/k}\Gm \monic \GL(K)_{/k}$. If $K/k$ is Galois, there is a canonical 
isomorphism of $k$-algebras $K\otimes_k K\iso \prod_\Gamma K$, where 
$\Gamma=\galois(K/k)$. It sends $x\otimes y$ to the tuple 
$(x\gamma(y))_\gamma$. Now for $A\in \algebras K$, we compute 
\begin{align*}
  G(A\otimes_k K) 
    &= G(A\otimes_K (K\otimes_k K)) \\
    &= G\left(A\otimes_K \prod_\Gamma K\right) \\
    &= \prod_\Gamma G(A) .
\end{align*}
Thus $(\weil_{K/k} \Gm)_K = \prod_\Gamma{\Gm}_{/K}$, hence $\weil_{K/k}\Gm$ 
is a torus. But $\weil_{K/k}\Gm$ is not diagonalizable. If it were,
by \autoref{thm:diag-groups} the 
representation $\weil_{K/k}\Gm\monic \GL(K)_{/k}$ would factor through the 
diagonal, whence all coordinates of the image of $(\weil_{K/k}\Gm)(k)=K^\times$ 
would be in $k$, which is clearly false. 
\end{example}

\begin{example}\label{eg:hodge-structure}
A specific case of \autoref{eg:nonsplit-torus} that is of special interest 
is when the field extension is $\dC/\dR$. One writes $\dS=\weil_{\dC/\dR}\Gm$, 
and calls a representation $\rho:\dS\to \GL(V)$ a \emph{Hodge structure} on 
$V$. The natural embedding $\weil_{\dC/\dR}\Gm\monic\GL(\dC)_{/\dR}$ can be 
made explicit via
\begin{align*}
  \dS(A) 
    &= (A\otimes_\dR\dC)^\times \\
    &= \{((a,b)\in A\times A:a^2+b^2\in A^\times\} \\
    &\simeq \left\{\begin{pmatrix} a & -b \\ b & a \end{pmatrix}\right\}\subset \GL_2(A) .
\end{align*}
See \S 2 of Deligne's paper \cite{deligne-1971} for more on Hodge structures. 
\end{example}

Just as we have a nice classification of diagonalizable groups, we'd like to 
have a classification theorem for tori (or, more generally, groups of 
multiplicative type). For general groups $G_{/k}$, we define 
\[
  \characters^\ast(G) = \hom_{\groups_{/k^\separable}}\left(G_{k^\separable},{\Gm}_{/k^\separable}\right) .
\]
This is an abelian group, and naturally has an action of 
$\Gamma_k=\galois(k^\separable/k)$. Indeed, for $\gamma\in \Gamma_k$, let 
$\gamma:\spectrum(k^\separable)\to \spectrum(k^\separable)$ be the 
induced map. The pullback $\gamma^\ast G$ is defined on 
$k^\separable$-algebras $A$ by 
\[
  (\gamma^\ast G)(A)=G(A\otimes_\gamma k^\separable). 
\]
In other words, $A$ is viewed as a $k^\separable$-algebra via 
$\gamma$, so that $x\cdot a = \gamma(x) a$. If $\chi\in \characters^\ast(G)$, 
the character $\gamma\chi$ is defined by the commutative diagram, in which 
the outer squares are pullbacks: 
\begin{equation}\tag{$\ast$}\label{eq:galois-char}
\begin{tikzcd}
  G_{k^\separable} \ar[d] \ar[r] 
    & \gamma^\ast G_{k^\separable} \ar[d] \ar[r, "\gamma^\ast \chi"]
    & \gamma^\ast {\Gm}_{/k^\separable} \ar[r] \ar[d] 
    & {\Gm}_{/k^\separable} \ar[d] \\
  k^\separable \ar[r, "\gamma^{-1}"]
    & k^\separable \ar[r, equal] 
    & k^\separable \ar[r, "\gamma"] 
    & k^\separable 
\end{tikzcd}
\end{equation}
Here we have written $k^\separable$ instead of $\spectrum(k^\separable)$ in 
order to save space. On $k^\separable$-points, this map is 
$(\gamma\chi)(g) = \gamma(\chi(\gamma^{-1} g))$. 

\begin{example}
Let $K/k$ be a finite Galois extension with $\Gamma=\galois(K/k)$. We have 
seen in \autoref{eg:nonsplit-torus} that there is a natural isomorphism 
$X^\ast(\weil_{K/k}\Gm) = \dZ[\Gamma]$ of abelian groups. It is a good 
exercise to check that this isomorphism respects the $\Gamma$-action, 
i.e.~that $X^\ast(\weil_{K/k}\Gm)=\dZ[\Gamma]$ as $\Gamma$-modules. 
\end{example}

\begin{example}
Let $\dS$ be the group in \autoref{eg:hodge-structure}. Then over $\dR$, all 
characters are powers of 
\[
  \det:\begin{pmatrix} a & -b \\ b & a \end{pmatrix}\mapsto a^2 + b^2 .
\]
But over $\dC$, there are two characters, 
\begin{align*}
  z:\begin{pmatrix} a & -b \\ b & a \end{pmatrix} &\mapsto a+b i \\
  \bar z:\begin{pmatrix} a & -b \\ b & a \end{pmatrix} &\mapsto a - b i .
\end{align*}
The group $\characters^\ast(G_\dC)$ has some extra 
structure, namely an action of the group $\galois(\dC/\dR)=\langle c\rangle$. 
The generator $c$ interchanges $z$ and $\bar z$. So under the obvious 
isomorphism $\characters^\ast(G)\simeq \dZ^2$, the action of 
$\galois(\dC/\dR)$ is $c\cdot(n_1,n_2) = (n_2,n_1)$. Moreover, 
$\h^0(\dR,\characters^\ast(G)) = \det^\dZ$, and the kernel of $\det$ on $G$ is 
\[
  \dS^{\det=1} = \left\{\begin{pmatrix} a & -b \\ b & a \end{pmatrix}:a^2+b^2=1\right\}\subset \GL(2)_{/\dR} .
\]
The group $\dS^{\det=1}$ fits inside a short exact sequence 
$1\to \dS^{\det=1}\to \dS\to \Gm\to 1$, so on the level of characters, we 
have a short exact sequence of $\dZ[\galois(\dC/\dR)]$-modules 
\[\
  0 \to \characters^\ast(\Gm) \to \characters^\ast(\dS) \to \characters^\ast(\dS^{\det=1}) \to 0.
\]
Complex conjugation $c$ acts on the quotient 
$\characters^\ast(\dS^{\det=1})$ as multiplication by $-1$. 
\end{example}

\begin{theorem}
Let $k$ be a field, $\Gamma=\galois(k^\mathrm{sep}/k)$. Then the functor 
$\characters^\ast$ induces an exact anti-equivalence of categories 
\[
  \{\text{groups of mult.~type over $k$}\} \iso \{\text{f.g.~ab.~groups with cont.~$\Gamma$-action}\} .
\]
\end{theorem}
\begin{proof}
This is \cite[X 1.4]{sga3-ii}. 
\end{proof}

In general, we call the action of a profinite group $\Gamma$ on a discrete 
set $X$ \emph{continuous} if the map $\Gamma\times X\to X$ is continuous. 
Equivalently, for each $x\in X$, the stabilizer $\stabilizer_\Gamma(x)$ is 
open. The action of $\Gamma$ on $\characters^\ast(G)$ is as given in 
\eqref{eq:galois-char}. 

\begin{hard}
In the proof of \autoref{thm:lie-kolchin}, we saw that it is possible to 
view $\characters^\ast(G)$ as a group scheme. This makes it possible to recover 
the Galois action on $\characters^\ast(G)$ in a more natural manner. It is 
a basic fact (see \cite[V 8.1]{sga1}) that the category of finite sets with 
continuous $\Gamma_k$-action is equivalent to the category of finite \'etale 
covers of $\spectrum(k)$. Thus the category of sets with continuous 
$\Gamma_k$-action is equivalent to the category of sheaves on $k_\etale$. 

In light of this, we construct $\characters^\ast(G)$ as a scheme; its 
restriction to $k_\etale$ recovers the usual definition of 
$\characters^\ast(G)$. Put 
\[
  \characters^\ast(G)(A) = \hom_{\groups_{/A}}\left(G_A,{\Gm}_{/A}\right) .
\]
If $G$ is of multiplicative type, then by \cite[XI 4.2]{sga3-ii}, the 
group functor $\characters^\ast(G)$ is represented by a smooth separated 
$k$-scheme. The restriction of this scheme to $k_\etale$ recovers the 
usual definition of $\characters^\ast(G)$. 
\end{hard}

\begin{example}
Let $k$ be a field of characteristic $\ne 2$. The Kummer theory tells us that 
quadratic extensions $K/k$ are all of the form $k(\sqrt d)/k$ for 
$d\in k^\times/2=\h^1(k,\dmu_2)$. Since $2\in k^\times$, we have 
$\dmu_2 = \dZ/2 = \automorphisms(\Gm)$. Thus $k$-forms of $\Gm$ are 
classified by $\h^1(k,\dZ/2) = k^\times/2$. We can work this out 
explicitly. For $c\in \h^1(k,\dmu_2)$, let $k_c/k$ be the corresponding 
quadratic extension, and let $c^\ast\Gm$ be defined by 
\[
  c^\ast \Gm = \ker\left( \weil_{k_c/k}\Gm \monic \GL(k_c)_{/k}\xrightarrow{\det} \Gm\right) .
\]
The group $c^\ast\Gm$ can be written explicitly as 
\[
  \left\{\begin{pmatrix} a & b c \\ b & a \end{pmatrix}:a^2 - c b^2=1\right\} \subset \GL(2)_{/k} .
\]
The isomorphism type of $c^\ast\Gm$ depends only on the class of $c$ in 
$k^\times/2$. Moreover, $\characters^\ast(c^\ast\Gm)\simeq \dZ$, with the 
action of $\Gamma_k$ factoring through $\galois(k_c/k)$ and being 
multiplication by $-1$ on the unique generator of $\galois(k_c/k)$. 
\end{example}

Note that general tori can be extremely difficult to classify. For example, 
when $k=\dQ$, the category of tori over $\dQ$ is anti-equivalent to the 
category of finitely generated abelian groups with $\Gamma_\dQ$-action. The 
group $\Gamma_\dQ$ can be recovered from its finite quotients, but it is 
still very mysterious. 

\begin{example}
Over $\dR$, if we have an involution $c$ acting on a lattice $\dZ^n$, then 
with respect to some basis, $c$ wiill have the form 
\[
  \begin{pmatrix} 1 \end{pmatrix}^{\oplus a} \oplus\begin{pmatrix} -1 \end{pmatrix}^{\oplus b} \oplus\begin{pmatrix} & 1 \\ 1 \end{pmatrix}^{\oplus c}
\]
So any torus over $\dR$ breaks up as a product of copies of $\Gm$, copies of 
$\dS$, and copies of $\dS^{\det=1}$. 
\end{example}

Over a general field, any torus is a quotient of products of 
$\weil_{K/k}\Gm$ for varying $K/k$. 

If $T$ is a diagonalizable group and $\rho:T\to \GL(V)$ a representation of 
$T$, then we have a direct sum decomposition 
\[
  V=\bigoplus_{\chi\in \characters^\ast(T)} V_\chi ,
\]
where $V_\chi=\{v\in V:\rho(g)v = \chi(g) v\text{ for all }g\}$. The 
$\chi$ for which $V_\chi\ne 0$ will be called the \emph{weights} of $V$. To 
summarize, every representation of a split torus is semisimple, the direct 
sum of a bunch of characters. 

For $T$ of multiplicative type, all representations are semisimple. The 
irreducible representations are given by Galois orbits in 
$\characters^\ast(T)$. 

Tori have strong rigidity properties which will be useful later. Recall that 
if $T_2,T_2$ are tori, we write $\hom(T_1,T_2)$ for the scheme over $k$ whose 
functor of points is 
\[
  \hom(T_1,T_2)(S) = \hom_{\groups_{/S}}\left({T_1}_{S},{T_2}_S\right) .
\]
By \cite[XI 4.2]{sga3-ii}, this is represented by a smooth scheme over $k$. If 
$X_{/k}$ is a scheme, a \emph{family of homomorphisms} from $T_1$ to $T_2$ 
parameterized by $X$ is just a morphism $X\to \hom(T_1,T_2)$. By general 
nonsense, it is equivalent to give a morphism $\phi:X\times T_1\to T_2$ such 
that for all $x\in X(S)$, the map ${T_1}_S\to {T_2}_S$ given by 
$t\mapsto \phi(x,t)$ is a homomorphism. We say a family of homomorphisms 
$\phi:X\to \hom(T_1,T_2)$ is \emph{constant} if the morphism $\phi$ is 
constant, i.e.~if $\phi(x)$ does not depend on $x$. Alternatively, the 
morphism $X\times T_1\to T_2$ factors as 
$X\times T_1\epic T_1\xrightarrow \phi T_2$. 

\begin{theorem}[Rigidity of tori]\label{thm:rigidity-tori}
Let $k$ be a field, $T_{/k}$ a torus, $X_{/k}$ a connected variety. Any 
family $X\to \hom(T,T)$ of homomorphisms is constant. 
\end{theorem}
\begin{proof}
We'll use two facts. 
\begin{enumerate}
\item For $n\geqslant 1$, let $\torsion n T = \ker(T\xrightarrow n T)$. Then 
$\{\torsion n T\}$ is dense in $T$ \cite[IX 4.7]{sga3-ii}. 

\item For each $n\geqslant 1$ invertible in $k$, the scheme $\torsion n T$ is 
finite \'etale over $k$. (This follows from \'etale descent.)
\end{enumerate}
The family $X\to \hom(T,T)$ induces, for each $n$ invertible in $k$, a 
family $X\to \hom(\torsion n T,\torsion n )$, which by \cite[X 4.2]{sga3-ii} is 
finite \'etale. Since $X$ is connected, each 
$X\to \hom(\torsion n T,\torsion n T)$ is constant. By the first fact, 
we obtain that $X$ is constant. 
\end{proof}


\begin{example}
This shows that \autoref{thm:rigidity-tori} fails for groups that are not 
tori. Let $G=\SL(2)$. Then the map $G\times G\to G$ given by 
$(g,h)\mapsto g h g^{-1}$ is a perfectly good family of homomorphisms that 
is not constant. One has $\PSL(2)\monic\automorphisms(\SL_2)$, with image of 
index two. The other coset is generated by $g\mapsto \transpose{g}^{-1}$. 
\end{example}










