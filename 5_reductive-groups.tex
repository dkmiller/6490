% !TEX root = 6490.tex

\section{Reductive groups}

Let $k$ be an algebraically closed field of characteristic zero. Let 
$G_{/k}$ be a reductive group, i.e.~$\urad G=1$. A good example is $\GL(n)$. 
We could like to determine $G$ up to isomorphism via some combinatorial data. 
Let $T\subset G$ be a maximal torus. Better, let $G_{/k}$ be a split reductive 
group and $k$ arbitrary of characteristic zero. Let $X=\characters^\ast(T)$. 
We have the adjoint representation $\adjoint:G\to \GL(\fg)$. Just as before, we 
can write 
\[
  \fg = \fg_0\oplus \bigoplus_{\alpha\in R} \fg_\alpha ,
\]
where $\fg_\alpha=\{x\in \fg:t x=\alpha(t) x\text{ for all }t\}$ and 
$R=\{\alpha\in X\smallsetminus 0:\fg_\alpha\ne 0\}$. 

Note that $R\subset X_\dR$ need not be a root system, since $R$ may only span a 
proper subspace of $X_\dR$. For example, if $G$ is itself a torus, then 
$R=\varnothing$. However, $R\subset V\subset X_\dR$, where $V=\dR\cdot R$, is a 
root system, and it determines the semisimple group $G/\rad G$ up to isogeny. 
The datum ``$R\subset X$'' is not in general enough to determine $G$. 

Let $\characters_\ast(T)=\hom(\Gm,T)$. There is a perfect pairing (composition) 
$\characters^\ast(T)\times \characters_\ast(T)\to \dZ=\hom(\Gm,\Gm)$. Here 
$\langle \alpha,\beta\rangle$ is the integer $n$ such that 
$\alpha\circ\beta$ is $(-)^n$. From $G$ we'll construct a root datum, which 
will be an ordered quadruple 
$(\characters^\ast(T),\roots(G,T),\characters_\ast(T),\check\roots(G,T))$. 
Before doing this, we'll define root data in general. 





\subsection{Root data}\label{sec:root-data}

The following definitions are from \cite[XXI]{sga3-iii}. A \emph{dual pair} 
is an ordered pair $(X,\check X)$, where $X$ and $\check X$ are finitely 
generated free abelian groups, together with a pairing 
$\langle\cdot,\cdot\rangle:X\times \check X\to \dZ$ that induces an isomorphism 
$\check X\simeq X^\vee=\hom(X,\dZ)$. Let $(X,\check X)$ be a dual pair, and 
suppose we have two elements $\alpha\in X$, $\check\alpha\in \check X$. Define 
the \emph{reflections} $s_\alpha$ and $s_{\check\alpha}$ by 
\begin{align*}
  s_\alpha(x) &= x-\langle x,\check\alpha\rangle\alpha \\
  s_{\check\alpha}(x) &= x-\langle\alpha,x\rangle\check\alpha .
\end{align*}

\begin{definition}
A \emph{root datum} consists of an ordered quadruple $(X,R,\check X,\check R)$, 
such that 
\begin{itemize}
  \item $(X,\check X)$ is a dual pair. 
  \item $R\subset X$ and $\check R\subset \check X$ are finite sets, 
  \item There is a specified mapping $\alpha\mapsto \check\alpha$ from $R$ to 
    $\check R$. 
\end{itemize}
These data are required to satisfy the following conditions:
\begin{enumerate}
  \item For each $\alpha\in R$, $\langle\alpha,\check\alpha\rangle=2$. 
  \item For each $\alpha\in R$, $s_\alpha(R)\subset R$ and 
    $s_{\check\alpha}(\check R)\subset \check R$. 
\end{enumerate}
\end{definition}

It turns out that $\alpha\mapsto \check\alpha$ is a bijection, and that 
$R$ and $\check R$ are closed under negation. If $\sR=(X,R,\check X,\check R)$ 
is a root datum, the \emph{Weyl group} of $\sR$ is the group  
$\weyl(\sR)\subset \GL(X)$ generated by $\{s_\alpha:\alpha\in R\}$. By 
\cite[XXI 1.2.8]{sga3-iii}, the Weyl group of a root datum is finite. 

\begin{definition}
Let $\sR_1=(X_1,R_1,\check X_1,\check R_1)$ and 
$\sR_2=(X_2,R_2,\check X_2,\check R_2)$ be root data. A \emph{morphism} 
$f:\sR_1\to \sR_2$ consists of a linear map $f:X_1\to X_2$ such that 
\begin{enumerate}
  \item $f$ induces a bijection $R_1\iso R_2$, 
  \item The dual map $f^\vee:\check X_2\to \check X_1$ induces a bijection 
    $\check R_2\iso \check R_1$. 
\end{enumerate}
\end{definition}

Write $\rootdata$ for the category of root data. There is a contravariant 
functor $(-)^\vee:\rootdata\to \rootdata$ which sends a root datum 
$\sR=(X,R,\check X,\check R)$ to $\sR^\vee=(\check X,\check R,X,R)$. This is 
clearly an anti-equivalence of categories. We will use this later to define 
the \emph{dual} of a split reductive group. If $G_{/k}$ is a split reductive 
group, then $G^\vee$ will be a split reductive group scheme over $\dZ$, whose 
root datum is $\sR(G,T)^\vee$. 





\subsection{Root datum of a reductive group}

Let $k$ be a field of characteristic zero, $G_{/k}$ a (connected) split 
reductive group. Let $T\subset G$ be a split maximal torus. The \emph{rank} of 
$G$ is the integer $r=\rank(G)=\dim(T)$. Let $\fg=\lie(G)$. As we have seen 
many times before, we can decompose $\fg$ as a representation of $T$: 
\[
  \fg=\ft\oplus \bigoplus_{\alpha\in R} \fg_\alpha ,
\]
where $\ft=\lie(T)$, $R=\roots(G,T)$ is the set of \emph{roots} of $G$, and 
$\fg_\alpha$ is the $\alpha$-typical component of $\fg$. Recall that the 
\emph{Weyl group} of $G$ is $\weyl(G,T)=\normalizer_G(T)/\centralizer_G(T)$. By 
\cite[XII 2.1]{sga3-ii}, the Weyl group is finite. 

A good source for what follows is \cite[II.1]{jantzen-2003}. Let 
$\alpha:T\to \Gm$ be a root, and put $T_\alpha=(\ker\alpha)^\circ$; this is a 
closed $(r-1)$-dimensional subgroup of $T$. Let 
$G_\alpha=\centralizer_G(T_\alpha)$; by \autoref{thm:centralizer-reductive}, 
this is reductive. One has $\zentrum(G_\alpha)^\circ=T_\alpha$. From the 
embedding $G_\alpha\monic G$, we get an embedding of Lie algebras 
$\lie(G_\alpha)\monic \fg$. By \cite[IX 3.5]{sga3-iii}, one has 
\[
  \lie(G_\alpha) = \ft\oplus \fg_\alpha \oplus \fg_{-\alpha} 
\]
and $\fg_\alpha,\fg_{-\alpha}$ are both one-dimensional. In classical Lie 
theory, we could define $U_\alpha=\exp(\fg_\alpha)$. Since the exponential map 
does not make sense in full generality, we need the following result. Recall 
that we interpret the Lie algebra of an algebraic group as a new algebraic 
group. 

\begin{theorem}
Let $T$ act on $G$ via conjugation. Then there is a unique $T$-equivariant 
closed immersion $u_\alpha:\fg_\alpha\to G$ which is also a group 
homomorphism. 
\end{theorem}
\begin{proof}
This is \cite[XXII 1.1.i]{sga3-iii}. 
\end{proof}

We let $U_\alpha$ be the image of $u_\alpha$. Since $\fg_\alpha$ is 
one-dimensional, it is isomorphic to $\Ga$ as a group scheme. So we will 
generally write $u_\alpha:\Ga\iso U_\alpha\subset G$. The fact that 
$u_\alpha$ is $T$-equivariant comes down to: 
\[
  t u_\alpha(x) t^{-1} = u_\alpha(\alpha(t) x) \qquad t\in T, x\in \Ga .
\]
The group $G_\alpha$ is generated by $T$, $U_{\pm\alpha}$ by Lie 
algebra considerations. So the group $G$ is generated by $T$ and 
$\{U_\alpha:\alpha\in R\}$. 

\begin{example}[type $\typeA_n$]
Inside $\SL(n+1)$, the maximal torus $T$ consists of diagonal matrices 
$\diagonal(t)=\diagonal(t_1,\dots,t_{n+1})$ for which 
$t_1\dotsm t_{n+1}=1$. The group $\characters^\ast(T)$ is generated by the 
characters $\chi_i(\diagonal(t))=t_i$. The roots are 
$\{\chi_i-\chi_j:i\ne j\}$. For such a root, one can verify that 
$G_\alpha=\centralizer_G(T_\alpha)$ consists of matrices of the 
form $(g_{a b})$ for which $g_{a b}=\delta_{a b}$ unless 
$a=b=i$, $a=b=j$, or $(a,b)=(i,j)$. So $G_\alpha\simeq \SL(2)$. 
It follows that $U_\alpha = 1+\Ga e_{i j}$. Note that indeed, $\SL(n+1)$ is 
generated by $\{U_\alpha:\alpha\in R\}$ and $T$. 
\end{example}

Back to our general setup $T\subset G_\alpha\subset G$. We can consider the 
``small Weyl group'' $\weyl(G_\alpha,T)\subset \weyl(G,T)$. Since 
$\rank(G_\alpha/\zentrum(G_\alpha)^\circ)=1$, we have 
$\weyl(G_\alpha,T)=\dZ/2$. Let $s_\alpha$ be the unique generator of 
$\weyl(G_\alpha,T)$. It is known that $W=\weyl(G,T)$ is generated by 
$\{s_\alpha:\alpha\in R\}$. The group $W$ acts on $\characters^\ast(T)$. 
Indeed, if $w\in W$, we have $w=\dot n$ for some $n\in \normalizer_G(T)$. 
Define $(w\cdot\chi)(t)=\chi(\dot n^{-1} t \dot n)$. 

Recall that $\characters_\ast(T)=\hom(\Gm,T)$. There is a natural pairing 
$\characters^\ast(T)\times \characters_\ast(T)\to \dZ$, for which 
$\langle \alpha,\beta\rangle$ is the unique $n$ such that $\alpha\beta$ is 
$t\mapsto t^n$. That is, $(\alpha\beta)(t) = t^{\langle\alpha,\beta\rangle}$. 
This pairing induces an isomorphism 
$\characters_\ast(T)\simeq \characters^\ast(T)^\vee$. 

\begin{theorem}
Let $\alpha\in R$. Then there exists a unique 
$\check\alpha\in \characters_\ast(T)$ such that 
$s_\alpha(x) = x-\langle x,\check\alpha\rangle\alpha$ for all 
$x\in \characters^\ast(T)$. Moreover, $\langle\alpha,\check\alpha\rangle=2$. 
\end{theorem}
\begin{proof}
This is \cite[XXII 1.1.ii]{sga3-iii}. 
\end{proof}

Equivalently, $s_\alpha(\alpha)=-\alpha$. We put 
$\check\roots(G,T)=\{\check\alpha:\alpha\in R\}$. The quadruple 
\[
  \sR(G,T) = (\characters^\ast(T),\roots(G,T),\characters_\ast(T),\check \roots(G,T))
\]
is the \emph{root datum} of $G$; it will determine $G$ up to isomorphism (not 
just isogeny). 

In defining coroots, we could also have used the fact that 
$\zentrum(G_\alpha)^\circ=T_\alpha$, which implies 
$T/T_\alpha\monic G_\alpha/T_\alpha$ is a maximal torus of dimension $1$. So 
the group $G_\alpha/T_\alpha$ is semisimple, and has one-dimensional maximal 
torus. Its Dynkin diagram has type $\typeA_1$, so $G_\alpha/T_\alpha$ is either 
$\SL(2)$ or $\PGL(2)$. 




