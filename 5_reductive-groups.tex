% !TEX root = 6490.tex

\section{Reductive groups}

Let $k$ be an algebraically closed field of characteristic zero. Let 
$G_{/k}$ be a reductive group, i.e.~$\urad G=1$. A good example is $\GL(n)$. 
We could like to determine $G$ up to isomorphism via some combinatorial data. 
Let $T\subset G$ be a maximal torus. Better, let $G_{/k}$ be a split reductive 
group and $k$ arbitrary of characteristic zero. Let $X=\characters^\ast(T)$. 
We have the adjoint representation $\adjoint:G\to \GL(\fg)$. Just as before, we 
can write 
\[
  \fg = \fg_0\oplus \bigoplus_{\alpha\in R} \fg_\alpha ,
\]
where $\fg_\alpha=\{x\in \fg:t x=\alpha(t) x\text{ for all }t\}$ and 
$R=\{\alpha\in X\smallsetminus 0:\fg_\alpha\ne 0\}$. 

Note that $R\subset X_\dR$ need not be a root system, since $R$ may only span a 
proper subspace of $X_\dR$. For example, if $G$ is itself a torus, then 
$R=\varnothing$. However, $R\subset V\subset X_\dR$, where $V=\dR\cdot R$, is a 
root system, and it determines the semisimple group $G/\rad G$ up to isogeny. 
The datum ``$R\subset X$'' is not in general enough to determine $G$. 

Let $\characters_\ast(T)=\hom(\Gm,T)$. There is a perfect pairing (composition) 
$\characters^\ast(T)\times \characters_\ast(T)\to \dZ=\hom(\Gm,\Gm)$. Here 
$\langle \alpha,\beta\rangle$ is the integer $n$ such that 
$\alpha\circ\beta$ is $(-)^n$. From $G$ we'll construct a root datum, which 
will be an ordered quadruple 
$(\characters^\ast(T),\roots(G,T),\characters_\ast(T),\check\roots(G,T))$. 
Before doing this, we'll define root data in general. 





\subsection{Root data}\label{sec:root-data}

The following definitions are from \cite[XXI]{sga3-iii}. A \emph{dual pair} 
is an ordered pair $(X,\check X)$, where $X$ and $\check X$ are finitely 
generated free abelian groups, together with a pairing 
$\langle\cdot,\cdot\rangle:X\times \check X\to \dZ$ that induces an isomorphism 
$\check X\simeq X^\vee=\hom(X,\dZ)$. Let $(X,\check X)$ be a dual pair, and 
suppose we have two elements $\alpha\in X$, $\check\alpha\in \check X$. Define 
the \emph{reflections} $s_\alpha$ and $s_{\check\alpha}$ by 
\begin{align*}
  s_\alpha(x) &= x-\langle x,\check\alpha\rangle\alpha \\
  s_{\check\alpha}(x) &= x-\langle\alpha,x\rangle\check\alpha .
\end{align*}

\begin{definition}
A \emph{root datum} consists of an ordered quadruple $(X,R,\check X,\check R)$, 
such that 
\begin{itemize}
  \item $(X,\check X)$ is a dual pair. 
  \item $R\subset X$ and $\check R\subset \check X$ are finite sets, 
  \item There is a specified mapping $\alpha\mapsto \check\alpha$ from $R$ to 
    $\check R$. 
\end{itemize}
These data are required to satisfy the following conditions:
\begin{enumerate}
  \item For each $\alpha\in R$, $\langle\alpha,\check\alpha\rangle=2$. 
  \item For each $\alpha\in R$, $s_\alpha(R)\subset R$ and 
    $s_{\check\alpha}(\check R)\subset \check R$. 
\end{enumerate}
\end{definition}

It turns out that $\alpha\mapsto \check\alpha$ is a bijection, and that 
$R$ and $\check R$ are closed under negation. If $\sR=(X,R,\check X,\check R)$ 
is a root datum, the \emph{Weyl group} of $\sR$ is the group  
$\weyl(\sR)\subset \GL(X)$ generated by $\{s_\alpha:\alpha\in R\}$. By 
\cite[XXI 1.2.8]{sga3-iii}, the Weyl group of a root datum is finite. 

\begin{definition}
Let $\sR_1=(X_1,R_1,\check X_1,\check R_1)$ and 
$\sR_2=(X_2,R_2,\check X_2,\check R_2)$ be root data. A \emph{morphism} 
$f:\sR_1\to \sR_2$ consists of a linear map $f:X_1\to X_2$ such that 
\begin{enumerate}
  \item $f$ induces a bijection $R_1\iso R_2$, 
  \item The dual map $f^\vee:\check X_2\to \check X_1$ induces a bijection 
    $\check R_2\iso \check R_1$. 
\end{enumerate}
\end{definition}

\begin{definition}
Let $\sR=(X,R,\check X,\check R)$ be a root datum. We say $\sR$ is 
\emph{reduced} if for any $\alpha\in R$, the only multiples of $\alpha$ in 
$R$ are $\pm \alpha$. 
\end{definition}

Just as with root systems, there is a notion of a base for root data. If 
$N\subset \dQ$ and $S\subset X$, we write $N\cdot S$ for the subset of 
$X_\dQ$ generated as a monoid by $\{n s:(n,s)\in N\times S\}$. We 
call a root $\alpha\in R$ \emph{indivisible} if there does not exist any 
$\beta\in R$ for which $\alpha= n \beta$ for some $n>1$, 
i.e.~$\dQ^+\cdot\alpha\cap R=\{\alpha\}$.

\begin{definition}
Let $\sR=(X,R,\check X,\check R)$ be a root system. A subset $\Delta\subset R$ 
is a \emph{base} if it satisfies any of the following conditions (reproduced 
from \cite[XXI 3.1.5]{sga3-iii}):
\begin{enumerate}
  \item All $\alpha\in \Delta$ are indivisible, and 
    $R\subset \dQ^+\cdot\Delta\cup \dQ^-\cdot\Delta$. 
  \item The set $\Delta$ is linearly independent, and 
    $R\subset \dN\cdot \Delta\cup \dZ^-\cdot\Delta$. 
  \item Each $\alpha\in R$ can be written uniquely as 
    $\sum_{\beta\in R} m_\beta \beta$, where all $m_\beta\in \dZ$ and have the 
    same sign. 
\end{enumerate}
\end{definition}

If $\Delta\subset R$ is a base, write $R^+=\dN\cdot \Delta$ for the set of 
\emph{positive roots} and $R^-=\dZ^-\cdot\Delta$ for the set of \emph{negative 
roots}. One has $R=R^+\sqcup R^-$, and $R^-=-R^+$. 

Write $\rootdata$ for the category of root data. There is a contravariant 
functor $\check\cdot:\rootdata\to \rootdata$ which sends a root datum 
$\sR=(X,R,\check X,\check R)$ to $\check\sR=(\check X,\check R,X,R)$. This is 
clearly an anti-equivalence of categories. We will use this later to define 
the \emph{dual} of a split reductive group. If $G_{/k}$ is a split reductive 
group, then $\check G$ will be a split reductive group scheme over $\dZ$, whose 
root datum is $\sR(G,T)^\vee$. 

We'll write $\rootdata^\mathrm{red}$ for the category of reduced root data. 

\begin{example}
Let $R\subset V$ be a root system. As in \autoref{sec:semisimple-classify}, we 
have lattices $Q(R)\subset V(R)\subset V$. Choose a $W$-invariant inner product 
$\langle\cdot,\cdot\rangle$ on $V$. Given $Q\subset X\subset P$, we can 
define a root datum as follows:
\begin{align*}
  X &= X \\
  R &= R \\
  \check X &= \{v\in V:\langle \alpha,v\rangle\in \dZ\text{ for all }\alpha\in R\} \\
  \check R &= \left\{\frac{2\alpha}{\langle \alpha,\alpha\rangle}:\alpha\in R\right\} .
\end{align*}
It is easy to check that this satisfies the required properties. 
\end{example}

For a classification of ``reduced simply connected root systems,'' see 
\cite[XXI 7.4.6]{sga3-iii}. 





\subsection{Classification of reductive groups}

Let $k$ be a field of characteristic zero, $G_{/k}$ a (connected) split 
reductive group. Let $T\subset G$ be a split maximal torus. The \emph{rank} of 
$G$ is the integer $r=\rank(G)=\dim(T)$. Let $\fg=\lie(G)$. As we have seen 
many times before, we can decompose $\fg$ as a representation of $T$: 
\[
  \fg=\ft\oplus \bigoplus_{\alpha\in R} \fg_\alpha ,
\]
where $\ft=\lie(T)$, $R=\roots(G,T)$ is the set of \emph{roots} of $G$, and 
$\fg_\alpha$ is the $\alpha$-typical component of $\fg$. Recall that the 
\emph{Weyl group} of $G$ is $\weyl(G,T)=\normalizer_G(T)/\centralizer_G(T)$. By 
\cite[XII 2.1]{sga3-ii}, the Weyl group is finite. 

A good source for what follows is \cite[II.1]{jantzen-2003}. Let 
$\alpha:T\to \Gm$ be a root, and put $T_\alpha=(\ker\alpha)^\circ$; this is a 
closed $(r-1)$-dimensional subgroup of $T$. Let 
$G_\alpha=\centralizer_G(T_\alpha)$; by \autoref{thm:centralizer-reductive}, 
this is reductive. One has $\zentrum(G_\alpha)^\circ=T_\alpha$. From the 
embedding $G_\alpha\monic G$, we get an embedding of Lie algebras 
$\lie(G_\alpha)\monic \fg$. By \cite[IX 3.5]{sga3-iii}, one has 
\[
  \lie(G_\alpha) = \ft\oplus \fg_\alpha \oplus \fg_{-\alpha} 
\]
and $\fg_\alpha,\fg_{-\alpha}$ are both one-dimensional. In classical Lie 
theory, we could define $U_\alpha=\exp(\fg_\alpha)$. Since the exponential map 
does not make sense in full generality, we need the following result. Recall 
that we interpret the Lie algebra of an algebraic group as a new algebraic 
group. 

\begin{theorem}
Let $T$ act on $G$ via conjugation. Then there is a unique $T$-equivariant 
closed immersion $u_\alpha:\fg_\alpha\monic G$ which is also a group 
homomorphism. 
\end{theorem}
\begin{proof}
This is \cite[XXII 1.1.i]{sga3-iii}. 
\end{proof}

We let $U_\alpha$ be the image of $u_\alpha$. Since $\fg_\alpha$ is 
one-dimensional, it is isomorphic to $\Ga$ as a group scheme. So we will 
generally write $u_\alpha:\Ga\iso U_\alpha\subset G$. The fact that 
$u_\alpha$ is $T$-equivariant comes down to: 
\[
  t u_\alpha(x) t^{-1} = u_\alpha(\alpha(t) x) \qquad t\in T, x\in \Ga .
\]
The group $G_\alpha$ is generated by $T$, $U_{\pm\alpha}$ by Lie 
algebra considerations. So the group $G$ is generated by $T$ and 
$\{U_\alpha:\alpha\in R\}$. 

\begin{example}[type $\typeA_n$]
Inside $\SL(n+1)$, the maximal torus $T$ consists of diagonal matrices 
$\diagonal(t)=\diagonal(t_1,\dots,t_{n+1})$ for which 
$t_1\dotsm t_{n+1}=1$. The group $\characters^\ast(T)$ is generated by the 
characters $\chi_i(\diagonal(t))=t_i$. The roots are 
$\{\chi_i-\chi_j:i\ne j\}$. For such a root, one can verify that 
$G_\alpha=\centralizer_G(T_\alpha)$ consists of matrices of the 
form $(g_{a b})$ for which $g_{a b}=\delta_{a b}$ unless 
$a=b=i$, $a=b=j$, or $(a,b)=(i,j)$. So $G_\alpha\simeq \SL(2)$. 
It follows that $U_\alpha = 1+\Ga e_{i j}$. Note that indeed, $\SL(n+1)$ is 
generated by $\{U_\alpha:\alpha\in R\}$ and $T$. 
\end{example}

Back to our general setup $T\subset G_\alpha\subset G$. We can consider the 
``small Weyl group'' $\weyl(G_\alpha,T)\subset \weyl(G,T)$. Since 
$\rank(G_\alpha/\zentrum(G_\alpha)^\circ)=1$, we have 
$\weyl(G_\alpha,T)=\dZ/2$. Let $s_\alpha$ be the unique generator of 
$\weyl(G_\alpha,T)$. It is known that $W=\weyl(G,T)$ is generated by 
$\{s_\alpha:\alpha\in R\}$. The group $W$ acts on $\characters^\ast(T)$. 
Indeed, if $w\in W$, we have $w=\dot n$ for some $n\in \normalizer_G(T)$. 
Define $(w\cdot\chi)(t)=\chi(\dot n^{-1} t \dot n)$. 

Recall that $\characters_\ast(T)=\hom(\Gm,T)$. There is a natural pairing 
$\characters^\ast(T)\times \characters_\ast(T)\to \dZ$, for which 
$\langle \alpha,\beta\rangle$ is the unique $n$ such that $\alpha\beta$ is 
$t\mapsto t^n$. That is, $\alpha(\beta(t)) = t^{\langle\alpha,\beta\rangle}$. 
This pairing induces an isomorphism 
$\characters_\ast(T)\simeq \characters^\ast(T)^\vee$. 

\begin{theorem}
Let $\alpha\in R$. Then there exists a unique 
$\check\alpha\in \characters_\ast(T)$ such that 
$s_\alpha(x) = x-\langle x,\check\alpha\rangle\alpha$ for all 
$x\in \characters^\ast(T)$. Moreover, $\langle\alpha,\check\alpha\rangle=2$. 
\end{theorem}
\begin{proof}
This is \cite[XXII 1.1.ii]{sga3-iii}. 
\end{proof}

Equivalently, $s_\alpha(\alpha)=-\alpha$. We put 
$\check\roots(G,T)=\{\check\alpha:\alpha\in R\}$. The quadruple 
\[
  \sR(G,T) = (\characters^\ast(T),\roots(G,T),\characters_\ast(T),\check \roots(G,T))
\]
is the \emph{root datum} of $G$; it will determine $G$ up to isomorphism (not 
just isogeny). 

In defining coroots, we could also have used the fact that 
$\zentrum(G_\alpha)^\circ=T_\alpha$, which implies 
$T/T_\alpha\monic G_\alpha/T_\alpha$ is a maximal torus of dimension $1$. So 
the group $G_\alpha/T_\alpha$ is semisimple, and has one-dimensional maximal 
torus. Its Dynkin diagram has type $\typeA_1$, so $G_\alpha/T_\alpha$ is either 
$\SL(2)$ or $\PGL(2)$. 

\begin{definition}
Let $k$ be a field, $G_{/k}$ a split reductive group. A \emph{pinning} of $G$ 
consists of the following data:
\begin{enumerate}
  \item A maximal torus $T\subset G$. 
  \item An isomorphism $T\iso \diag(M)$ for some free abelian group $M$. 
  \item A base $\Delta\subset \roots(G,T)$. 
  \item For each $\alpha\in \Delta$, a non-zero element 
    $X_\alpha\in \fg_\alpha$. 
\end{enumerate}
\end{definition}

If $G_1,G_2$ are pinned reductive groups, a morphism $f:G_1\to G_2$ is said to 
be \emph{compatible with the pinnings} if $f(T_1)=f(T_2)$, $f$ induces a 
bijection (written $f_\ast$) $R_1\iso R_2$, and such that 
\[
  f(u_\alpha(X_{1,\alpha})) = u_{f_\ast\alpha}(X_{2,f_\ast\alpha}) .
\]

Let $\reductivegroups_{/k}^\mathrm{pinn}$ be the category of split reductive 
groups over $k$ with pinnings. We can define a pinning of a root system 
$\sR=(X,R,\check X,\check R)$ to be the choice of a base $\Delta\subset R$. Let 
$\rootdata^\mathrm{red,pinn}$ be the category of pinned reduced root data. The 
following combines the ``existence and uniqueness theorems'' in the 
classification of reductive groups. 

\begin{theorem}\label{thm:reductive-classification}
The operation $G\mapsto \sR(G,T)$ induces an equivalence of categories 
\[
  \reductivegroups_{/k}^\mathrm{pinn}\iso \rootdata^\mathrm{red,pinn} .
\]
\end{theorem}
\begin{proof}
See \cite[XXIII 4.1]{sga3-iii} for a proof that $\sR$ is fully faithful, 
\cite[XXV 2]{sga3-iii} for a proof of essential surjectivity. 
\end{proof}





\subsection{Chevalley-Demazure group schemes}

In \autoref{thm:reductive-classification}, we classified split reductive 
groups over an arbitrary field. It turns out that the classification works over 
an arbitrary (non-empty) base scheme. More precisely, let $S$ be a scheme, 
$G_{/S}$ a smooth affine group scheme of finite type. We say $G$ is 
\emph{reductive} if each geometric fiber $G_{\bar s}$ is reductive. A 
\emph{maximal torus} in $G$ is a subgroup scheme $T\subset G$ of multiplicative 
type such that for all $s\in S$, the geometric fiber 
$T_{\bar s}\subset G_{\bar s}$ is a maximal torus. Suppose there exists an 
abelian group $M$ with an embedding $\diag(M)_{/S}\monic G$ whose image is a 
maximal torus. Just as before, there is a decomposition 
$\fg=\ft\oplus \bigoplus \fg_\alpha$, except now $\fg$ (and the $\fg_\alpha$) 
is a locally free sheaf on $S$. We say that $G$ is \emph{split} if the 
following conditions hold: 
\begin{enumerate}
  \item Each $\fg_\alpha$ is a free $\sO_S$-module. 
  \item Each root $\alpha$ (resp.~each coroot $\check\alpha$) is constant, 
    i.e.~induced by an element of $M$ (resp.~$M^\vee$). 
\end{enumerate}
Given the obvious notion of 
``pinned reductive group over $S$,'' the classification result in 
\autoref{thm:reductive-classification} actually gives an equivalence of 
categories 
\[
  \reductivegroups_{/S}^\mathrm{pinn} \iso \rootdata^\mathrm{red,pinn} .
\]
In particular, for each reduced root datum $\sR$, there is a reductive group 
scheme $G_{\sR/\dZ}$ with root datum $\sR$, such that for any scheme $S$, the 
unique (up to isomorphism) reductive group scheme with root datum $\sR$ is the 
base-change $G_{\sR/S} = (G_{\sR/\dZ})_S$. In particular, if $k$ is a field, 
the unique split reductive $k$-group with root datum $\sR$ is 
$G_{\sR/k}$. One calls $G_{\sR/\dZ}$ the \emph{Chevalley-Demazure group scheme 
of type $\sR$}. 

We will outline a construction of $G_{\sR/\dZ}$ if $\sR$ is the adjoint 
form or a root system of type $\typeA_n$, $\typeD_n$, or $\typeE_n$. (These 
are the so-called \emph{simply-laced} root systems, given this name because 
their Dynkin diagrams have no multiple edges.)

Let $G_{/k}$ be a split reductive group, $T\subset G$ a maximal torus. For 
$\fg=\lie(G)$, $\ft=\lie(T)$, we have a canonical decomposition 
\[
  \fg = \ft\oplus \bigoplus_{\alpha\in R} \fg_\alpha .
\]
Recall that for each $\alpha\in R$, there is a unique connected subgroup 
$U_\alpha\subset G$ with $\lie(U_\alpha)=\fg_\alpha$. Moreover, 
$U_\alpha\simeq \Ga$, and $G$ is generated by $T$ and the $U_\alpha$. 
Let $\sR(G,T)=(X,R,\check X,\check R)$ be the root datum of $G$. If we choose 
a base $S\subset R$, then $T$ together with $\{U_\alpha:\alpha\in S\}$ 
generates a Borel subgroup $B$ of $T$ containing $T$. In fact, there is a 
bijection between the set of bases of $R$ and Borel subgroups 
$T\subset B\subset T$. 

If $B\subset G$ is a Borel subgroup, put 
$R^+=\{\alpha\in R:U_\alpha\subset B\}$. We can then define 
\[
  S = \{\alpha\in R^+:\alpha\text{ is irreducible}\} .
\]
Here, $\alpha\in R^+$ is \emph{irreducible} if there is no way of writing 
$\alpha=\sum_{\beta\in R^+} m_\beta \beta$ in which $\sum m_\beta>1$. 

Start with an irreducible reduced root system $\sR=(X,R,\check X,\check R)$ of 
one of the following forms: $\typeA_n$, $\typeD_n$, or $\typeE_n$. See 
\autoref{sec:root-classify} for their Dynkin diagrams. We will construct a 
split semisimple group $G$ with the given root system. First, we'll define 
$\fg=\lie(G)$. Let $Q=\dZ\cdot R$ and $\check Q=\dZ\cdot \check R$. Let 
$n=\rank(X)$ be the rank of $\sR$. We define: 
\begin{align*}
  \ft &= \check Q \\
  \fg &= \ft\oplus \bigoplus_{\alpha\in R} \dZ\cdot e_\alpha .
\end{align*}
This is a free $\dZ$-module (i.e., a coherent sheaf on $\spectrum(\dZ)$). We 
need to define a Lie bracket on $\fg$. It suffices to define: 
\begin{align*}
  [x_1,x_2] &= 0 && x_i\in \ft \\
  [x,e_\alpha] &= \langle \alpha,u\rangle e_\alpha && x\in \ft, \alpha\in R \\
  [e_\alpha,e_{-\alpha}] &= \check\alpha && \alpha\in R \\
  [e_\alpha,e_\beta] &= 0 && \alpha+\beta\notin R \\
  [e_\alpha,e_\beta] &= c_{\alpha,\beta} \cdot e_{\alpha+\beta} && \alpha+\beta\in R .
\end{align*}
[We are working with \emph{simply laced} Dynkin diagrams.] We need to define 
the structure constants $c_{\alpha,\beta}$. Let $S=\{\alpha_1,\dots,\alpha_n\}$ 
be a base for $R$. Define a bi-additive map $f:Q\times Q\to \dZ$ by 
\[
  f(\alpha_i,\alpha_j) = \begin{cases} \langle \alpha_i,\alpha_j\rangle & i<j \\ 0 & i>j \\ 1 & i=j \end{cases} 
\]
Here we assume $\langle\alpha,\alpha\rangle=2$ for all $\alpha\in R$. Given $S$ 
we can define the set $R^+$ of positive roots. Put 
\[
  \varepsilon(\alpha) = \begin{cases} 1 & \alpha\in R^+ \\ -1 & \text{else} \end{cases} .
\]
Define 
\[
  c_{\alpha,\beta} = \varepsilon(\alpha)\varepsilon(\beta) \varepsilon(\alpha+\beta) (-1)^{f(\alpha+\beta)}\in \{\pm 1\} .
\]
We claim that $[\cdot,\cdot]$ as defined above extends to a Lie bracket on 
$\fg$. 

[wait for full construction of $G$]




Let $G_{/k}$ be a split reductive group, 
$\fg=\ft\oplus \bigoplus_{\alpha\in R} \fg_\alpha$ the root decomposition of 
its Lie algebra. Recall that there are $T$-equivariant embeddings 
$\fg_\alpha\iso U_\alpha\simeq \Ga\subset G$. The group $G$ is generated by 
$T$ and $\{U_\alpha:\alpha\in R\}$. We constructed a bijection bewteen 
isomorphism classes of split connected reductive groups over $k$ and 
root data. 

Let $\sR=(X,R,\check X,\check R)$ be a root datum corresponding to an 
adjoint simple group. For any field $k$, there is a unique split reductive 
group $G_{\sR/k}$ that is adjoint, with root datum $\sR$. So $\sR$ could be 
any of 
$\typeA_n,\typeB_n,\typeC_n,\typeD_n,\typeE_6,\typeE_7,\typeE_8,\typeF_4,\typeG_2$. 

Let $Q=\dZ\cdot R$, $\check Q=\dZ \check R$, $n=\rank(X)$. Define 
$\fg_\dZ = \ft\oplus \bigoplus_{\alpha\in R} \dZ e_\alpha$, where 
$\ft=\check Q\simeq \dZ^n$. Give $\fg_\dZ$ a Lie bracket: 
\begin{align*}
  [\ft,\ft] &= 0 \\
  [x,e_\alpha] &= \langle \alpha,x\rangle e_\alpha \qquad (x\in \ft,\alpha\in R) \\
  [e_\alpha,e_{-\alpha}] &= \check\alpha (\alpha\in R) \\
  [e_\alpha,e_\beta] &= 0 (\alpha+\beta\notin R) \\
  [e_\alpha,e_\beta] &= c_{\alpha,\beta} e_{\alpha+\beta} (\alpha+\beta\in R) .
\end{align*}
The ``mystery constants'' $c_{\alpha,\beta}$ are up to sign 
$c_{\alpha,\beta}=\pm(r+1)$, where the $r,r'$ are maximal integers such 
$\beta-r\alpha,\dots,\beta-\alpha,\beta,\beta+\alpha,\dots,\beta+r'\alpha$ are 
all in $R$. [$r=r_{\alpha,\beta}$; $r'$ not really needed here -- could have 
stopped with $\beta$.] There is a recipe for the signs (different choices). 
In [?], we gave a recipe for the $c_{\alpha,\beta}$ for simply laced diagrams. 

The key point is: we end up with a simple Lie algebra $\fg_\dZ$ over $\dZ$. 
For any field $k$, $\fg_\dZ\otimes k$ is a Lie algebra over $k$ with the same 
root datum. We want to recover a group from $\fg_k=\fg_\dZ\otimes k$. 

We'll construct $G$ such that $\adjoint:G\to \GL(\fg)$ is an embedding. First, 
we'll construct the $U_\alpha\subset \GL(\fg)$. Define a homomorphism 
$u_\alpha:\Ga\to \GL(\fg)$ by $x\mapsto \exp(x\cdot \adjoint(e_\alpha)) = \sum_{n\geqslant 0} \frac{x^n\adjoint(e_\alpha)^n}{n!}$. One has: 
\[
  u_\alpha(x) = 1+x\adjoint(e_\alpha) + x^2\frac{\adjoint(e_\alpha)^2}{2} .
\]
(except in $\typeG_2$. This makes sense if $2$ is invertible. The whole thing 
$\frac 1 2 \adjoint(e_\alpha)^2$ works over $\dZ$\ldots the whole power series 
actually lives over $\dZ$.) So we've defined an injective homomorphism 
$u_\alpha:\Ga\monic \GL(\fg)$; put $U_\alpha=\image(u_\alpha)$. 

We'll construct a split torus $T=\Gm^n$. Identify $X(T)$ with $Q=\dZ R$. Then 
$T\monic \GL(\fg)$, where $T$ acts on $\ft$ trivially, and on 
$\fg_\alpha$ by the character $\alpha\in Q=X(T)$. Let $G$ be the algebraic 
subgroup of $\GL(\fg)$ generated by $T$ and the $\{U_\alpha:\alpha\in R\}$. Then 
$G$ is a connected reductive group over $k$, $T\subset G$ is a split maximal 
torus, we have the same decomposition of $\fg$, and $G$ has the same root 
datum as we started out with. 

[See SGA 3 for this done more carefully.]




