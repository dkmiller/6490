% !TEX root = 6490.tex

\section{Reductive groups}

Let $k$ be an algebraically closed field of characteristic zero. Let 
$G_{/k}$ be a reductive group, i.e.~$\urad G=1$. A good example is $\GL(n)$. 
We could like to determine $G$ up to isomorphism via some combinatorial data. 
Let $T\subset G$ be a maximal torus. Better, let $G_{/k}$ be a split reductive 
group and $k$ arbitrary of characteristic zero. Let $X=\characters^\ast(T)$. 
We have the adjoint representation $\adjoint:G\to \GL(\fg)$. Just as before, we 
can write 
\[
  \fg = \fg_0\oplus \bigoplus_{\alpha\in R} \fg_\alpha ,
\]
where $\fg_\alpha=\{x\in \fg:t x=\alpha(t) x\text{ for all }t\}$ and 
$R=\{\alpha\in X\smallsetminus 0:\fg_\alpha\ne 0\}$. 

Note that $R\subset X_\dR$ need not be a root system, since $R$ may only span a 
proper subspace of $X_\dR$. For example, if $G$ is itself a torus, then 
$R=\varnothing$. However, $R\subset V\subset X_\dR$, where $V=\dR\cdot R$, is a 
root system, and it determines the semisimple group $G/\rad G$ up to isogeny. 
The datum ``$R\subset X$'' is not in general enough to determine $G$. 

Let $\characters_\ast(T)=\hom(\Gm,T)$. There is a perfect pairing (composition) 
$\characters^\ast(T)\times \characters_\ast(T)\to \dZ=\hom(\Gm,\Gm)$. Here 
$\langle \alpha,\beta\rangle$ is the integer $n$ such that 
$\alpha\circ\beta$ is $(-)^n$. From $G$ we'll construct a root datum, which 
will be an ordered quadruple 
$(\characters^\ast(T),\roots(G,T),\characters_\ast(T),\check\roots(G,T))$. 
Before doing this, we'll define root data in general. 





\subsection{Root data}\label{sec:root-data}

The following definitions are from \cite[XXI]{sga3-iii}. A \emph{dual pair} 
is an ordered pair $(X,\check X)$, where $X$ and $\check X$ are finitely 
generated free abelian groups, together with a pairing 
$\langle\cdot,\cdot\rangle:X\times \check X\to \dZ$ that induces an isomorphism 
$\check X\simeq X^\vee=\hom(X,\dZ)$. Let $(X,\check X)$ be a dual pair, and 
suppose we have two elements $\alpha\in X$, $\check\alpha\in \check X$. Define 
the \emph{reflections} $s_\alpha$ and $s_{\check\alpha}$ by 
\begin{align*}
  s_\alpha(x) &= x-\langle x,\check\alpha\rangle\alpha \\
  s_{\check\alpha}(x) &= x-\langle\alpha,x\rangle\check\alpha .
\end{align*}

\begin{definition}
A \emph{root datum} consists of an ordered quadruple $(X,R,\check X,\check R)$ , 
such that 
\begin{itemize}
  \item $(X,\check X)$ is a dual pair. 
  \item $R\subset X$ and $\check R\subset \check X$ are finite sets, 
  \item There is a specified mapping $\alpha\mapsto \check\alpha$ from $R$ to 
    $\check R$. 
\end{itemize}
These data are required to satisfy the following conditions:
\begin{enumerate}
  \item For each $\alpha\in R$, $\langle\alpha,\check\alpha\rangle=2$. 
  \item For each $\alpha\in R$, $s_\alpha(R)\subset R$ and 
    $s_{\check\alpha}\subset \check R$. 
\end{enumerate}
\end{definition}

It turns out that $\alpha\mapsto \check\alpha$ is a bijection, and that 
$R$ and $\check R$ are closed under negation. 




